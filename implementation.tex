\chapter{Implementation Aspects}\label{ch:imple}
In this chapter, we discuss how the encoder and decoder are implemented. We start with describing the encoder's implementation and reasons for the chosen method. Then we discuss the implementation of the decoder and the implications of different implementations on the execution complexity.
\section{\acrlong{vn} Storage Memory Format}
The hard values of the \glspl{vn} or the bits in the codeword can be stored in two different ways:
\begin{enumerate}
  \item Byte variable for a bit: In x86 architecture, the smallest addressable memory unit is a byte. Each byte of memory contains eight bits out of which the \gls{lsb} represents one \gls{vn}. An example of such a storage scheme is shown in Figure~\ref{fig:bit_byte}. This form of storage allows us to directly access each bit as \texttt{uint8} and use it to perform signal processing as shown in Figure~\ref{fig:bit_byte_sp}.
  \begin{figure}[htbp]
    \centering
    \tikzsetnextfilename{bit_byte}
    \includegraphics[width=\textwidth, height=\textwidth]{graphics/bit_byte}
    \caption[Illustration of Byte-for-a-bit storage format.]{Three bits of value 0, 1 and 1 are stored in single byte each. The arrow indicates the position of \gls{lsb} where the bit is stored in each byte. The bits marked with x are unused. Note that the bytes are represented in little-endian format.}
    \label{fig:bit_byte}
  \end{figure}
  \begin{figure}[htbp]
    \centering
    \tikzsetnextfilename{bit_byte_sp}
    \includegraphics[width=\textwidth, height=\textwidth]{graphics/bit_byte_sp}
    \caption[Usage of bits stored in Byte-for-a-bit format.]{Signal processing blocks directly access and use use the bits.}
    \label{fig:bit_byte_sp}
  \end{figure}
  \item Packed byte of bits: Each byte of memory contains eight bits representing eight \glspl{vn}. An example of such storage is shown in Figure~\ref{fig:packed_bits}. With this form of storage, additional functions are required to access and store each bit from and to its corresponding position because the minimum quantity of bits that can be accessed from the memory at once is a byte or \texttt{char} or \texttt{uint8}. This is shown in figure~\ref{fig:packed_spb}.
  \begin{figure}[htbp]
    \centering
    \tikzsetnextfilename{packed_bits}
    \includegraphics[width=\textwidth, height=\textwidth]{graphics/packed_bits}
    \caption[Illustration of Packed-byte-of-bits storage format.]{Packed bits of bytes.}
    \label{fig:packed_bits}
  \end{figure}
  \begin{figure}[htbp]
    \centering
    \tikzsetnextfilename{packed_bits_sp}
    \includegraphics[width=\textwidth, height=\textwidth]{graphics/packed_bits_sp}
    \caption[Usage of bits stored in Packed-byte-of-bits format.]{Access and use of bits using helper functions. The double arrow indicates that the Store Bit function reads and writes to the output memory.}
    \label{fig:packed_spb}
  \end{figure}
\end{enumerate}
In our implementations, we use the packed-byte-of-bits format. This format reduces the memory requirements for storing information and codeword bits by a factor of eight. The algorithms for \emph{Load Bit} an \emph{Store Bit} are shown in Algorithm~\ref{alg:load} and \ref{alg:store}, respectively. Table~\ref{tab:mem_req} shows the memory required to store the hard values of \glspl{vn}.
\begin{table}[htbp]
  \centering
  \begin{tabular}{|l|c|}
    \hline
    \textbf{Storage Format} &\textbf{Memory Required (Bytes)}\\
    \hline
    \hline
    Byte for a bit &$n_m$\\
    \hline
    Packed byte of bits &$\ceil{\frac{n_m}{8}}$\\
    \hline
  \end{tabular}
  \caption{Memory required to store hard values of \glspl{vn}.}
  \label{tab:mem_req}
\end{table}

\begin{algorithm}[H]
  \KwData{$j$ (Bit index in buffer)}
  \KwResult{$v$ (Corresponding bit value as \texttt{uint8})}
  $i$ (Calculate byte index from $j$)\;
  $b$ (Calculate bit index within the byte)\;
  $v$ (Read the $i$-th byte from buffer)\;
  Left shift $v$ by $(8-b)$ bits\;
  Right shift $v$ by $7$\;
  \caption{Load Bit}
  \label{alg:load}
\end{algorithm}

\begin{algorithm}[H]
  \KwData{$j$ (Bit index in buffer), $c$ (Bit value as \texttt{uint8})}
  \KwResult{$c$ stored in the right bot position of the buffer}
  $i$ (Calculate byte index from $j$)\;
  $b$ (Calculate bit index within the byte)\;
  $v$ (Read the $i$-th byte from buffer)\;
  \eIf{$c$ equal to 1}{
    $v$ bitwise OR with 1 left shifted by $b$ bits\;
  }{
    $z$ (NOT of 1 left shifted by $b$ bits)\;
    $v$ bitwise AND with $z$\;
  }
  Store the resulting byte in $i$-th byte position in buffer\;
  \caption{Store Bit}
  \label{alg:store}
\end{algorithm}

However, the gain in minimum memory requirement comes with a cost of increased execution complexity. The \emph{Load Bit} blocks performs memory-read and bit-shift operations. The \emph{Store Bit} block performs some arithmetic and logical operations along with memory-read and write operations. Table~\ref{tab:load_store} shows the operations performed in each call of \emph{Load Bit} and \emph{Store Bit}. Note that this is just one way of implementing the functions.
\begin{table}[htbp]
  \centering
  \begin{tabular}{|l|c|c|c|c|c|}
    \hline
    \textbf{Function called} &\textbf{Reads} &\textbf{Writes} &\textbf{Bit Shift} &\textbf{Arithmetic} &\textbf{Logical}\\
    \hline
    \hline
    Load Bit &1 &0 &2 &2 &0\\
    \hline
    Store Bit &1 &1 &1 &2 &2\\
    \hline
  \end{tabular}
  \caption[Operations performed for different storage formats.]{Number of different operations performed during each call of \emph{Load Bit} and \emph{Store Bit}.}
  \label{tab:load_store}
\end{table}

Table~\ref{tab:summ_mem_req} compares the memory requirement and operations required to encode a codeword with $n_m$ information bits with $R_\infty=1/2$.
\begin{table}[htbp]
  \centering
  \begin{tabular}{|p{3cm}|>{\centering\arraybackslash}p{3cm}|c|c|c|c|c|}
    \hline
    \textbf{Storage Format} &\textbf{Memory Required (Bytes)} &\textbf{Reads} &\textbf{Writes} &\textbf{Bit Shift} &\textbf{Arithmetic} &\textbf{Logical}\\
    \hline
    \hline
    Byte for a bit &$3n_m$ &$n_m$ &$\frac{n_m}{n-1}$ &0 &0 &0\\
    \hline
    Packed byte of bits &$\ceil{\frac{3n_m}{8}}$ &$n_m+\frac{n_m}{n-1}$ &$\frac{n_m}{n-1}$ &$2n_m+\frac{n_m}{n-1}$ &$2\left(n_m+\frac{n_m}{n-1}\right)$ &$2\frac{n_m}{n-1}$\\
    \hline
  \end{tabular}
  \caption[Memory and complexity of different storage formats.]{Memory requirement and operations required to encode a codeword with $n_m$ information bits and asymptotic code rate $R_\infty$. We assume that each bits in the output buffer are accessed only once.}
  \label{tab:summ_mem_req}
\end{table}

Although the execution complexity of using Packed-byte-of-bits format is high compared to the other format, the Packed-byte-of-bits format's complexity is very negligible compared to the overall decoding complexity. Hence, the Packed-byte-of-bits format is chosen in our implementation to reduce the memory requirement. Standard tools such as MATLAB\textsuperscript{TM} uses 64-bit double-precision floating-point storage for all variable by default.

\section{\acrlong{vn} Indexing}
The \gls{llr} values of \glspl{vn} are stored in an array of memory where each element is a 32-bit floating point value. During each \gls{cn} update, the \gls{bp} algorithm selects the corresponding \gls{vn}'s \gls{llr} to perform the updates. This selection of \glspl{vn} is done via indexing. There are two ways to perform the selection:
\begin{enumerate}
  \item \emph{Method 1}: During each \gls{cn} update, compute the indices of the corresponding \glspl{vn} so that the \gls{bp} algorithm selects them.
  \item \emph{Method 2}: Before start of decoding, compute and store the edge indices for all \glspl{cn} in the \gls{pcm}.
\end{enumerate}

Method 1 computes the indices on the fly during each \gls{cn} update. The number of \glspl{cn} depends on $n_m$. Thus, the complexity of calculating the indices increases linearly with $I$ and $n_m$. On the other hand, method 2 does not increase the index computation complexity with increasing $I$ or $n_m$. However, the memory required to store the indices of all \glspl{cn} increases linearly with increasing number of \glspl{cn}. Each index is stored in a \texttt{uint32}. Table~\ref{tab:idx_com} shows the memory and computation requirements for indices calculation in method 1 and method 2. We chose method 1 in our implementations as it requires less memory for the indices. The computation complexity in calculating the indices is very negligible compared to the decoding complexity.
\begin{table}[htbp]
  \centering
  \begin{tabular}{|l|c|c|}
    \hline
    \textbf{Method} &\textbf{Memory Required (Bytes)} &\textbf{\gls{cn} Index Calculations}\\
    \hline
    \hline
    Method 1 &$12n$ &$n_c I$\\
    \hline
    Method 2 &$12n_cn$ &$n_c$\\
    \hline
  \end{tabular}
  \caption[Memory and complexity of different \acrshort{vn} indexing technieqes.]{Memory requirement and computations required to calculate and store indices for decoding a codeword with $n_m$ information bits and $I$ iterations. Let the number of \glspl{cn} be $n_c$ and size of \texttt{uint32} is 4 bytes.}
  \label{tab:idx_com}
\end{table}
 