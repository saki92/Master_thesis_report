\documentclass[11pt,type=msc,colorback,accentcolor=tud1b,bigchapter,firstlineindent,longdoc,bibliography=totoc,listof=totoc]{tudthesis}
%\usepackage{ngerman}
\usepackage[english]{babel}
%\usepackage{amsfonts}
\usepackage{hyperref}
\usepackage{eucal}
\usepackage{amsmath}
\usepackage{setspace}
\usepackage{tikz,pgfplots}
\usetikzlibrary{positioning,arrows,external,patterns}
\usepackage{tikzscale}
\usepackage{etoolbox}
\usepackage{float}
%\usepackage{fontspec}
\usepackage[T1]{fontenc}
\usepackage{textcomp}
\usepackage[utf8x]{inputenc}
\usepackage{titlesec}
\usepackage{graphicx}
\usepackage{pdfpages}
\usetikzlibrary{calc}
\usetikzlibrary{fit,backgrounds,automata}
\usepackage[a-1b]{pdfx}
%\usepackage[bookmarks]{hyperref}
\usepgfplotslibrary{external} 
\tikzexternalize[prefix=TikzPictures/]

%\usepackage{breakurl}
\usepackage[acronym,toc,nonumberlist]{glossaries}
\newglossary[sylg]{symbolslist}{syi}{syo}{List of Symbols}
\makeglossaries
\begin{acronym}
  \acro{fec}[FEC]{Forward Error Correction}
  \acro{ldpc}[LDPC]{Low-Density Parity-Check}
  \acro{ldpcbc}[LDPC-BC]{Low-Density Parity-Check Block Code}
  \acro{ldpccc}[LDPC-CC]{Low-Density Parity-Check Convolutional Code}
  \acro{scldpc}[SC-LDPC]{Spatially-Coupled LDPC}
  \acro{pcm}[PCM]{Parity Check Matrix}
  \acro{cn}[CN]{Check Node}
  \acro{vn}[VN]{Vriable Node}
  \acro{bp}[BP]{Belief-Propagation}
  \acro{cb}[CB]{Code Block}
  \acro{bpl}[BPL]{Broadband Power Line}
  \acro{bc}[BC]{Block Code}
  \acro{gm}[GM]{Generator Matrix}
  \acro{cw}[CW]{codeword}
  \acro{awgn}[AWGN]{Additive White Gaussian Noise}
  \acro{snr}[SNR]{Signal-to-Noise Ratio}
  \acro{phy}[PHY]{physical}
  \acro{map}[MAP]{Maximum a posteriori}
  \acro{ml}[ML]{Maximum-Likelihood}
  \acro{map}[MAP]{Maximum-A-Posteriore}
  \acro{mpa}[MPA]{Message-Passing Algorithm}
  \acro{spa}[SPA]{sum-product algorithm}
  \acro{llr}[LLR]{Log-likelihood ratio}
  \acro{v2c}[V2C]{variable-node-to-check-node}
  \acro{c2v}[C2V]{check-node-to-variable-node}
  \acro{msa}[MSA]{Min-Sum Algorithm}
  \acro{wd}[WD]{Windowed Decoding}
  \acro{qam}[QAM]{Quadrature Amplitude Modulation}
  \acro{qpsk}[QPSK]{Quadrature Phase Shift Keying}
  \acro{pdf}[pdf]{probability density function}
  \acro{ber}[BER]{Bit-Error Rate}
  \acro{bler}[BLER]{Block-Error Rate}
  \acro{lsb}[LSB]{Least Significant Bit}
  \acro{osi}[OSI]{Open System Interconnection}
  \acro{fir}[FIR]{Finite Impulse Response}
  \acro{iir}[IIR]{Infinite Impulse Response}
  \acro{bd}[BD]{Base Decoder}
  \acro{id}[ID]{Improved Decoder}
  \acro{iesc}[IESC]{Improved-Early-Success-Criterion}
  \acro{aneu}[ANEU]{Average Number of Edge Updates}
  \acro{lrl}[LRL]{Left-Right-Left}
  \acro{app}[APP]{A-Posteriori Probability}
  \acro{bpsk}[BPSK]{Binary Phase Shift Keying}
\end{acronym}
%\tikzexternalize[mode=list and make]{Report}
\newcommand{\getmydate}{%
	\ifcase\month%
	\or Januar\or Februar\or M\"arz%
	\or April\or Mai\or Juni\or Juli%
	\or August\or September\or Oktober%
	\or November\or Dezember%
	\fi\ \number\year%
}

\renewcommand{\H}{\mathbf{H}}
\renewcommand{\G}{\mathbf{G}}
\newcommand{\x}{\mathbf{x}}
\newcommand{\m}{\mathbf{m}}
\newcommand{\bb}{\mathbf{b}}
\newcommand{\cnval}{c}
\newcommand{\vnval}{v}
\newcommand{\sdot}{}
\newcommand{\xs}{0.0001}
\newcommand*{\pdf}{p}
\newcommand*{\prob}{\text{Pr}}
\newcommand{\mc}{\mathcal{c}}
\DeclareMathOperator*{\argmax}{argmax}
\DeclareMathOperator{\arctanh}{atanh}
\DeclareMathOperator{\re}{Re}
\DeclareMathOperator{\im}{Im}
\DeclareMathOperator{\sign}{sign}
\renewcommand{\arraystretch}{1.5} %for table row spacing
% General scaling (fiddle with value to find ideal setting)
\newcommand{\plotscale}{1}

\newcommand{\splotscale}{0.7}

% Golden ratio as axes aspect ratio
\newcommand{\plotratio}{0.618}

% Define line width within plots
\newlength{\mylinewidth}
\setlength{\mylinewidth}{0.6pt}

% Define font size within plots
\newcommand{\plotfontsize}{\footnotesize}

% Apply settings to all figures
\tikzset
{
  ultra thin/.style= {line width=0.25\mylinewidth},
  very thin/.style=  {line width=0.5\mylinewidth},
  thin/.style=       {line width=\mylinewidth},
  semithick/.style=  {line width=1.5\mylinewidth},
  thick/.style=      {line width=2\mylinewidth},
  very thick/.style= {line width=3\mylinewidth},
  ultra thick/.style={line width=4\mylinewidth},
  every picture/.style={thin},
  every node/.append style={font=\plotfontsize}
}

% Mostly for changing the minor-grid-line widths
\pgfplotsset
{
  compat=newest,
  minor grid style={black!10},
  every axis/.append style=
  {
    label style={font=\plotfontsize},
    tick label style={font=\plotfontsize},
    legend style={font=\plotfontsize}
  }
}

% Line width for dsp figures
\newlength{\tikzthin}
\setlength{\tikzthin}{\mylinewidth}

% For inputting pgf files generated from Octave with matlab2tikz
\usepgfplotslibrary{patchplots}
\pgfplotsset{plot coordinates/math parser=false}
\newlength{\fwidth}
\newlength{\fheight}
\setlength{\fwidth}{\plotscale\linewidth}
\setlength{\fheight}{\plotratio\fwidth}

\newlength{\sfwidth}
\newlength{\sfheight}
\setlength{\sfwidth}{\splotscale\linewidth}
\setlength{\sfheight}{\plotratio\sfwidth}

% Scale unscalable tikz figures
\let\OrgPgfTransformScale\pgftransformscale

% Color definitions for document-wide use
\definecolor{mycolor1}{rgb}{0.00000,0.44700,0.74100}
\definecolor{mycolor2}{rgb}{0.85000,0.32500,0.09800}
\definecolor{mycolor3}{rgb}{0.92900,0.69400,0.12500}
\definecolor{mycolor4}{rgb}{0.49400,0.18400,0.55600}
\definecolor{mycolor5}{rgb}{0.46600,0.67400,0.18800}
\definecolor{mycolor6}{rgb}{0.30100,0.74500,0.93300}
\definecolor{mycolor7}{rgb}{0.63500,0.07800,0.18400}

\newcommand*{\rom}[1]{\uppercase\expandafter{\romannumeral #1\relax}}
\begin{document}
	\colorlet{tudidentbar}{tud1b} 
	\thesistitle{Design and Implementation of Improved Decoding Algorithms for LDPC Convolutional Codes}%
	{Entwurf und Implementierung von verbesserten Decodieralgorithmen f{\"u}r LDPC-Faltungscodes }
	\author{Sakthivel Velumani}
	\birthplace{Salem, Indien}
	\referee{M.Sc. Janik Frenzel}{M.Sc. Bastian Alt}[Prof. Dr. techn. Heinz Koeppl]
	\department{Fachbereich Elektrotechnik und Informationstechnik}
	\group{Bioinspired Communication Systems Lab}
	\date{\today}
	\dateofexam{14.01.2019}{14.01.2019}
	\tuprints{82349}{8234}
	\makethesistitle
  \pagenumbering{Roman}
	%\colorlet{tudidentbar}{tud0b} 
	\affidavit{Sakthivel Velumani}
  \newpage
  \textbf{\large Acknowledgments}\\
\vskip3ex


  %\onehalfspacing
  \chapter*{Abstract}
\addcontentsline{toc}{chapter}{Abstract}
In this work, techniques to improve the performance of window based decoder for \gls{ldpccc} are developed. In order to evaluate the performance of our techniques, we choose the codes from the standard IEEE 1901. Our focus is on terminated \gls{ldpccc} and so we first investigate the termination mechanism in the encoder. We found that a proper end-termination for the IEEE 1901's \glspl{ldpccc} is infeasible. Two improvement techniques for window decoders are developed and evaluated. The first technique focuses on the movement direction of the window in which the window moves forward and backward across the \gls{pcm}. Two different configurations of the first technique are discussed. The second improvement technique focuses on the convergence criterion for the windows and it changes based on window size. The simulations are carried on an \gls{awgn} channel based system model. The first technique is proved to have better decoding performance at a reduced decoding complexity. The second technique has a reduced complexity with about the same performance as the decoder without the technique.
	\tableofcontents
	\newpage
  \listoffigures
  \newpage
  \listoftables
  \newpage

  \newglossarystyle{mystyle}{%
    \renewcommand*{\glspostdescription}{}%
    \renewenvironment{theglossary}{\begin{longtable}[l]{@{}lcl}}{\end{longtable}}
    \renewcommand*{\glossaryheader}{}%
    \renewcommand*{\glsgroupheading}[1]{}%
    \renewcommand*{\glsgroupskip}{}%
    \renewcommand*{\glossaryentryfield}[5]{%
      \glstarget{##1}{##2}	% Name
      & ##4					        % Symbol
      & ##3					        % Description
      % & ##5					      % Page list
      \\% end of row
    }%
    \renewcommand*{\glossarysubentryfield}[6]{%
      \glossaryentryfield{##2}{##4}{##3}
    }%
  }

  % Print acronyms
  \printglossary[type=acronym, title=Acronyms, toctitle=Acronyms, style=mystyle]
  % Print list of symbols
  \printglossary[type=symbolslist, style=mystyle]
  \newpage
  % Save current value of roman page number
  \newcounter{romanpagenumbers}
  \setcounter{romanpagenumbers}{\value{page}}
  \pagenumbering{arabic}
	\section{Introduction}
\begin{itemize}
  \item 2 to 3 sentences about LDPC-CCs and its applications so far.
  \item Why it is a better code than LDPC-BCs.
  \item About latency and efficiency trade-off.
  \item About use of LDPC-BCs in NR and potential LDPC-CCs in future generations.
  \item Existing decoding algorithms.
  \item My decoding improvements.
  \item Finding of termination sequence of BPL codes.
  \item Organization of this report.
  \item Notation.
\end{itemize}
  \chapter{Background}\label{ch:back}
In this chapter, we provide an overview of channel coding and some selected channel codes. We start by discussing the need for channel coding. We then describe linear block codes, \gls{ldpc} codes and \glspl{ldpccc} which are the main focus of this thesis. Then we go on to discuss \gls{bp} and the sliding-window technique used in decoders for \glspl{ldpccc}. We finish the section describing our system model.

\section{Introduction to Channel Coding}
The term \emph{coding} is generally associated with the mapping of information to a set of symbols or numbers~\cite{Bossert}. Source coding aims to remove the redundancy in the information whereas channel coding aims to make the information immune to random distortion. A model of a digital communication system is shown in Figure~\ref{fig:chanCoding}. Let us consider that the \emph{source} block produces a sequence of information bits given by the vector $\gls{mbd}$. These bits might stream from any digital information source such as multimedia files, text documents, etc. These vectors of bits are encoded in the \emph{source-encoder} block to produce reduced set of information bit vectors $\gls{mbm}$ called source codewords. The reduction usually means that the length of $\gls{mbm}$ is at most the length of $\gls{mbd}$. The mapping of information bits to a set of reduced information bits allows unique reconstruction of the information bits at the receiver. The source encoder is chosen depending on the type of the information source~\cite{proak}.

The next block in the digital communication model is the \emph{channel encoder}. Whereas the source encoder compresses the information bit vectors, the channel encoder expands them by adding redundant bits in a structured manner. This structured redundancy makes the transmitted information bits less susceptible to distortions such as interference in the channel noise. A \emph{channel} is a medium through which the information is transferred from transmitter to receiver. A \emph{code} is a set of rules that defines the encoding principle of the encoder. The type of code is chosen depending on the channel and the application requirements. In general, the source encoder or decoder is placed at higher layers of the \gls{osi} model, while channel-coding blocks are placed at the \gls{phy} layer. The outputs of the channel encoder are called channel codewords. The codeword vectors $\gls{mbx}$ are then modulated in the \emph{modulator} block where the bits are transformed into symbol vectors $\gls{mbu}$. The symbol vectors are then transmitted as analog signals through the channel. Due to the addition of interference and noise, the channel output is in general not the same as the channel input: $\gls{mbv}\neq\gls{mbu}$. The \emph{demodulator} converts the received symbol vectors $\gls{mbv}$ into vectors of bits $\gls{mby}$ which corresponds to the vector of encoded bits $\gls{mbx}$. The \emph{channel decoder} uses the redundancy in the received codeword to deduce an estimate $\gls{mhbm}$ of the source codeword. The source decoder then deduces an estimate $\gls{mhbd}$ of the information bit vector from $\gls{mhbm}$.

\begin{figure}[htbp]
  \centering
  \tikzsetnextfilename{channel_coding}
  \includegraphics[width=\textwidth, height=\textwidth]{graphics/channel_coding}
  \caption{Block diagram of a digital communication system.}
  \label{fig:chanCoding}
\end{figure}

The addition of redundant bits by the channel encoder enables the mapping between a set of information words and a set of all possible \emph{receive words}. Let us assume the length of an information word $\gls{mbm}$ to be $k$ bits and the length of a codeword $\gls{mbx}$ to be $n$ bits such that $n>k$. Thus, the information word set has $2^k$ words and the receive word set has $2^n$ words. The codeword set of size $2^k$ is a subset of the receive word set. The mapping between different set sizes allows us to detect if the received word is in the codeword set. The information words and codewords contain elements from the binary set $\mathbb{F}_2=\{0,1\}$. $\mathbb{F}_2$ or GF(2) is called a finite field of order $2$. Hence, all arithmetic operations with information bits and codewords are performed modulo 2.

\section{Channel Codes}
There are different types of channel codes. The choice of one depends on the application requirements, type of channel medium and resource availability. In this thesis, we focus on \glspl{ldpccc}, a special class of \gls{ldpc} codes.
\subsection{Linear Block Codes}
Linear block codes are codes in which a codeword is formed by a linear combination of two or more base vectors that span the codeword space~\cite{proak}. Hence, the base vectors are also codewords. As a result, a linear combination of any two or more codewords forms another codeword. The codeword space of $2^k$ vectors is a subspace of the space of all $2^n$ vectors. An $(n,k)$ linear block code maps $k$ message bits to $n$ codeword bits. The remaining $n-k$ redundant bits are called parity bits and they are determined by an encoding rule. Linear block codes are classified into two categories: \emph{systematic} and \emph{non-systematic}. Systematic codes have all their message bits transmitted in an unaltered manner whereas the non-systematic codes do not have such formation. Without loss of generality, we assume that a codeword of a systematic linear block code has the following structure: 
\begin{align}
\x^T=
\begin{bmatrix}
\m^T &\gls{mbb}^T
\end{bmatrix}
\end{align}
where $\x\in\mathbb{F}_2^{n\times 1}$ is the codeword vector, $\m\in\mathbb{F}_2^{k\times 1}$ and $\bb\in\mathbb{F}_2^{(n-k)\times 1}$ denote message and parity vectors, respectively. The code rate is given by \begin{align}\gls{mR}=\frac{k}{n}.\end{align}

Codewords of linear block codes are expressed as \begin{align}\x=\G\odot\m\end{align} where $\G\in\mathbb{F}_2^{n\times k}$ is the \gls{gm} and $\odot$ represents multiplication modulo 2. A parity check is described by the expression \begin{align}\mathbf{s}=\H\odot\x\end{align} where $\H\in\mathbb{F}_2^{(n-k)\times n}$ is called the \gls{pcm} and $\mathbf{s}\in\mathbb{F}_2^{(n-k)\times 1}$ is called the syndrome. Each row of the \gls{pcm} represents a parity-check equation. Only when $\mathbf{s}=\mathbf{0}$, the parity checks are fulfilled. The relation between \gls{pcm} and \gls{gm} is given by $\gls{mbH}\odot\gls{mbG}=\mathbf{0}$. With either \gls{gm} or \gls{pcm} given, the other one is not unique. For example, if the \gls{pcm} of a $(7,4)$ hamming code is given by
\begin{align} \label{eq:H_ham}
\H =
\begin{bmatrix}
1 &1 &1 &0 &1 &0 &0 \\
1 &1 &0 &1 &0 &1 &0 \\
1 &0 &1 &1 &0 &0 &1
\end{bmatrix},
\end{align}
then the \gls{gm} can be formed by combining any 3 rows of null$(\H)$, i.e., the right null space of $\H$.
\begin{align} \label{eq:G_ham}
\G^T =
\begin{bmatrix}
1 &1 &0 &0 &0 &0 &1 \\
1 &1 &1 &0 &1 &0 &0 \\
0 &0 &1 &1 &1 &1 &0 \\
0 &1 &1 &1 &0 &0 &0
\end{bmatrix}
\end{align}

Figure~\ref{fig:intro_ber} shows a simple comparison of \gls{ber} between uncoded transmission, transmission using a $(7,4)$ hamming code and transmission using a \gls{ldpc} with about the same rate as the hamming code.
\begin{figure}[htbp]
  \centering
  \tikzsetnextfilename{intro_ber}
  \includegraphics[width=0.9\linewidth]{plots/intro_ber}
  \caption{Probability of error for uncoded, hamming coded and \gls{ldpc} coded transmissions.}
  \label{fig:intro_ber}
\end{figure}

The \gls{pcm} can be represented by a bipartite graph called Tanner graph~\cite{Tanner1981}. The Tanner graph has two sets of nodes: \glspl{vn} represent columns and \glspl{cn} represent rows of the \gls{pcm}. Each non-zero entry in the \gls{pcm} is represented by an edge between the respective \gls{vn} and \gls{cn}. The \emph{degree} of a node is the number of edges connected to it. The Tanner graph of the example \gls{pcm} in (\ref{eq:H_ham}) is shown in Figure \ref{fig:tannGraph}.

\begin{figure}[htbp]
  \centering
  \tikzsetnextfilename{tanner_graph}
  \includegraphics[width=0.5\textwidth, height=0.2\textwidth]{graphics/tanner_graph}
  \caption{Tanner graph of the code from (\ref{eq:H_ham}). The dark shaded circles represents \glspl{vn} and the crossed circles \glspl{cn}. All \glspl{cn} have degree 3 whereas the \gls{vn} degrees vary between 1 and 3.}
  \label{fig:tannGraph}
\end{figure}

\subsection{\acrlong{ldpc} Block Codes}
\glspl{ldpcbc} are a class of linear block codes which were introduced by Robert Gallager in 1963~\cite{Gallager1963}. As the name specifies, they are defined by a sparse \gls{pcm} containing mostly 0's and relatively few 1's. The sparsity of the \gls{pcm} or its Tanner graph is a key property that allows for the algorithmic efficiency of decoding \glspl{ldpcbc}. These codes are divided into two types: regular and irregular codes.

In a regular $(n,q,r)$ code, all \glspl{vn} have degree \gls{mq} and all \glspl{cn} have degree \gls{mr}.

\subsection{Convolutional Codes}
Convolutional codes in general, are codes in which the parity bits are generated by convolving information bits or information and parity bits. The polynomial coefficients for the convolution is given by the generator polynomial. This generator polynomial is also the taps of a \gls{fir} filter in case of non-recursive codes and an \gls{iir} filter in case of recursive codes. An example of parity-bit generation in a non-recursive systematic convolutional code is shown in Figure~\ref{fig:conv_code}.
\begin{figure}[htbp]
  \centering
  \tikzsetnextfilename{conv_code}
  \includegraphics[width=0.5\textwidth, height=0.2\textwidth]{graphics/conv_code}
  \caption{Example of a non-recursive convolutional code with asymptotic rate $R_\infty=1/2$. $x[i]$ is the input and $y[i]$ is the output.}
  \label{fig:conv_code}
\end{figure}

The generator polynomial of this example is given by
\begin{align}
G^{(0)}(D)&=1\\
G^{(1)}(D)&=1+D^2.
\end{align}
The impulse response of the parity-bit generator is given by
\begin{align}g^{(1)}[i]=\begin{bmatrix}
1 &0 &1
\end{bmatrix}.\end{align}
The output is given by the convolution form:
\begin{align}
y^{(1)}[i]&=x[i]*g^{(1)}[i]\label{eq:conv_conv}\\
&=\sum_{l=0}^{2}x[l]g^{(1)}[i-l].
\end{align}
The \emph{constraint length} of a convolutional code is $l_c=m_s+1$ where $m_s$ is the largest degree in $g[i]$. In the example in Figure~\ref{fig:conv_code}, $l_c=3$.

\subsection{\acrlong{ldpc} Convolutional Codes}
\glspl{ldpccc} or \gls{scldpc} codes are formed by imposing the above mentioned convolutional structure on \glspl{ldpcbc}. They were invented by Alberto Felstr{\"o}m and Kamil Zigangirov~\cite{Felstrom1999}. These codes are characterized by a sparse infinite-length \gls{pcm} which has a diagonal structure. The \gls{pcm} of these codes is constructed by coupling \glspl{pcm} of \glspl{ldpcbc} as given by

\begin{align}\label{eq:H_infty}
\H_{[-\infty,\infty]} =& 
\begin{bmatrix}
  \ddots &\ddots &\ddots &\ddots\\
  &\H_{m_s}(t-1) &\dots &\H_1(t-1) &\H_0(t-1)\\
  & &\H_{m_s}(t) &\dots &\H_1(t) &\H_0(t)\\
  & & &\H_{m_s}(t+1) &\dots &\H_1(t+1) &\H_0(t+1)\\
  & & & &\ddots &\ddots &\ddots &\ddots
  \end{bmatrix}
  \begin{matrix}
  \mathbf{s}(t-1)\\
  \mathbf{s}(t)\\
  \mathbf{s}(t+1)\\
  \end{matrix}
\end{align}
where the $\H_\mu(t)\in\mathbb{F}_2^{(n-k)\times n},\mu=0,\dots,m_s$ are \glspl{pcm} of different \glspl{ldpcbc} of rate $R_\infty=\gls{mk}/\gls{mn}$ for different time instances and $m_s$ is the memory of the code. Hence, the asymptotic rate of the resulting \gls{ldpccc} is $\gls{mRi}=k/n$. $\gls{mbs}(t)\in\mathbb{F}_2^{(n-k)\times 1}$ denotes the syndromes resulting from the parity check equations. The codewords of such a code have the form 
\begin{align}
\x^T=
\begin{bmatrix}
\dots &\x(t-1)^T &\x(t)^T &\x(t+1)^T &\dots
\end{bmatrix}
\end{align}
where each $\x(t)\in\mathbb{F}_2^{n\times 1}$. Given the \gls{pcm} $\H$ and a valid codeword $\x$, the following expression holds:
\begin{align}\label{eq:ldpccc_conv}
\mathbf{s}(t)=\sum_{\tau=0}^{m_s}\H_\tau(t)\x(t-\tau)\mod 2.
\end{align}
The equation (\ref{eq:ldpccc_conv}) is a convolution representing the convolutional structure of $\H$ in (\ref{eq:H_infty}). Similar to equation (\ref{eq:conv_conv}) for convolutional codes.

The bits in the codeword $\x$ are coupled together over a distance called the \emph{constraint length} which is given by $l_c=(m_s+1)n$ bits.

\subsection{Termination of Convolutional Codes}
In general, \glspl{ldpccc} have codewords and \glspl{pcm} of infinite length. For packet-based communication networks, however, the whole packet has to be retransmitted in case of incorrect information bits in higher layers. Also, in a wireless medium the channel parameters change over time which requires the encoder to change its code rate on the fly. For the aforementioned reasons, terminated codes are a better choice.

Termination is the process of limiting the coupling length, so that the codewords have finite length. This allows the decoder to stop decoding the current received word if a bit cannot be corrected, thus reducing the decoding complexity. The termination process requires appending \emph{termination bits} to the end of the codeword to ensure that the last $m_s$ parity-check equations of the terminated \gls{pcm} are fulfilled. Termination also ensures that the encoder returns to an all-zero state before encoding the next codeword. For recursive convolutional codes, the termination bits are determined by solving a system of linear equations in $\mathbb{F}_2$. Whereas for non-recursive convolutional codes, appending a series of bits with value zero is sufficient.

Termination introduces a rate loss because the termination bits do not contain any information. Hence, for the rate calculation of a terminated \gls{ldpccc}, the termination bits are not taken into account. However, the rate loss is compensated by an increase in decoding performance as the termination reduces the \gls{cn} degrees at the end of the codeword and smaller \gls{cn} degrees are better.

The \gls{pcm} of a terminated \gls{ldpccc} is a sub-matrix of the infinitely long \gls{pcm} of the code (\ref{eq:H_infty}). The terminated \gls{pcm} has a structure as given by

\begin{align}
\H_L = 
\overbrace{\begin{bmatrix}
  \H_{0}(0)\\
  \H_1(1) &\H_0(1)\\
  \vdots &\H_1(2) &\ddots\\
  \H_{m_s}(m_s) &\vdots &\ddots &\H_0(L-1)\\
  &\H_{m_s}(m_s+1) &\ddots &\H_{1}(L)\\
  & & &\vdots\\
  & & &\H_{m_s}(L+m_s)
\end{bmatrix}}^{Ln}
\left.\begin{matrix}
\\
\\
\\
\\
\\
\\
\\
\\
\end{matrix}\right\}(L+m_s)(n-k)
\end{align}
where \gls{mL} is the \emph{coupling length} denoting the number of \glspl{cb} in the codeword. Each \gls{cb} contains $n$ bits. Hence, the total length of the terminated codeword is $n_L=Ln$ bits. The effect of termination in the Tanner graph of a $R_\infty=1/2$ code is shown in Figure~\ref{fig:tannGraphLdpccc}.

\begin{figure}[htbp]
  \centering
  \tikzsetnextfilename{tanner_graph_ldpccc}
  \includegraphics[width=0.5\textwidth, height=0.2\textwidth]{graphics/tanner_graph_ldpccc}
  \caption{Tanner graph of a terminated \gls{ldpccc}. The dark circles and lines are the \glspl{vn} and edges of terminated code, the light circles and dashed lines are the omitted \glspl{vn} and edges as a result of termination.}
  \label{fig:tannGraphLdpccc}
\end{figure}

The entire graph in Figure~\ref{fig:tannGraphLdpccc} can be seen as a Tanner graph of an infinitely long \gls{ldpccc}. As a result of termination, only the center part of the graph remains. The dark circles are the \glspl{vn} of the terminated \gls{ldpccc} and the solid lines are their corresponding edges.

\subsection{\acrlong{ldpccc} Used in IEEE 1901}\label{sec:bpl_bg}
In this thesis, we use the \glspl{ldpccc} specified in the \gls{bpl} or IEEE 1901 standard to evaluate our decoder~\cite{Bpl}. From now on, we refer to the \glspl{ldpccc} in the IEEE 1901 standard as \emph{\gls{bpl}~codes}. The \gls{bpl} codes are specified as sets of parity-check polynomials for all asymptotic rates $R_\infty=k/n,\ n\in\{2,3,4,5\}$ where $k=n-1$. In other words, the \gls{bpl} codes have only one parity bit in each \gls{cb}. 

The codes are defined as
\begin{align}\label{eq:bpl_poly}
\sum_{i=1}^{k}A_{i,\tau}(D)M_i(D)+\sum_{i=1}^{n-k}C_{i,\tau}(D)B_i(D)\equiv 0\mod 2
\end{align}
where $k$ is the number of message bits in each \gls{cb}, $\tau \in \{0,\dots,T-1\}$ is the phase of the code that is given by $\tau=(t\ \text{mod}\ T)$, $T$ is the periodicity of the codes, $M_i(D),i=1,\dots,k$ represents message bits and $B_i(D),i=1,\dots,n-k$ represents parity bits. $A_{i,\tau}$ and $C_{i,\tau}$ define the connections between the bits based on delay $D$.

The memory $m_s$ of the code is
\begin{align}
m_s=\max\left(\{\deg(A_{i,\tau}(D)):i=1,\dots,k;\forall\tau\}\cup\{\deg(C_{i,\tau}(D)):i=1,\dots,n-k;\forall\tau\}\right)
\end{align}
where $\tau \in \{0,\dots,T-1\}$ and deg$(f(x))$ denotes the set of all degrees of $x$ in $f(x)$.

The \gls{bpl} codes are periodic with $T=3$. Periodic codes have time-varying parity-check polynomials which repeat every $T$ \glspl{cb}. For illustration, the parity-check polynomial of the \gls{bpl} code for $R_\infty=2/3$ and $\tau=0$ is given by
\begin{align}
&(D^{214}+D^{185}+1)M_1(D)+(D^{194}+D^{67}+1)M_2(D)+(D^{215}+D^{145}+1)B(D)=0\mod 2,
\end{align}
for $\tau=1$ as
\begin{align}
&(D^{160}+D^{62}+1)M_1(D)+(D^{226}+D^{209}+1)M_2(D)+(D^{206}+D^{127}+1)B(D)=0\mod 2,
\end{align}
and for $\tau=2$ as
\begin{align}
&(D^{196}+D^{143}+1)M_1(D)+(D^{115}+D^{104}+1)M_2(D)+(D^{211}+D^{119}+1)B(D)=0\mod 2.
\end{align}
$m_s=215$ for $n=2$ and $m_s=226$ for $n\neq2$. All $A_{i,\tau}$ and $C_{i,\tau}$ have three taps for each bit in the \gls{cb}. There is a maximum of $3n$ taps or edges per \gls{cn}.

\cite{Bpl} specifies that termination is achieved by appending bits with value $0$ to the end of the message bits before encoding. These bits are called \emph{zero-tail bits}. The number of zero-tail bits $\gls{mnz}$ depends on the number of message bits $\gls{mnm}$ in the codeword and the asymptotic rate $R_\infty$. The number of \glspl{cb} in the terminated codeword is \begin{align}L=\frac{n_m+n_z}{n-1}.\end{align} Since the zero-tail bits are known at the receiver, they are not transmitted. Only the parity bits generated from the zero-tail bits are transmitted. Hence, the actual rate of the terminated \gls{bpl} code is
\begin{align}\label{eq:rate_term}
R_L=\frac{n_m}{Ln-n_z}.
\end{align}
The relation between $R_L$ and $R_\infty$ is given by rearranging (\ref{eq:rate_term})
\begin{align}
R_L=\nu R_\infty
\end{align}
where $\nu=\frac{Ln}{Ln-n_z}$ and $R_\infty=\frac{n_m}{Ln}=\frac{k}{n}$. Hence,
\begin{align}
\lim_{L\to\infty}R_L=R_\infty.
\end{align}
Encoding and termination of \gls{bpl} codes is explained in detail in Chapter~\ref{ch:encode}.

\section{Decoding of \acrlong{ldpc} Codes}\label{sec:decoding_ldpc}
A channel decoder attempts to find the transmitted codeword $\x$ from the received word $\gls{mby}$. The best decoder in terms of performance is a \gls{map} decoder. Its complexity grows exponentially with the information word length because it finds---among all possible codewords---the codeword that has the highest probability given the received word. The estimate of the transmitted codeword from a \gls{map} decoder is given by
\begin{align}
\gls{mhbx}&=\argmax_{\x_i}\pdf_{X\mid Y}\left(\x_i,\gls{mby}\right) \nonumber\\
&=\argmax_{\x_i}\frac{\pdf_{Y\mid X}\left(\gls{mby},\x_i\right)\prob(\x_i)}{\prob(\gls{mby})} \nonumber\\
&=\argmax_{\x_i}\gls{mpdf}_{Y\mid X}\left(\gls{mby},\x_i\right)\prob(\x_i)\label{eq:aap}
\end{align}
where $\x_i$ is a codeword from the set of all codewords, $\gls{mby}$ is the received word, $Y$ and $X$ are random variables representing received word and transmitted codeword respectively. For equiprobable codewords $\x_i$, a \gls{map} decoder is equivalent to a \gls{ml} decoder.

Due to the high complexity of \gls{map} decoders, \gls{ldpc} codes are usually decoded using iterative \acrlong{mpa}.
\subsection{Belief Propagation}
The \gls{mpa} uses the \gls{bp} technique~\cite{Hagenauer1996} to compute the \emph{a-posteriori} probability of the bits in the transmitted codeword given the received word in an iterative fashion. The idea behind belief propagation is exchanging uncertainties between the bits which are connected as defined by the encoder or \gls{pcm}. Refer to Section~\ref{sec:enc_design} to see how different bits in the codeword are dependent on each other. The algorithm uses \glspl{llr} instead of \glspl{app} as in (\ref{eq:aap}) for numerical stability. The \gls{llr} values given by the channel for the received bits are
\begin{align}\mathcal{L}(y_i)=\log\frac{\prob\left(X_i=1\mid\gls{mby}\right)}{\prob\left(X_i=0\mid\gls{mby}\right)}\end{align}
where $i=0,\dots,n-1$ is the index of bits in the codeword.

\begin{figure}[htbp]
  \centering
  \tikzsetnextfilename{app_llr}
  \includegraphics[width=0.9\linewidth]{plots/app_llr}
  \caption{Relation between \gls{app} and \gls{llr}.}
  \label{fig:app_llr}
\end{figure}

Figure~\ref{fig:app_llr} shows the relation between $\prob\left(X_i=0\mid\gls{mby}\right)$ and $\mathcal{L}(y_i)$. The \gls{app} values range from 0 to 1 while the \gls{llr} takes values ranging from $-\infty$ to $+\infty$ which makes calculation of messages easier.

In a single iteration of the algorithm, the \glspl{llr} of each bits in the codeword are updated through two intermediate message computations: \gls{v2c} message and \gls{c2v} message.
\begin{itemize}
  \item \gls{v2c} message: Each \gls{vn} passes its \glspl{llr} on to its neighboring nodes (neighboring nodes are the \glspl{cn} to which the \gls{vn} is connected in the Tanner graph). These \glspl{llr} contain only extrinsic information from all other \glspl{cn} in the previous iteration. The expression for the \gls{v2c} message is given by~\cite{Hagenauer1996}
  \begin{align}
    \mathcal{L}^{\mathrm{vc}}_{ij}=\mathcal{L}(y_i)+\sum_{j^\prime\in\mathcal{E}_v(i)\backslash j} \mathcal{L}^{\mathrm{cv}}_{j^\prime i}
  \end{align}
  \begin{figure}[htbp]
    \centering
     \tikzsetnextfilename{v2c}
    \includegraphics[width=0.5\textwidth, height=0.2\textwidth]{graphics/v2c}
    \caption{Example showing that messages from all other \glspl{cn} sum-up with the channel's \gls{llr} to form the V2C message to the first \gls{cn}.}
    \label{fig:v2c}
  \end{figure}
  where $\mathcal{L}^{\mathrm{vc}}_{ij}$ is the \gls{v2c} message from the $i$-th \gls{vn} to the $j$-th \gls{cn}, $\mathcal{L}^{\mathrm{cv}}_{ji}$ is the C2V message from the $j$-th \gls{cn} to the $i$-th \gls{vn} in the previous iteration and $\mathcal{E}_v(i)$ is the set containing  all $n_c$ \glspl{cn} connected to the $i$-th \gls{vn}.
  \item \gls{c2v} message: Each \gls{cn} processes the received \gls{v2c} messages and computes extrinsic information for its neighboring \glspl{vn}. These extrinsic informations contain \gls{v2c} messages from \glspl{vn} other than the destination \gls{vn}. The expression for C2V messages is given by~\cite{Hagenauer1996}
  \begin{align}\label{eq:c2v}
  \mathcal{L}^{\mathrm{cv}}_{ji}=2\arctanh\left(\prod_{i^\prime\in\mathcal{E}_c(j)\backslash i}\tanh\left(\frac{\mathcal{L}^{\mathrm{vc}}_{i^\prime j}}{2}\right)\right)
  \end{align}
  \begin{figure}[htbp]
    \centering
     \tikzsetnextfilename{c2v}
    \includegraphics[width=0.5\textwidth, height=0.2\textwidth]{graphics/c2v}
    \caption{Example showing that messages from all other \glspl{vn} combine using equation~(\ref{eq:c2v}) to form the C2V message to the second \gls{vn}.}
    \label{fig:c2v}
  \end{figure}
  where $\mathcal{E}_c(j)$ is the set containing all $n_v$ \glspl{vn} connected to the $j$-th \gls{cn}.
\end{itemize}
The process of sending a \gls{v2c} message, receiving a \gls{c2v} message from the same edge, and summing it up with the current \gls{llr} is termed an \emph{edge update}. The above steps indicate the \gls{bp} technique. It is also called \gls{spa}.

The high complexity C2V message computation can be approximated by a low-complexity computation called the \emph{\acrlong{msa}}. The \gls{msa} version of the expression in (\ref{eq:c2v}) is given by
\begin{align}\label{eq:msa}
\mathcal{L}^{\mathrm{cv}}_{ji}\approx\left(\prod_{i^\prime\in\mathcal{E}_c(j)\backslash i}\sign\left(\mathcal{L}^{\mathrm{vc}}_{i^\prime j}\right)\right)\cdot\min_{i^\prime}\lvert\mathcal{L}^{\mathrm{vc}}_{i^\prime j}\rvert.
\end{align}

\begin{figure}[htbp]
  \centering
   \tikzsetnextfilename{flooding}
  \includegraphics[width=0.5\textwidth, height=0.2\textwidth]{graphics/flooding}
  \caption{Illustration of parallel scheduling with an example Tanner graph.}
  \label{fig:flooding}
\end{figure}

One can perform the edge updates in different sequences. Two such sequencing methods are \emph{parallel scheduling} and \emph{serial scheduling}. In parallel scheduling which is also referred to as \emph{flooding}, the \glspl{vn} send V2C messages to all \glspl{cn} at once and then C2V messages are computed and passed to all \glspl{vn}. In Figure~\ref{fig:flooding}, the left graph indicates V2C message passing and the right graph indicated C2V message passing. In serial scheduling~\cite{Zhang2002}, the \glspl{vn} are updated in a \gls{cn}-by-\gls{cn} or row-by-row (in the \gls{pcm}) manner. Each \gls{cn} processes its incoming V2C messages and sends the corresponding C2V messages to its neighboring \glspl{vn}. This is called a \emph{row update} or a \emph{layer update}. Figure~\ref{fig:serial_sch} shows how message passing is performed in serial scheduling. Parallel scheduling is much faster on a parallel computing platform because the edge updates are independent to each other. But compared to its serial counterpart, parallel scheduling lacks performance because the edges are updated in parallel and only the original messages from the \glspl{vn} are used. Serial scheduling yields better performance than parallel scheduling as the \glspl{cn} that are being processed later will have updated \glspl{llr} from the \glspl{vn}. In a conventional \gls{bp} decoder with any type of scheduling, the entire message passing is repeated for a maximum of $I$ iterations.

With serial scheduling, different orders of \gls{cn} processing results in decoding performance changes. Figure~\ref{fig:serial_sch} illustrates serial scheduling with \gls{cn} processing from left to right i.e., top to bottom in the \gls{pcm}. For irregular codes, a good choice is to start processing the \gls{cn} that has the lowest \gls{vn} degree and move to higher \gls{vn} degree \glspl{cn}~\cite{Fren1902:Static}.

\begin{figure}[htbp]
  \centering
  \tikzsetnextfilename{serial_sch}
  \includegraphics[width=0.5\textwidth, height=0.2\textwidth]{graphics/serial_sch}
  \caption{Illustration of serial scheduling with the same Tanner graph as in Figure~\ref{fig:flooding}. The processing starts from the left most \gls{cn} and moves right to the last \gls{cn}.}
  \label{fig:serial_sch}
\end{figure}

In our implementation, we use an improved \gls{msa} as proposed in~\cite{Jones2003} which is a combination of (\ref{eq:c2v}) and (\ref{eq:msa}). We do serial scheduling with layer updates starting from top to bottom. We also do a bottom to top layer update which is discussed in Chapter~\ref{ch:dec_improve}.

\subsection{LDPC-CC-specific Decoding Techniques}
The conventional \gls{bp}-based block decoder can be used to decode any \gls{ldpcbc} or terminated \gls{ldpccc} in which the \gls{bp} is performed throughout the whole Tanner graph at once. For \glspl{ldpccc}, the convolutional structure imposes a constraint on the \glspl{vn}: two \glspl{vn} of the \gls{pcm} that are at least $(m_s+1)n$ columns apart cannot be involved in the same parity-check equation. This characteristic can be exploited to perform iterative \gls{bp} decoding only to a part of the codeword at a time. Two of such decoding techniques are \emph{pipeline} and \emph{window} decoding.
\subsubsection{Pipeline Decoder}
A decoding technique that was proved to be efficient for \glspl{ldpccc} is the \emph{pipeline decoding} introduced in~\cite{Felstrom1999}. The pipeline decoder employs $I$ parallel processing units. Each processing unit covers $\gls{mlc}=(m_s+1)n$ \glspl{vn}, so that during a single decoding iteration the messages are only passed to other \glspl{vn} within the same processing unit. Hence, all the processing units cover a total of $Il_c$ \glspl{vn}. In each time instance, $n$ \glspl{vn} and $n-k$ \glspl{vn} enter the rightmost processor and $n$ estimated \glspl{vn} leave from the leftmost processor. Hence, at the end of $I$ time instances, all \glspl{cn} are processed $I$ times. Figure~\ref{fig:pipeline} illustrates pipeline decoding of a code with $R_\infty=1/2$, $\gls{mms}=3$ and $I=3$. The codeword is indicated by the rectangular bar above the \gls{pcm}. The red windows are the three processing units. The green (backhatched) part of the codeword indicates the estimated \glspl{vn} and the blue (hatched) part indicates the next available estimated \glspl{vn}. The brown (vertically hatched) and red (horizontally hatched) regions indicate the \glspl{vn} that are currently being processed and \glspl{vn} that are yet to be processed, respectively. The arrows indicate the \glspl{cn} that are processed at the current time instance.

The decoding speed (the output bit rate of the decoder) of a pipeline decoder is given by
\begin{align}
\gls{mkapa}_{\text{p}}=\frac{n}{\gls{mpsi}}
\end{align}
where $\psi$ is the time taken to perform one \gls{cn} processing.

\begin{figure}[htbp]
  \centering
  \tikzsetnextfilename{pipeline}
  \includegraphics[width=0.5\textwidth, height=0.2\textwidth]{graphics/pipeline}
  \caption{\gls{pcm} illustrating the pipeline decoding technique for code with $R_\infty=1/2$, $m_s=2$ and $I=3$. The arrows indicate the CNs that are processed at current time instance.}
  \label{fig:pipeline}
\end{figure}

\subsubsection{Window Decoder}\label{sec:back_wd}
Figure~\ref{fig:wd} illustrates a window decoder with window size of $W=4$ and window position $\rho=5$. The solid red box is the current window instance. The blue hatched region of the codeword is the \emph{target \glspl{vn}}, the red vertically hatched region is rest of the \glspl{vn} that are updated inside the current window. The blue colored edges in the \gls{pcm} are the edges that are updated during the current window instance. The red colored edges are outside the window but they are still updated since their corresponding \glspl{cn} are inside the window. The green backhatched region of the codeword are the \glspl{vn} that receive updates from the red colored edges.
\begin{figure}[htbp]
  \centering
  \tikzsetnextfilename{pcm_wd}
  \includegraphics[width=0.5\textwidth, height=0.2\textwidth]{graphics/pcm_wd}
  \caption{\gls{pcm} illustrating the \acrlong{wd} technique for $R_\infty=1/2$ codes and $m_s=2$. The current window position $\rho=5$ is indicated by the solid red box and the next window position is indicated by the dashed red box.}
  \label{fig:wd}
\end{figure}

\begin{figure}[htbp]
  \centering
  \tikzsetnextfilename{pcm_wd_last}
  \includegraphics[width=0.5\textwidth, height=0.2\textwidth]{graphics/pcm_wd_last}
  \caption{Target \glspl{vn} for first, middle and the last window positions.}
  \label{fig:wd_last}
\end{figure} 

One may choose to not update the edges that are outside the window. This is to prevent the correctly estimated \gls{llr} values of the \glspl{vn} from updating to incorrect values that might propagate from inside the window. Another option might be to only read the \glspl{llr} values from \glspl{vn} connected to the red edges. This prevents the estimated \gls{llr} of the \glspl{vn} from updating to incorrect values while having the advantage of using their \glspl{llr} to update the \glspl{vn} inside the window. In this thesis, we choose to update the \glspl{vn} in the green backhatched region. This has an advantage of increasing the magnitude of their \glspl{llr} which in turn has a positive influence on the \glspl{vn} in the next window instances. Also, the \glspl{vn} in the left of the window are more reliable than in the right of the window. Hence, the probability of \gls{llr} of  the \glspl{vn} in the green backhatched region flipping their sign is low. This motivates us to choose this window configuration. The brown dotted region of the codeword is the \glspl{vn} that are already estimated and no longer receive updates from further decoding. The red dashed box indicates the next window instance. 

A window must include $Wn$ \glspl{vn} and $W(n-k)$ \glspl{cn}. Hence, the size of the window should be $\gls{mW}\geq (m_s+1)$ because smaller window sizes will not include at least one full \gls{cn} with all its edges. At each window instance, the first $n$ \glspl{vn} are considered to be the target nodes. However, all \glspl{vn} connected to the $W(n-k)$ \glspl{cn} are updated, only the correctness of target nodes are considered as the criteria for moving on to the next window. The \gls{bp} decoding is performed within the current window for a maximum number of $I$ iterations or until the parity-check equations involving the target nodes are fulfilled (whichever occurs earlier). Then the window shifts forward such that the next $n$ \glspl{vn} become the target nodes. The process continues until all \glspl{vn} in the received word are decoded.

With the window decoder, there are several ways to perform the decoding when the window reaches the rightmost end of the \gls{pcm}. The conventional way is to keep moving the window until the last $n$ \glspl{vn} are the target \glspl{vn}. Another method is when the window touches the rightmost column of the \gls{pcm}, the window extends its height to include the remaining $m_s$ \glspl{cn} of the \gls{pcm}. Also, all the \glspl{vn} inside the last window are considered as target \glspl{vn}. The last window scenario is illustrated in Figure~\ref{fig:wd_last}. We choose the second method as it reduces the number of iterations compared to the conventional method. The lower \gls{cn} degree in the end of the \gls{pcm} compensates to the reduced decoding performance for having fewer iterations.

The decoding speed in bits of a window decoder is given by
\begin{align}
\kappa_{\text{w}}=\frac{n}{\psi\cdot I}.
\end{align}

The main benefit of window decoding over full-block decoding is that the memory requirements are reduced because at any instance the \gls{bp} is performed on a smaller number of \glspl{vn} rather than the whole graph. For non-packet-wise transmissions, the decoded bits are sent to higher layer for processing rather than waiting for the whole codeword to be decoded as in a block decoder. However, these benefits come with a cost of reduced performance since the \gls{bp} is limited to fewer \glspl{vn} and \glspl{cn}.

In this thesis, we choose window decoding over pipeline decoding because implementing parallel processing units in software is infeasible.

\section{System Model}\label{sec:sys_mod}
In this thesis, we consider a digital baseband system model for simulations as shown in Figure \ref{fig:system}. The information bits are generated using a random-number generator with uniform distribution. The sequence of bits $\m$ of the generated random bit is encoded using the \emph{BPL Codes Encoder} block. The details of this block are discussed in Chapter~\ref{ch:encode}. The \emph{\gls{qam} Modulator} block receives the codewords $\x\in\mathbb{F}_2^{Ln\times 1}$ from the encoder and maps them to complex-valued symbols depending on the chosen modulation scheme. The output of the modulator is a vector of symbols given by $\gls{mbu}\in\mathbb{C}^{\frac{Ln}{o}\times 1}$ where \gls{mo} is the order of modulation and lets assume that $Ln=go, g\in\mathbb{Z}_+$. Figure~\ref{fig:qpsk} shows a \gls{qpsk} symbol constellation. It is seen that the adjacent symbols differ by only one bit. This is called \emph{Gray Mapping}~\cite{proak}. In all our simulations we use \gls{qpsk} with gray mapping and no interleaving. Hence, $\gls{mbu}_i,i=0,\dots,\frac{Ln}{o}-1\in\{e^{j\frac{\pi}{4}},e^{j\frac{3\pi}{4}},e^{-j\frac{3\pi}{4}},e^{-j\frac{\pi}{4}}\}$.
\begin{figure}[htbp]
  \centering
  \tikzsetnextfilename{system_model}
  \includegraphics[width=\textwidth, height=\textwidth]{graphics/system_model}
  \caption{System model for simulations.}
  \label{fig:system}
\end{figure}

\begin{figure}[htbp]
  \centering
  \tikzsetnextfilename{qpsk}
  \includegraphics[width=\textwidth, height=\textwidth]{graphics/qpsk}
  \caption{Symbol constellation of \gls{qpsk} modulation with gray mapping.}
  \label{fig:qpsk}
\end{figure}

The channel model we consider is a simple \gls{awgn} channel with no fading or multi-path components. So the received symbols are given by \begin{align}\gls{mbv}=\gls{mbu}+\gls{mbz}\end{align}
\begin{figure}[htbp]
  \centering
  \tikzsetnextfilename{bpsk_awgn}
  \includegraphics[width=0.9\textwidth,height=0.45\textwidth]{graphics/bpsk_awgn}
  \caption{Conditional probability density functions of a \gls{bpsk} symbol over \gls{awgn} channel.}
  \label{fig:bpsk_awgn}
\end{figure}
where $\gls{mbz}=\mathcal{CN}(0,2\sigma^2)\in\mathbb{C}^{\frac{Ln}{o}\times 1}$ is a complex random variable of Gaussian distribution with zero mean and variance $2\sigma^2$. Note that only one symbol in $\gls{mbu}$ is transmitted per channel use. The received symbol vector is $\gls{mbv}\in\mathbb{C}^{\frac{Ln}{o}\times 1}$. The output of the \emph{QAM Demodulator} is a vector of \glspl{llr} $\gls{mby}\in\mathbb{R}^{Ln\times 1}$ corresponding to the bits in $\x$. The \glspl{llr} are computed using the symbols in $\gls{mbv}$ and the modulation order. Each \gls{qpsk} symbol is a combination of two \gls{bpsk} symbols that are orthogonal to each other. Hence, the \glspl{llr} of both the bits can be computed separately. The \gls{llr} of the first bit of each received \gls{qpsk} symbol $y$ is computed as
\begin{align}
\mathcal{L}_{\text{first}}&=\log\frac{\gls{mprob}\left(X_1=0\mid\re(y)\right)}{\prob\left(X_1=1\mid\re(y)\right)}
\end{align}
and of the second bit as
\begin{align}
\mathcal{L}_{\text{second}}&=\log\frac{\prob\left(X_2=0\mid\im(y)\right)}{\prob\left(X_2=1\mid\im(y)\right)}.
\end{align}
Using Bayes' Theorem and assuming that the bits in $\x$ are equiprobable, we have
\begin{align}
\mathcal{L}_{\text{first}}=\log\frac{\pdf_{\re(Y)\mid X}\left(\re(y), X_1=0\right)}{\pdf_{\re(Y)\mid X}\left(\re(y), X_1=1\right)},\\
\mathcal{L}_{\text{second}}=\log\frac{\pdf_{\im(Y)\mid X}\left(\im(y), X_2=0\right)}{\pdf_{\im(Y)\mid X}\left(\im(y), X_2=1\right)}.
\end{align}
On substituting the conditional \glspl{pdf} of \gls{bpsk} symbols as shown in Figure~\ref{fig:bpsk_awgn}, we get
\begin{align}
\mathcal{L}_{\text{first}}&=\log\frac{e^{-\frac{(\re(y)-1)^2}{2\sigma^2}}}{e^{-\frac{(\re(y)+1)^2}{2\sigma^2}}}\\
&=\frac{2}{\sigma^2}\re(y)\label{eq:final_llr1}.
\end{align}
Similarly,
\begin{align}
\mathcal{L}_{\text{second}}&=\frac{2}{\sigma^2}\im(y)\label{eq:final_llr2}.
\end{align}
Note that the symbol values in Figure~\ref{fig:bpsk_awgn} are $X_i\in\{-1,+1\},i=1,2$ which corresponds to information bits 1 and 0 respectively. The noise variances of the real and imaginary parts of the complex random variable $Z$ are each $\sigma^2$ from
\begin{align}
\text{E}\{||Z||^2\}&=\text{E}\{||\re(Z)||^2\}+\text{E}\{||\im(Z)||^2\}.
\end{align}
The \gls{snr} \gls{msnr} of the \gls{qpsk} signal is
\begin{align}
\frac{E_s}{N_0}&=2\frac{E_b}{N_0}
\end{align}
where $E_s$ is the energy per \gls{qpsk} symbol, $E_b$ is the energy per bit and $N_0$ is the \gls{awgn} power spectral density. The \gls{snr} of each bit is equivalent to the \gls{snr} of \gls{bpsk} signal and is given by
\begin{align}
\frac{E_b}{N_0}&=\frac{\text{E}\{||X_1||^2\}}{\text{E}\{||\re(Z)||^2\}}\\
&=\frac{\text{E}\{||X_2||^2\}}{\text{E}\{||\im(Z)||^2\}}\\
&=\frac{1}{\sigma^2}.
\end{align}
Hence, the \glspl{llr} can be written as
\begin{align}
\mathcal{L}_{\text{first}}&=2\frac{E_b}{N_0}\re(y),\\
\mathcal{L}_{\text{second}}&=2\frac{E_b}{N_0}\im(y).
\end{align}

After the \glspl{llr} are computed, they are decoded using the \gls{bp} and \gls{wd} techniques used inside the \emph{Window Decoder with BP} block. The decoding-improvement techniques that are used in this block are discussed in Chapter~\ref{ch:dec_improve}. The estimated bits $\widehat{\m}$ are then compared with the output of the random bit generator $\m$ to calculate the \gls{ber} and \gls{bler}. \gls{ber} \gls{mber} is the ratio between the number of error-bits and the total number of bits transmitted. The probability of error for each bit in the codeword is given by
\begin{align}
P_b(i)&=\prob\{\widehat{X}_i\neq X_i\}
\end{align}
and the overall probability of bit error is give by
\begin{align}
P_b&=\frac{1}{Ln}\sum_{i=0}^{Ln-1}P_b(i)\label{eq:indiv_prob}
\end{align}
\gls{bler} \gls{mbler} is the ratio between the number of error-blocks and the total number of blocks transmitted. A block is considered to be an error-block if at least one information bit is incorrect. The \gls{bler} is given by
\begin{align}
P_L&=\prob\{\widehat{\m}\neq \m\}.
\end{align}
With \gls{ber} and \gls{bler} being the metrics for measuring decoding performance, \gls{aneu} is the metric for measuring decoding complexity. \gls{aneu} \gls{meta} is given by
\begin{align}
\eta&=\sum_{\rho=0}^{\xi-1}\eta_\rho I_\rho
\end{align}
where $\xi$ is the total number of window positions, $\eta_\rho$ is the number of edges inside and $\gls{mI}_\rho$ is the number of iterations performed in \gls{mrho}-th window position.
  \section{Encoding of the \acl{bpl} Codes}\label{sec:encode}
In this section, we discuss how the encoder for \ac{bpl} codes is designed and how the termination is handled. In some places of this section, we use examples of \acp{ldpccc} with small $L$ and $m_s$ as the \ac{bpl} codes are too large to be represented on paper.

\subsection{Encoder Design}\label{sec:enc_design}
Encoding algorithms are simpler than decoding algorithms because there exists no uncertainty in the output bits of an encoder. The encoding of \ac{bpl} codes is done in the following steps:
\begin{enumerate}
  \item The information bits are divided into several \acp{cb} and copied into the output buffer of the encoder. An example of this step for a code with $R_\infty=2/3$ is shown in Figure~\ref{fig:encode_copy}.
  \begin{figure}[htbp]
    \centering
    \includegraphics[width=\textwidth, height=\textwidth]{graphics/encoder_copy}
    \caption{Copy input bits to output buffer. The shaded box represents the position of the parity bits.}
    \label{fig:encode_copy}
  \end{figure}
  \item The parity bits are generated by performing addition modulo 2 with all bits (both information and parity bits) in the output buffer according to the parity-check polynomials of the code. An example of the parity-bit generation of a code with $R_\infty=2/3$ is shown in Figure~\ref{fig:encoder_paritygen}.
   \begin{figure}[htbp]
    \centering
    \includegraphics[width=\textwidth, height=\textwidth]{graphics/encoder_parity}
    \caption{Generation of the last parity bit. Note: Only for representative purpose.}
    \label{fig:encoder_paritygen}
  \end{figure}
  \item The termination sequence is appended to the output buffer of the encoder and is discussed in Section~\ref{sec:bpl_termi} in detail.
\end{enumerate}

\subsection{Termination Sequence}\label{sec:bpl_termi}
The objective of terminating any \ac{ldpccc} is to have a finite codeword length. Termination essentially means truncating the \ac{pcm}. To satisfy the last $m_s$ parity checks, an appropriate termination sequence must be appended to the codeword. To find the appropriate termination sequence, one must solve a system of linear equations because the usual encoding procedure satisfies only the first parity check and not the remaining $m_s-1$ parity checks.
\begin{figure}[htbp]
  \centering
  \includegraphics[width=\textwidth, height=\textwidth]{graphics/term_solve}
  \caption{The last part of a \ac{pcm} of an example \ac{ldpccc} code with $R_\infty=2/3$ and $m_s=4$. Depicted above the \ac{pcm} is the last part of the codeword vector.}
  \label{fig:bpl_term}
\end{figure}

The \ac{pcm} in (\ref{eq:bpl_pcm}) of a \ac{bpl} code has a structure as shown in Figure~\ref{fig:bpl_term} where $\mathbf{e}\in\mathbb{F}_2^{m_sn\times 1}$ is the last part of the codeword vector. Vector $\mathbf{a}\in\mathbb{F}_2^{l_t\times 1}$ is the termination sequence that should be determined and appended to the end of $\mathbf{e}$. All the rows of the matrix $[\mathbf{P},\mathbf{D}]$ are the parity checks that are targeted to be satisfied by the termination sequence. Hence, provided a correct termination sequence $\mathbf{a}$, the following equations from~\cite{Chen2006} hold true.
\begin{align}
\begin{bmatrix}\mathbf{P} &\mathbf{D}\end{bmatrix}\odot
\begin{bmatrix}
\mathbf{e}\\
\mathbf{a}
\end{bmatrix}&=\mathbf{0}\\
\mathbf{P}\odot\mathbf{e}&=\mathbf{D}\odot\mathbf{a}
\end{align}
The termination sequence is given by
\begin{align}\label{eq:term_sol}
\mathbf{a}=\mathbf{D}^{-1}\odot\mathbf{P}\odot\mathbf{e}.
\end{align}
The maximum length $l_t$ of the termination sequence $\mathbf{a}$ should not exceed the one given in the \ac{bpl} standard. However, a shorter termination sequence with $l_t\geq n$ can be chosen such that $\mathbf{D}$ is a square matrix. Once the termination sequence $\mathbf{a}$ is determined from equation (\ref{eq:term_sol}), it is appended to the codeword and the remaining part of the termination sequence (if any) is filled with zeros. These zeros do not contribute to any parity checks. For an encoder that is implemented on hardware, these zeros are required to bring the encoder to an all-zero state. But we omit these zeros since we do software implementation. Hence, the resulting \ac{pcm} and codeword looks like the one shown in Figure~\ref{fig:bpl_term} with no zeros at the end of the codeword and without their corresponding edges in the \ac{pcm}.
%\begin{figure}[htbp]
%  \centering
%  \includegraphics[width=\textwidth, height=\textwidth]{graphics/term_result}
%  \caption{The last part of the \ac{pcm} and codeword after adding the actual termination sequence.}
%  \label{fig:term_res}
%\end{figure}

Unfortunately, we found that a proper termination sequence cannot be computed for \ac{bpl} codes. It is found through numerical evaluations that $\mathbf{D}$ is not full-rank and hence its inverse do not exist. A full rank $\mathbf{D}$ matrix is possible if $\mathbf{D}\in\mathbb{F}_2^{r\times c}$ and $r>c$. But with the help of numerical solvers, we found that the solution for such an overdetermined system do not exist either. Hence, a termination sequence to satisfy the last $m_s$ parity checks cannot be found for \ac{bpl} codes. So we choose to omit the last $m_s$ parity checks from the \ac{pcm} and do the termination by \emph{zero-tailing} as mentioned in the standard.

The steps for performing zero-tailing termination are as follows.
\begin{enumerate}
  \item The $n_z$ zero-tail bits are appended to the input buffer of the encoder after the $n_m$ information bits.
  \item The encoding as mentioned in Section~\ref{sec:enc_design} is performed to all the information bits and zero-tail bits.
\end{enumerate}
With the zero-tail-termination, the zero-tail bits are always known at the receiver. Hence, the zero-tail bits are not transmitted in an actual transmission system. Only the parity bits generated from the zero-tail bits are transmitted.

The zero-tail-termination does not satisfy the last $m_s$ parity checks of the \ac{pcm}. Hence, the \ac{pcm} of a zero-tail-terminated \ac{bpl} codes looks like the one shown in Figure~\ref{fig:pcm_zero}.
\begin{figure}[htbp]
  \centering
  \includegraphics[width=\textwidth, height=\textwidth]{graphics/term_zero}
  \caption{The last part of the \ac{pcm} and codeword after adding the zero-tailing termination sequence.}
  \label{fig:pcm_zero}
\end{figure}

The \ac{pcm} of the zero-tail-terminated codes have the same \ac{cn} degrees in the middle and in the end. But in contrast, properly terminated codes have lower \ac{cn} degrees at the end. This reduces the performance of the codes as lower \ac{cn} degree means better reliability of the variable nodes connected to them. Since the zero-tail bits are known at the receiver, they improve the reliability of the connected \acp{vn} during the decoding process.
  \section{Decoding Improvements}\label{ch:dec_improve}
In this section, we discuss the techniques that are investigated in this thesis to improve the decoding performance. We start with the discussion of existing techniques that improve the decoding performance. We then discuss the techniques investigated in this thesis and possible areas of further research and improvements. We finish this section by discussing the results from our simulations.

\subsection{Existing Techniques and Motivation}
Improvement techniques for window decoding of \ac{ldpccc} are being widely discussed in literatures. An attractive technique that we found in the literature is the \emph{Zigzag decoder} proposed in~\cite{Shadi2015}. In a normal window decoder, the window moves from left to right of the \ac{pcm}. So, the messages pass from left to right of the codeword. The \acp{vn} in the left and right ends of the codeword are more reliable than that are in the middle. The Zigzag decoder moves the window from right to left of the \ac{pcm} within a small part where the \acp{vn} are not decoded. This allows the messages to pass from right to left of the codeword and hence influencing the high reliability of the \acp{vn} in the right on the \acp{vn} in the left.

The Zigzag decoder performs better than the conventional left to right decoder but the decoding complexity in terms of number of iterations are higher. The implementation complexity is also high due to the nature of the decoder. Hence, this motivates us to develop a technique to utilize the high reliability of \acp{vn} in right of the codeword while have similar decoding and implementation complexity as the conventional window decoder.

Another improvement technique that we found to be interesting is the \emph{early-success} technique in which the decoding stops before reaching the maximum number of iterations $I$. Although its very natural to stop decoding when the target \ac{vn} are decoded, one should carefully choose an optimum criterion for deciding whether the target \ac{vn} are decoded. A heuristic choice is to check whether the \emph{target-\acp{cn}} i.e., the \acp{cn} connected to the target \acp{vn} are fulfilled. A new early-success criterion was proposed in~\cite{Kang2018} based on reliable \acp{vn} and is called \ac{psc}. They distinguish the \acp{vn} inside the window as \emph{complete-\acp{vn}} and \emph{incomplete-\acp{vn}}. Complete-\acp{vn} are \acp{vn} whose connected \acp{cn} are completely inside the window. The remaining \acp{vn} in the window are called the incomplete-\acp{vn}. This is illustrated in Figure~\ref{fig:comp_vn}. The authors say that the complete-\acp{vn} are more reliable than the incomplete ones as they get updates from all their connected \acp{cn}. Hence, correctness on only the complete-\acp{cn} i.e., the \acp{cn} connected to only the complete-\acp{vn} are considered as early-success criterion.

Although the \ac{psc} rule proposed in~\cite{Kang2018} reduces the number of iterations with same performance, they are suitable only for small windows where $W<2(m_s+1)$. For larger window sizes, the number complete-\acp{vn} are greater than the number of incomplete-\acp{vn}. This also makes the number of complete-\acp{cn} to be grater than the number of target-\acp{cn}. So for larger windows, the \ac{psc} rule actually increases the number of iterations. This motivates us develop a better early-success criterion that depends on the window size.

\subsection{Base Decoder Configuration}
Before discussing the details of the techniques used in the improved decoder, it is essential to define a decoder configuration to which the improvements are made. This allows better understanding of the techniques and evaluation of the simulation results. Let us call the decoder without out improvements as \ac{bd} and the one with our improvements as \ac{id}. The \ac{bd} uses the \ac{wd} technique to decoder the \ac{bpl} codes. The window slides from the left end of the \ac{pcm} to the right end. Within each window, serial scheduling is performed by updating the \acp{cn} from top to bottom for a maximum of $I_{\text{max}}$ iterations.

In the last window instance i.e., when the window touches the right most column of the \ac{pcm}, all the \acp{vn} inside the window are considered as target \acp{vn}. Hence, the early success criterion is to check if all \acp{cn} inside the window are fulfilled. This idea was proposed in~\cite{Ali2018} with flooding schedule. We adapt this technique in our \ac{bd} to minimize the total number to iterations.

An early success technique is used in our \ac{bd}. This technique prevents the decoder from exhausting all iteration in a window if the target \acp{vn} are decoder before reaching the maximum number of iteration. Thus saving computational effort. The criterion for finding the correctness of the target \acp{vn} is to check if all the \acp{cn} connected to the target \acp{vn} fulfill their parity-checks.

We now know that the \ac{bpl} codes are terminated using the zero-tail termination technique. The zero-tail bits and their positions are always known at the receiver. Hence, the \acp{llr} of these bits are made to $+\infty$ which indicates that these bits most certainly have the value 0.

The configurations of the \ac{bd} is summarized below.
\begin{enumerate}
  \item \ac{bp} based window decoder and the window moves from left to right.
  \item Maximum number of iterations for each window is $I_{\text{BD}}$.
  \item Serial scheduling with top to bottom \ac{cn} update order.
  \item In the last window, all \acp{vn} are considered as target \acp{vn}.
  \item Target \acp{cn} are considered as early-success criterion.
\end{enumerate}

\subsection{\acl{lrl} \acl{wd}}
In this thesis, we propose the \emph{\ac{lrl} decoder}. The \ac{lrl} decoder moves the window from left to right and from right to left of the \ac{pcm} unlike the \ac{bd} that moves the window only from left to right. As already mentioned earlier, the \acp{vn} in the left and right of the codeword are more reliable than the \acp{vn} in the middle due to the lower \ac{cn} degrees in both the ends. This can be seen in Figure~\ref{fig:indiv_ber} which plots the individual probability of error for all the bits in the codeword. We can see that the bits in the left and right of the codeword have lower probability of error than the bits in the middle. During the first half decoding phase of the \ac{lrl} decoder, the window moves from the first window position to the last window position of the \ac{pcm}. During the second phase of the decoding i.e., after decoding the last window position, the window moves to the left till it reaches the first window position. This is illustrated in Figure~\ref{fig:pcm:lrl}.
\begin{figure}[htbp]
  \centering
  \includegraphics[width=0.5\textwidth, height=0.2\textwidth]{graphics/pcm_lrl}
  \caption{\ac{pcm} illustrating \ac{lrl} decoder.}
  \label{fig:pcm_lrl}
\end{figure}
So during the first decoding phase, the \acp{vn} in the left increases the reliability of the \acp{vn} to the right as the window moves forward. Then during the second decoding phase, the better \acp{vn} in the right of the codeword increase the reliability of the \acp{vn} to the left as the window moves backwards. The maximum number of iterations within each window is half of the maximum number of iterations in the \ac{bd} i.e., $I_{\text{LRL}}=I_{\text{BD}}/2$. This is done to maintain the same decoding complexity as the \ac{bd} because each window in the \ac{lrl} decoder is decoded twice.

With the \ac{lrl} decoder, we propose two different configurations of windows. They are illustrated in Figure~\ref{fig:win_config_lrl}. The red box indicates the window. The figure to the left shows a window configuration where the left most \acp{vn} (blue hatched) are the target \acp{vn}. The order of \ac{cn} update is from top to bottom as one indicated by the brown arrow. Let us call this \ac{lrl} configuration-\rom{1}. The illustration to the right of the Figure~\ref{fig:win_config_lrl} shows that the right most \acp{vn} in the window are the target \acp{vn}. The order of \ac{cn} update is from bottom to top as indicated by the brown arrow. Let us call this \ac{lrl} configuration-\rom{2}.

\begin{figure}[htbp]
  \centering
  \includegraphics[width=0.5\textwidth, height=0.2\textwidth]{graphics/win_config_lrl_new}
  \caption{Different window configurations used in \ac{lrl} decoder. Left image illustrates \ac{lrl} configuration-\rom{1} and right image illustrates \ac{lrl} configuration-\rom{2}.}
  \label{fig:win_config_lrl}
\end{figure}

We discuss the performance of the \ac{lrl} decoder with both configurations. First we use \ac{lrl} configuration-\rom{1} in both phases of the \ac{lrl} decoder. When \ac{lrl} configuration-\rom{1} is used in the second phase, the window iterates for more number iterations than when \ac{lrl} configuration-\rom{2} is used. This is because the target \ac{vn} are at the end of the window corresponding to the window direction.

Then we use the \ac{lrl} configuration-\rom{1} during the first phase and \ac{lrl} configuration-\rom{2} during the second phase. During the second phase the \ac{cn} updates are done from bottom to top to increase the effect of high reliability of \acp{vn} in the right on to the \acp{vn} in the middle. The simulation results of both the configurations are evaluated in Chapter~\ref{ch:simulation}.

\subsection{\acl{ipsc}}
The next technique we propose is the \ac{ipsc} rule. As mentioned earlier, a \ac{psc} was proposed in~\cite{Kang2018} that uses only complete-\ac{cn} as early-success criterion. In the \ac{ipsc} rule, we introduce an additional early-success rule for $W>2(m_s+1)$. That is, when the window size $W>2(m_s+1)$ the target \ac{cn} are considered as early-success criterion. This could reduce the number of iterations or edge updates because for $W>2(m_s+1)$ the number of complete-\ac{cn} are greater than the number of target \ac{vn} and so the window could converge sooner. A window is said to have converged when the target \ac{vn} are decoded. Table~\ref{tab:ipsc} summarizes the \ac{ipsc} technique. The simulation results of the \ac{ipsc} are evaluated in Chapter~\ref{ch:simulation}.

\begin{table}[htbp]
\centering
\begin{tabular}{|l|l|l|}
  \hline
  \textbf{S.No.} &Window Size &Early-success Criterion\\
  \hline
  \hline
  1. &$W\leq2(m_s+1)$ &Complete-\acp{vn}.\\
  \hline
  2. &$W>2(m_s+1)$ &Target \acp{vn}.\\
  \hline
\end{tabular}
\caption{Early-success criteria for \ac{ipsc}}
\label{tab:ipsc}
\end{table}
  \chapter{Simulation Results and Evaluation}\label{ch:simulation}
In this chapter, we analyze the simulation results of our proposed techniques. We start by discussing the simulation setup. Then, we analyze the plots to evaluate the performance of our techniques.

\section{Experiment Setup}
For our simulations, we used the baseband system model described in Section~\ref{sec:sys_mod}. The Table~\ref{tab:sim_param} lists the different parameters of the simulation setup. All the simulations are performed with these parameters unless otherwise specified in the plot captions.
\begin{table}[htbp]
\centering
\begin{tabular}{|l|l|}
  \hline
  \textbf{Parameter} &\textbf{Value}\\
  \hline
  \hline
  No. of information bits $n_i$ &3500\\
  \hline
  Asymptotic code rate $R_\infty$ &$2/3$\\
  \hline
  No. of termination bits $m_t$ &380\\
  \hline
  Modulation &\gls{qpsk}\\
  \hline
  Window Size $W$ &300\\
  \hline
  No. of Iterations $I$ &5\\
  \hline
  No. of Runs &4000\\
  \hline
\end{tabular}
\caption{Experimental settings for simulations.}
\label{tab:sim_param}
\end{table}

In most of our results we plot \gls{bler} instead of \gls{ber} because in modern packet based transmission system, the whole block is discarded in physical layer if they are incorrect. So, it is better to evaluate the performance by \gls{bler} than \gls{ber}.
\section{Effect of Zero-tail Termination}
In Chapter~\ref{ch:encode} we concluded that the termination for \gls{bpl} codes are performed through zero-tail termination. Here, we evaluate the effect of zero-tail termination on the probability of error of each bit in the codeword. Figure~\ref{fig:eval_no_sat} shows the probability of error of each bit in the codeword when the zero-tail bits are not known at the receiver, i.e., at the decoder the \glspl{llr} of zero-tail bits are saturated to $+\infty$. Figure~\ref{fig:eval_sat} shows the probability of bits in the codeword when zero-tail bits are known at the receiver. The probabilities are calculated using (\ref{eq:indiv_prob}) and are averaged over 4000 runs.  The plots include the information bits and the parity bits but do not include the termination sequence.
\begin{figure}[htbp]
  \centering
  \tikzsetnextfilename{eval_no_sat}
  \includegraphics[width=0.9\linewidth]{plots/eval_no_sat}
  \caption{Probability of error for information and parity bits in the codeword for $\zeta=2$ dB. Zero-tail bits are not known at the receiver.}
  \label{fig:eval_no_sat}
\end{figure}

\begin{figure}[htbp]
  \centering
  \tikzsetnextfilename{eval_sat}
  \includegraphics[width=0.9\linewidth]{plots/eval_sat}
  \caption{Probability of error for information and parity bits in the codeword for $\zeta=2$ dB. Zero-tail bits are known at the receiver.}
  \label{fig:eval_sat}
\end{figure}
 
From Figure~\ref{fig:eval_sat} we can see that when zero-tail bits are know at the receiver, the decoder reduces the $P_b(i)$ of the information and parity bits in the right of the codeword. On the other hand, Figure~\ref{fig:eval_no_sat} shows that the lack of knowledge of zero-tail bits at the receiver does not reduce $P_b(i)$ at the end of the codeword. So, the knowledge of zero-tail bits at the decoder effectively reduces the \gls{cn} degrees at the right of the \gls{pcm} as in a properly terminated codeword. Hence, zero-tail termination is an acceptable alternative to proper termination.
 
\section{Evaluation of \acrlong{bd}}
Here, we analyze the performance of our \gls{bd}. At first, we analyze the performance and complexity over different window sizes. Figure~\ref{fig:eval_bd_bler} shows the overall \gls{bler} $P_l$ over an \gls{snr} range of $2\leq\zeta\text{ (dB)}\leq 5$ for different window sizes. Figure~\ref{fig:eval_bd_aneu} shows the \gls{aneu} over the same range of \gls{snr}.
\begin{figure}[htbp]
  \centering
  \tikzsetnextfilename{eval_bd_bler}
  \includegraphics[width=0.9\linewidth]{plots/eval_bd_bler}
  \caption{\gls{bler} vs \gls{snr} of the \acrfull{bd}.}
  \label{fig:eval_bd_bler}
\end{figure}
\begin{figure}[htbp]
   \centering
   \tikzsetnextfilename{eval_bd_aneu}
  \includegraphics[width=0.8\linewidth]{plots/eval_bd_aneu}
  \caption{\gls{bler} vs \gls{snr} of the \acrfull{bd}.}
  \label{fig:eval_bd_aneu}
\end{figure}

In Figure~\ref{fig:eval_bd_bler}, we see that the \gls{bler} improves with increasing window size $W$. When the window size is increased, more \glspl{vn} are included in the window enabling information to flow between a large number of \glspl{vn}. Figure~\ref{fig:eval_bd_aneu} shows that the \gls{aneu} is high for larger window in the low \gls{snr} region as the windows do not converge. In high \gls{snr} region, larger windows converge quicker and hence, they have lower complexity than smaller windows.

Secondly, we analyze the performance over different code rates. Figure~\ref{fig:eval_bler_rate} shows different \gls{bler} plots for all available rates $R_\infty$ for \gls{bpl} codes. It is well known that the performance of the code increases with decreasing $R_\infty$.
\begin{figure}[htbp]
  \centering
  \tikzsetnextfilename{eval_bler_rate}
  \includegraphics[width=0.9\linewidth]{plots/eval_bler_rate}
  \caption{\gls{bler} vs \gls{snr} of the \acrfull{bd} with $W=500$.}
  \label{fig:eval_bler_rate}
\end{figure}

Finally, we compare the performance over $I=5,10$. Figure~\ref{fig:eval_bd_iter_bler_300} and Figure~\ref{fig:eval_bd_iter_aneu_300} shows the \gls{bd}'s performance and complexity over different number of iterations $I$ per window. We can see that the complexity is less for fewer $I=5$ but the performance of the decoder remains the same which indicates that the windows converge quicker, not utilizing all 5 iterations.
\begin{figure}[htbp]
  \centering
  \tikzsetnextfilename{eval_bd_iter_bler_300}
  \includegraphics[width=0.9\linewidth]{plots/eval_bd_iter_bler_300}
  \caption{Comparison of \gls{bler} of the \acrfull{bd} for different $I$.}
  \label{fig:eval_bd_iter_bler_300}
\end{figure}
\begin{figure}[htbp]
  \centering
  \tikzsetnextfilename{eval_bd_iter_aneu_300}
  \includegraphics[width=0.8\linewidth]{plots/eval_bd_iter_aneu_300}
  \caption{Comparison of \gls{aneu} of the \acrfull{bd} for different $I$.}
  \label{fig:eval_bd_iter_aneu_300}
\end{figure}

\section{Evaluation of \acrlong{lrl} Decoder}
Now, we compare the performance of \gls{lrl} decoder configuration-\rom{1} with the \gls{bd}. Figure~\ref{fig:eval_bd_lrl_bler} shows two plots of \gls{bler} each for \gls{bd} and \gls{lrl} decoder with window size $W=300$. Similarly, Figure~\ref{fig:eval_bd_lrl_aneu} shows two plots of \gls{aneu} each for \gls{bd} and \gls{lrl} decoder with window size of $W=300$.
\begin{figure}[htbp]
  \centering
  \tikzsetnextfilename{eval_bd_lrl_bler}
  \includegraphics[width=0.9\linewidth]{plots/eval_bd_lrl_bler}
  \caption{Comparison of \gls{bler} between the Base Decoder and \gls{lrl} decoder.}
  \label{fig:eval_bd_lrl_bler}
\end{figure}
\begin{figure}[htbp]
  \centering
  \tikzsetnextfilename{eval_bd_lrl_aneu}
  \includegraphics[width=0.8\linewidth]{plots/eval_bd_lrl_aneu}
  \caption{Comparison of \gls{aneu} between the Base Decoder and \gls{lrl} decoder.}
  \label{fig:eval_bd_lrl_aneu}
\end{figure}

From both the figures, we see a significant decrease in \gls{bler} and \gls{aneu} of the \gls{lrl} decoder. It indicates that the second phase of the \gls{lrl} decoder has improved the certainty of the bits through the information from the bits in the right of the codeword. This phenomenon can be seen in Figure~\ref{fig:eval_bd_lrl_iber} which plots the individual bit error probabilities for both decoders. Hence, the proposed \gls{lrl} decoder is better than the \gls{bd} in terms of decoding performance and complexity.
\begin{figure}[htbp]
  \centering
  \tikzsetnextfilename{eval_sat}
  \includegraphics[width=0.9\linewidth]{plots/eval_sat}
  \caption{Individual bit error probability for \gls{bd} and \gls{lrl} decoder for $W=500$ and $\zeta=2.4$ dB.}
  \label{fig:eval_bd_lrl_iber}
\end{figure}

Figure~\ref{fig:eval_bd_lrl_bler_700} and Figure~\ref{fig:eval_bd_lrl_aneu_700} compares the performance and complexity between the \gls{bd} and \gls{lrl} decoder configuration-\rom{1} for window size $W=700$. We see that as the window size increases there is an improvement in \gls{bler} but no significant decrease in the complexity. It is because with larger windows, the \gls{bp}'s performance is better as seen in Figure~\ref{fig:eval_bd_bler}. So, as the window size increases the performance of both \gls{bd} and \gls{lrl} decoder approaches the performance of full-block decoder, i.e., a decoder with maximum $W$.
\begin{figure}[htbp]
  \centering
  \tikzsetnextfilename{eval_bd_lrl_bler_700}
  \includegraphics[width=0.9\linewidth]{plots/eval_bd_lrl_bler_700}
  \caption{Comparison of \gls{bler} between the Base Decoder and \gls{lrl} decoder with $W=700$.}
  \label{fig:eval_bd_lrl_bler_700}
\end{figure}
\begin{figure}[htbp]
  \centering
  \tikzsetnextfilename{eval_bd_lrl_aneu_700}
  \includegraphics[width=0.8\linewidth]{plots/eval_bd_lrl_aneu_700}
  \caption{Comparison of \gls{aneu} between the Base Decoder and \gls{lrl} decoder with $W=700$.}
  \label{fig:eval_bd_lrl_aneu_700}
\end{figure}

Next, we compare the performance of both window configurations of \gls{lrl} decoder. Table~\ref{tab:wind_conf} shows different window configurations used in simulation.
\begin{table}[htbp]
  \centering
  \begin{tabular}{|l|l|l|}
    \hline
    \textbf{Simulation Configuration} &\textbf{First Phase} &\text{Second Phase}\\
    \hline
    \hline
    \gls{lrl} configuration-\rom{1} &Window configuration-\rom{1} &Window configuration-\rom{1}\\
    \hline
    \gls{lrl} configuration-\rom{2} &Window configuration-\rom{1} &Window configuration-\rom{2}\\
    \hline
  \end{tabular}
  \caption{Experimental settings for simulations.}
  \label{tab:wind_conf}
\end{table}
Figure~\ref{fig:eval_comp_lrl_bler_300} compares the performance and Figure~\ref{fig:eval_comp_lrl_aneu_300} compares the complexity between both \gls{lrl} configurations. We see that the \gls{lrl} decoder with configuration-\rom{2} yields the same \gls{bler} performance as configuration-\rom{1} with a slightly reduced complexity. The \gls{ber} plot from Figure~\ref{fig:eval_comp_lrl_ber_300} also indicates the same. The reduced complexity is mainly because of the bottom to top \gls{cn} update in the second phase of the \gls{lrl} decoder configuration-\rom{2}. Also since the target \gls{vn} are in the right end of the window, the window converges faster than the configuration-\rom{1}.
\begin{figure}[htbp]
  \centering
  \tikzsetnextfilename{eval_comp_lrl_bler_300}
  \includegraphics[width=0.9\linewidth]{plots/eval_comp_lrl_bler_300}
  \caption{Comparison of \gls{bler} between the \gls{lrl} decoders configuration-\rom{1} and configuration-\rom{2}.}
  \label{fig:eval_comp_lrl_bler_300}
\end{figure}
\begin{figure}[htbp]
  \centering
  \tikzsetnextfilename{eval_comp_lrl_ber_300}
  \includegraphics[width=0.9\linewidth]{plots/eval_comp_lrl_ber_300}
  \caption{Comparison of \gls{ber} between the \gls{lrl} decoders configuration-\rom{1} and configuration-\rom{2}.}
  \label{fig:eval_comp_lrl_ber_300}
\end{figure}
\begin{figure}[htbp]
  \centering
  \tikzsetnextfilename{eval_comp_lrl_aneu_300}
  \includegraphics[width=0.8\linewidth]{plots/eval_comp_lrl_aneu_300}
  \caption{Comparison of \gls{aneu} between the \gls{lrl} decoders configuration-\rom{1} and configuration-\rom{2}.}
  \label{fig:eval_comp_lrl_aneu_300}
\end{figure}

\section{Evaluation of \acrlong{ipsc} Technique}
Here, we evaluate the performance of our \gls{ipsc} technique. Figure~\ref{fig:eval_ipsc_bler_300} and Figure~\ref{fig:eval_ipsc_aneu_300} compares the performance and complexity between the \gls{bd} with different convergence criteria with $W=300$. Similarly, Figure~\ref{fig:eval_ipsc_bler_600} and Figure~\ref{fig:eval_ipsc_aneu_600} compares for $W=600$.
\begin{figure}[htbp]
  \centering
  \tikzsetnextfilename{eval_ipsc_bler_300}
  \includegraphics[width=0.9\linewidth]{plots/eval_ipsc_bler_300}
  \caption{Comparison of \gls{bler} between different early-success criteria.}
  \label{fig:eval_ipsc_bler_300}
\end{figure}
\begin{figure}[htbp]
  \centering
  \tikzsetnextfilename{eval_ipsc_aneu_300}
  \includegraphics[width=0.9\linewidth]{plots/eval_ipsc_aneu_300}
  \caption{Comparison of \gls{aneu} between different early-success criteria.}
  \label{fig:eval_ipsc_aneu_300}
\end{figure}
\begin{figure}[htbp]
  \centering
  \tikzsetnextfilename{eval_ipsc_bler_600}
  \includegraphics[width=0.9\linewidth]{plots/eval_ipsc_bler_600}
  \caption{Comparison of \gls{bler} between different early-success criteria with $W=600$.}
  \label{fig:eval_ipsc_bler_600}
\end{figure}
\begin{figure}[htbp]
  \centering
  \tikzsetnextfilename{eval_ipsc_aneu_600}
  \includegraphics[width=0.9\linewidth]{plots/eval_ipsc_aneu_600}
  \caption{Comparison of \gls{aneu} between different early-success criteria with $W=600$.}
  \label{fig:eval_ipsc_aneu_600}
\end{figure}

From the Figures~\ref{fig:eval_ipsc_bler_300}, \ref{fig:eval_ipsc_aneu_300}, \ref{fig:eval_ipsc_bler_600} and \ref{fig:eval_ipsc_aneu_600} we see that for $W\geq2(m_s+1)$, checking only the target \glspl{cn} for convergence decreases the decoding complexity for $W>2m_s+1$. Hence, for convergence criteria it is better to use complete-\glspl{cn} if $W\leq2m_s+1$ and target-\glspl{cn} if $W>2m_s+1$.
  \section{Implementation Aspects}
\begin{itemize}
  \item Compact packed bits vs uint8 bits
  \item Encoder output buffer format
  \item In decoder, different approximations and its performance vs complexity
  \item setIdxVar on the fly vs once
\end{itemize}
In this section, we will discuss how the encoder and decoder are implemented. We start with describing the encoder's implementation and reasoning the chosen method. Then we discuss about the implementation of the decoder and the implications of different implementations on the performance.
\subsection{\aclp{vn}' Memory Format}
The \acp{vn} or the bits in the codeword can be stored in two different ways:
\begin{enumerate}
  \item Byte for a bit: Each byte of memory contains eight bits out of which the \ac{lsb} represents one \ac{vn}. An example of such a storage scheme is shown below.
  \begin{figure}[htbp]
    \centering
    \includegraphics[width=\textwidth, height=\textwidth]{graphics/bit_byte}
    \caption{Three bits of value 0, 1 and 1 are stored in single byte each. The arrow indicates the position of \ac{lsb} where the bit is stored in each byte. The bits marked with x are unused.}
  \end{figure}
  This form of storage allows us to directly access and use each bit as \texttt{uint8}.
  \begin{figure}[htbp]
    \centering
    \includegraphics[width=\textwidth, height=\textwidth]{graphics/bit_byte_sp}
    \caption{Direct access and use of bits.}
  \end{figure}
  \item Packed byte of bits: Each byte of memory contains eight bits representing eight \acp{vn}. An example of such storage is shown below.
  \begin{figure}[htbp]
    \centering
    \includegraphics[width=\textwidth, height=\textwidth]{graphics/packed_bits}
    \caption{Packed bits of bytes.}
  \end{figure}
  With this form of storage, additional functions are required to access and store each bit from and to its corresponding position because the minimum quantity of bits that can be accessed from the memory at once is a byte or \texttt{char} or \texttt{uint8}. This is shown in figure~\ref{fig:packed_spb}.
  \begin{figure}[htbp]
    \centering
    \includegraphics[width=\textwidth, height=\textwidth]{graphics/packed_bits_sp}
    \caption{Access and use of bits using helper functions.}
    \label{fig:packed_spb}
  \end{figure}
\end{enumerate}
In our implementations, we use packed byte of bits format. This format reduces the memory requirements for storing information and codeword bits by a factor of eight.

\subsection{\aclp{vn} Indexing}
The \ac{llr} values of \acp{vn} are stored in an array of memory where each element is a 32-bit floating point value. During the belief propagation, these values are accessed and used in the \ac{v2c} message computation. For each layer update, only specific \ac{llr} values from the array are used. This is done by computing and storing the absolute indices 
  \chapter{Conclusion and Outlook}\label{ch:conclu}
The accomplishments of this work is summarized in this chapter. Some advices for future works in window decoding of \gls{ldpccc} are also given.

As mobile cellular technologies in the future adapt \glspl{ldpccc} as error-correcting codes, there will be needs for efficient decoding algorithms. Smart phones now-a-days come with multiple cameras and sensors hence, increasing the power consumption. Also, popular applications such as 3D games and photo editors consume more power. These factors force the mobile device manufacturers to use power efficient modems. Our simulation results proved that the developed \acrfull{lrl} decoder has a better \acrfull{bler} performance at a much lower decoding complexity than a conventional sliding-window decoder. The \acrfull{ipsc} is also proved to decrease the decoding complexity. The decrease in decoding complexity means increase in battery-power saving.

We also showed why the \gls{bpl} codes cannot be terminated normally and has to be zero-tail terminated. We also proved that the zero-tail termination effectively reduces the \gls{cn} degree and hence the \acrfull{ber}. Although the zero-tail termination decreases the \gls{cn} degree at the end termination, it is not as low as the \gls{cn} degree in the start termination. Hence, care should be taken to ensure that the codeword can be terminated in a proper manner when a code is being designed.

Several adjustments can be made \gls{lrl} decoder. One such adjustment could be to move the window once from left to middle and right to middle of the \gls{pcm}. It could arguably give better \gls{ber} performance than just moving the window once from left end to right end of the \gls{pcm}. The \gls{lrl} decoder can also be combined with other decoding techniques. Another suggestion with regard to convergence criterion is to use soft value based parity check along with the \gls{ipsc} technique.
  \newpage
  \pagenumbering{Roman}
  \setcounter{page}{\value{romanpagenumbers}}
  \bibliographystyle{IEEEtran}
  \bibliography{IEEEabrv,references}
\end{document}
