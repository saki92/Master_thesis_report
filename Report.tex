\documentclass[11pt,type=msc,colorback,accentcolor=tud1b,bigchapter,firstlineindent,longdoc,bibliography=totoc,listof=totoc]{tudthesis}
%\usepackage{ngerman}
\usepackage[english]{babel}
%\usepackage{amsfonts}
\usepackage[bookmarks]{hyperref}
\usepackage{eucal}
\usepackage{amsmath}
\usepackage{setspace}
\usepackage{tikz,pgfplots}
\usetikzlibrary{positioning,arrows,external,patterns}
\usepackage{tikzscale}
\usepackage{etoolbox}
\usepackage{float}
%\usepackage{fontspec}
\usepackage[T1]{fontenc}
\usepackage{textcomp}
\usepackage[utf8x]{inputenc}
\usepackage{titlesec}
\usepackage{graphicx}
\usepackage{pdfpages}
\usetikzlibrary{calc}
\usepackage{mathtools}
\DeclarePairedDelimiter{\ceil}{\lceil}{\rceil}
\usetikzlibrary{fit,backgrounds,automata}
\usepackage[a-1b]{pdfx}
%\usepackage[bookmarks]{hyperref}
\usepgfplotslibrary{external} 
\tikzexternalize[prefix=TikzPictures/]

%\usepackage{breakurl}
\usepackage[acronym,toc,nonumberlist]{glossaries}
\newglossary[sylg]{symbolslist}{syi}{syo}{List of Symbols}
\makeglossaries
\newglossaryentry{mbm}{name=\ensuremath{\mathbf{m}}, symbol={}, description={Vector of source encoded (information) bits}, type=symbolslist}
\newglossaryentry{mbd}{name=\ensuremath{\mathbf{d}}, symbol={}, description={Vector of source bits}, type=symbolslist}
\newglossaryentry{mbx}{name=\ensuremath{\mathbf{x}}, symbol={}, description={Vector of codeword bits}, type=symbolslist}
\newglossaryentry{mbu}{name=\ensuremath{\mathbf{u}}, symbol={}, description={Vector of modulated symbols}, type=symbolslist}
\newglossaryentry{mbv}{name=\ensuremath{\mathbf{v}}, symbol={}, description={Vector of received symbols}, type=symbolslist}
\newglossaryentry{mbz}{name=\ensuremath{\mathbf{z}}, symbol={}, description={Vector of noise samples}, type=symbolslist}
\newglossaryentry{mby}{name=\ensuremath{\mathbf{y}}, symbol={}, description={Vector of received codeword bits}, type=symbolslist}
\newglossaryentry{mhbm}{name=\ensuremath{\widehat{\mathbf{m}}}, symbol={}, description={Vector of estimated information bits}, type=symbolslist}
\newglossaryentry{mhbx}{name=\ensuremath{\widehat{\mathbf{x}}}, symbol={}, description={Vector of estimated codeword bits}, type=symbolslist}
\newglossaryentry{mhbd}{name=\ensuremath{\widehat{\mathbf{d}}}, symbol={}, description={Vector of estimated source bits}, type=symbolslist}
\newglossaryentry{mbb}{name=\ensuremath{\mathbf{b}}, symbol={}, description={Vector of parity bits}, type=symbolslist}
\newglossaryentry{mR}{name=\ensuremath{R}, symbol={}, description={Actual Code Rate}, type=symbolslist}
\newglossaryentry{mbG}{name=\ensuremath{\mathbf{G}}, symbol={}, description={Generator Matrix}, type=symbolslist}
\newglossaryentry{mbH}{name=\ensuremath{\mathbf{H}}, symbol={}, description={Parity-Check Matrix}, type=symbolslist}
\newglossaryentry{mbs}{name=\ensuremath{\mathbf{s}}, symbol={}, description={Syndrome}, type=symbolslist}
\newglossaryentry{mc}{name=\ensuremath{c_i}, symbol={}, description={$i$-th Check Node}, type=symbolslist}
\newglossaryentry{mv}{name=\ensuremath{v_i}, symbol={}, description={$i$-th Variable Node}, type=symbolslist}
\newglossaryentry{mn}{name=\ensuremath{n}, symbol={}, description={Number of bits in a Code Block}, type=symbolslist}
\newglossaryentry{mk}{name=\ensuremath{k}, symbol={}, description={Number of information bits in a Code Block}, type=symbolslist}
\newglossaryentry{mq}{name=\ensuremath{q}, symbol={}, description={Variable Node degree}, type=symbolslist}
\newglossaryentry{mr}{name=\ensuremath{r}, symbol={}, description={Check Node degree}, type=symbolslist}
\newglossaryentry{mRi}{name=\ensuremath{R_\infty}, symbol={}, description={Asymptotic Code Rate}, type=symbolslist}
\newglossaryentry{mL}{name=\ensuremath{L}, symbol={}, description={Coupling Length}, type=symbolslist}
\newglossaryentry{mlc}{name=\ensuremath{l_c}, symbol={}, description={Constraint Length}, type=symbolslist}
\newglossaryentry{mnz}{name=\ensuremath{n_z}, symbol={}, description={Number of Zero-tail bits in the codeword}, type=symbolslist}
\newglossaryentry{mnm}{name=\ensuremath{n_m}, symbol={}, description={Number of information bits in the codeword}, type=symbolslist}
\newglossaryentry{mpdf}{name=\ensuremath{p}, symbol={}, description={\acrlong{pdf}}, type=symbolslist}
\newglossaryentry{mprob}{name=\ensuremath{\text{Pr}}, symbol={}, description={Probability}, type=symbolslist}
\newglossaryentry{mllr}{name=\ensuremath{\mathcal{L}}, symbol={}, description={\acrfull{llr}}, type=symbolslist}
\newglossaryentry{mpsi}{name=\ensuremath{\psi}, symbol={}, description={Time taken to perform one CN Processing}, type=symbolslist}
\newglossaryentry{mkapa}{name=\ensuremath{\kappa}, symbol={}, description={Decoder's output bit rate}, type=symbolslist}
\newglossaryentry{mrho}{name=\ensuremath{\rho}, symbol={}, description={Window position in a window decoder}, type=symbolslist}
\newglossaryentry{mms}{name=\ensuremath{m_s}, symbol={}, description={Memory of a \glspl{ldpccc}}, type=symbolslist}
\newglossaryentry{mW}{name=\ensuremath{W}, symbol={}, description={Window Size}, type=symbolslist}
\newglossaryentry{mo}{name=\ensuremath{o}, symbol={}, description={Modulation Order}, type=symbolslist}
\newglossaryentry{mI}{name=\ensuremath{I}, symbol={}, description={Maximum number of iterations allowed within a window}, type=symbolslist}
\newglossaryentry{msnr}{name=\ensuremath{\zeta}, symbol={}, description={\acrfull{snr} in dB}, type=symbolslist}
\newglossaryentry{meta}{name=\ensuremath{\eta}, symbol={}, description={\acrfull{aneu}}, type=symbolslist}
\newglossaryentry{mber}{name=\ensuremath{P_b}, symbol={}, description={Probability of bit error}, type=symbolslist}
\newglossaryentry{mbler}{name=\ensuremath{P_L}, symbol={}, description={Probability of block error}, type=symbolslist}

\newacronym{fec}{FEC}{Forward Error Correction}
\newacronym{ldpc}{LDPC}{Low-Density Parity-Check}
\newacronym{ldpcbc}{LDPC-BC}{Low-Density Parity-Check Block Code}
\newacronym{ldpccc}{LDPC-CC}{Low-Density Parity-Check Convolutional Code}
\newacronym{scldpc}{SC-LDPC}{Spatially-Coupled LDPC}
\newacronym{pcm}{PCM}{Parity Check Matrix}
\newacronym{cn}{CN}{Check Node}
\newacronym{vn}{VN}{Variable Node}
\newacronym{bp}{BP}{Belief Propagation}
\newacronym{cb}{CB}{Code Block}
\newacronym{bpl}{BPL}{Broadband Power Line}
\newacronym{bc}{BC}{Block Code}
\newacronym{gm}{GM}{Generator Matrix}
\newacronym{cw}{CW}{codeword}
\newacronym{awgn}{AWGN}{Additive White Gaussian Noise}
\newacronym{snr}{SNR}{Signal-to-Noise Ratio}
\newacronym{phy}{PHY}{Physical}
\newacronym{ml}{ML}{Maximum-Likelihood}
\newacronym{map}{MAP}{Maximum-A-Posteriori}
\newacronym{mpa}{MPA}{Message-Passing Algorithm}
\newacronym{spa}{SPA}{sum-product algorithm}
\newacronym{llr}{LLR}{Log-likelihood ratio}
\newacronym{v2c}{V2C}{variable-node-to-check-node}
\newacronym{c2v}{C2V}{check-node-to-variable-node}
\newacronym{msa}{MSA}{Min-Sum Algorithm}
\newacronym{wd}{WD}{Windowed Decoding}
\newacronym{qam}{QAM}{Quadrature Amplitude Modulation}
\newacronym{qpsk}{QPSK}{Quadrature Phase Shift Keying}
\newacronym{pdf}{PDF}{Probability Density Function}
\newacronym{ber}{BER}{Bit-Error Rate}
\newacronym{bler}{BLER}{Block-Error Rate}
\newacronym{lsb}{LSB}{Least Significant Bit}
\newacronym{osi}{OSI}{Open System Interconnection}
\newacronym{fir}{FIR}{Finite Impulse Response}
\newacronym{iir}{IIR}{Infinite Impulse Response}
\newacronym{bd}{BD}{Base Decoder}
\newacronym{id}{ID}{Improved Decoder}
\newacronym{psc}{PSC}{Partial-Syndrome-Check}
\newacronym{ipsc}{IPSC}{Improved Partial-Syndrome-Check}
\newacronym{aneu}{ANEU}{Average Number of Edge Updates}
\newacronym{lrl}{LRL}{Left-Right-Left}
\newacronym[longplural={A-Posteriori Probabilities}]{app}{APP}{A-Posteriori Probability}
\newacronym{bpsk}{BPSK}{Binary Phase Shift Keying}
\newacronym{wimax}{WiMAX}{Worldwide Interoperability for Microwave Access}
\newacronym{lan}{LAN}{Local Area Network}
\newacronym{bec}{BEC}{Binary Errasure Channel}
\newacronym{nr}{NR}{New Radio}
\newacronym{dvb}{DVB}{Digital Video Broadcasting}
%\tikzexternalize[mode=list and make]{Report}
\newcommand{\getmydate}{%
	\ifcase\month%
	\or Januar\or Februar\or M\"arz%
	\or April\or Mai\or Juni\or Juli%
	\or August\or September\or Oktober%
	\or November\or Dezember%
	\fi\ \number\year%
}

\renewcommand{\H}{\mathbf{H}}
\renewcommand{\G}{\mathbf{G}}
\newcommand{\x}{\mathbf{x}}
\newcommand{\m}{\mathbf{m}}
\newcommand{\bb}{\mathbf{b}}
\newcommand{\cnval}{c}
\newcommand{\vnval}{v}
\newcommand{\sdot}{}
\newcommand{\xs}{0.0001}
\newcommand*{\pdf}{p}
\newcommand*{\prob}{\text{Pr}}
\newcommand{\mc}{\mathcal{c}}
\DeclareMathOperator*{\argmax}{argmax}
\DeclareMathOperator{\arctanh}{atanh}
\DeclareMathOperator{\re}{Re}
\DeclareMathOperator{\im}{Im}
\DeclareMathOperator{\sign}{sign}
\renewcommand{\arraystretch}{1.5} %for table row spacing
% General scaling (fiddle with value to find ideal setting)
\newcommand{\plotscale}{1}

% Golden ratio as axes aspect ratio
\newcommand{\plotratio}{0.618}

% Define line width within plots
\newlength{\mylinewidth}
\setlength{\mylinewidth}{0.6pt}

% Define font size within plots
\newcommand{\plotfontsize}{\footnotesize}

% Apply settings to all figures
\tikzset
{
  ultra thin/.style= {line width=0.25\mylinewidth},
  very thin/.style=  {line width=0.5\mylinewidth},
  thin/.style=       {line width=\mylinewidth},
  semithick/.style=  {line width=1.5\mylinewidth},
  thick/.style=      {line width=2\mylinewidth},
  very thick/.style= {line width=3\mylinewidth},
  ultra thick/.style={line width=4\mylinewidth},
  every picture/.style={thin},
  every node/.append style={font=\plotfontsize}
}

% Mostly for changing the minor-grid-line widths
\pgfplotsset
{
  compat=newest,
  minor grid style={black!10},
  every axis/.append style=
  {
    label style={font=\plotfontsize},
    tick label style={font=\plotfontsize},
    legend style={font=\plotfontsize}
  }
}

% Line width for dsp figures
\newlength{\tikzthin}
\setlength{\tikzthin}{\mylinewidth}

% For inputting pgf files generated from Octave with matlab2tikz
\usepgfplotslibrary{patchplots}
\pgfplotsset{plot coordinates/math parser=false}
\newlength{\fwidth}
\newlength{\fheight}
\setlength{\fwidth}{\plotscale\linewidth}
\setlength{\fheight}{\plotratio\fwidth}

% Scale unscalable tikz figures
\let\OrgPgfTransformScale\pgftransformscale

% Color definitions for document-wide use
\definecolor{mycolor1}{rgb}{0.00000,0.44700,0.74100}
\definecolor{mycolor2}{rgb}{0.85000,0.32500,0.09800}
\definecolor{mycolor3}{rgb}{0.92900,0.69400,0.12500}
\definecolor{mycolor4}{rgb}{0.49400,0.18400,0.55600}
\definecolor{mycolor5}{rgb}{0.46600,0.67400,0.18800}
\definecolor{mycolor6}{rgb}{0.30100,0.74500,0.93300}
\definecolor{mycolor7}{rgb}{0.63500,0.07800,0.18400}

\newcommand*{\rom}[1]{\uppercase\expandafter{\romannumeral #1\relax}}
\begin{document}
	\colorlet{tudidentbar}{tud1b} 
	\thesistitle{Design and Implementation of Improved Decoding Algorithms for LDPC Convolutional Codes}%
	{Entwurf und Implementierung von verbesserten Decodieralgorithmen f{\"u}r LDPC-Faltungscodes }
	\author{Sakthivel Velumani}
	\birthplace{Salem, Indien}
	\referee{Janik Frenzel, M.Sc.}{Bastian Alt, M.Sc.}[Prof. Dr. techn. Heinz K\"oppl]
	\department{Fachbereich Elektrotechnik und Informationstechnik}
	\group{Bioinspired Communication Systems Lab}
	\date{\today}
	\dateofexam{14.01.2019}{14.01.2019}
	\tuprints{82349}{8234}
	\makethesistitle
  \pagenumbering{Roman}
	%\colorlet{tudidentbar}{tud0b} 
	\affidavit{Sakthivel Velumani}
  \newpage
  \chapter*{Acknowledgment}


  %\onehalfspacing
  \chapter{Abstract}

	\tableofcontents
	\newpage
  \listoffigures
  \newpage
  \listoftables
  \newpage

  \newglossarystyle{mystyle}{%
    \renewcommand*{\glspostdescription}{}%
    \renewenvironment{theglossary}{\begin{longtable}[l]{@{}lcl}}{\end{longtable}}
    \renewcommand*{\glossaryheader}{}%
    \renewcommand*{\glsgroupheading}[1]{}%
    \renewcommand*{\glsgroupskip}{}%
    \renewcommand*{\glossaryentryfield}[5]{%
      \glstarget{##1}{##2}	% Name
      & ##4					        % Symbol
      & ##3					        % Description
      % & ##5					      % Page list
      \\% end of row
    }%
    \renewcommand*{\glossarysubentryfield}[6]{%
      \glossaryentryfield{##2}{##4}{##3}
    }%
  }

  % Print acronyms
  \printglossary[type=acronym, title=Acronyms, toctitle=Acronyms, style=mystyle]
  % Print list of symbols
  \printglossary[type=symbolslist, style=mystyle]
  \newpage
  % Save current value of roman page number
  \newcounter{romanpagenumbers}
  \setcounter{romanpagenumbers}{\value{page}}
  \pagenumbering{arabic}
	\chapter{Introduction}
\glspl{ldpccc} are a class of linear block codes which are widely used in data transmission systems. Terminated \glspl{ldpccc} are proven to be better than \glspl{ldpcbc} because of their convolutional nature and larger codeword lengths. Along side the code performance, the \gls{ldpccc} have an advantage to its decodability. Its convolutional structure allows for window decoding which requires fewer resources than a full-block decoder. However, window decoding always have a trade-off between decoding latency and decoding performance.

\glspl{ldpccc} are used in applications like \gls{bpl} (IEEE 1901) and \gls{wimax} (IEEE 802.16) standards to name a few. \gls{ldpcbc} are being used in more widely used applications like wireless \gls{lan}, \gls{dvb} and the upcoming \gls{nr}. But due to the better performance and increasing popularity of \gls{ldpccc}, their applications are more likely to be increased in the future.
\begin{figure}[htbp]
  \centering
  \tikzsetnextfilename{mob_gen}
  \includegraphics[width=\textwidth, height=\textwidth]{graphics/mob_gen}
  \caption{Evolution of mobile telephone technologies and type of error-correcting codes used.}
  \label{fig:mob_gen}
\end{figure}

This thesis is focused on developing improvement techniques for window decoders. There are quite a number of window decoding algorithms that are proposed in recent years. One of them is the \emph{Zigzag decoder} that was proposed by Abu-Surra in~\cite{Shadi2015}. The decoder moves the window forward and backward in small parts of the \gls{pcm}. Another decoder proposed in~\cite{Kang2018} used a technique to reduce the incorrect information about the bits propagating from left to right of the codeword as the window moves. Three more improvement techniques for window decoder were proposed in~\cite{Ali2018} which are discussed in later chapters. The aforementioned decoding techniques either use flooding schedule or are evaluated with \gls{bec}. We are interested in serial scheduling and \gls{awgn} channel.

In this thesis, we propose a \gls{lrl} decoder with two configurations and an \gls{ipsc} technique. The \gls{lrl} decoder which is motivated by the Zigzag decoder and moves the window forward and backward within the \gls{pcm}. The \gls{ipsc} is an improvement over \gls{psc} proposed in~\cite{Kang2018}. We also investigate the termination problem in the \gls{ldpccc} used in the standard IEEE 1901. We evaluate the proposed techniques with an \gls{awgn} channel system model.

The organization of this thesis is as follows. Chapter~\ref{ch:back} contains the theory of channel coding and channel codes with an emphasis on \gls{ldpccc}. We also discuss the system model used for our simulations. In Chapter~\ref{ch:encode}, we discuss the termination problem in IEEE 1901's \gls{ldpccc}. We then discuss the proposed decoding techniques and their evaluation in Chapters~\ref{ch:dec_improve} and \ref{ch:simulation}, respectively. Finally in Chapter~\ref{ch:conclu}, we conclude our findings of this and discuss about possible future works.
  \section{Background}
In this section, we provide an overview of channel coding and some selected channel codes. We start by discussing the need for channel coding. We then describe linear block codes, \ac{ldpc} codes and \acp{ldpccc} which are the main focus of this thesis. Then we go on to discuss \ac{bp} and the sliding-window technique used in decoders for \acp{ldpccc}. We finish the section describing our system model.

\subsection{Introduction to Channel Coding}
The term \emph{coding} is generally associated with the mapping of information to a set of symbols or numbers~\cite{Bossert}. Source coding aims to compress the information whereas channel coding aims to make the information immune to random distortion. A model of a digital communication system is shown in Figure~\ref{fig:chanCoding}. Let us consider that the \emph{source} block produces a sequence of information bits given by the vector $\mathbf{d}$. These bits might stream from any digital information source such as multimedia files, text documents, etc. These vectors of bits are encoded in the \emph{source-encoder} block to produce compressed information bit vectors $\mathbf{m}$ called source codewords. The compression means that the length of $\mathbf{m}$ is at most the length of $\mathbf{d}$. The mapping allow unique reconstruction of the information bits at the receiver. The source encoder is chosen depending on the type of the information source~\cite{proak}.

The next block in the digital communication model is the \emph{channel encoder}. Whereas the source encoder compresses the information bit vectors, the channel encoder expands them by adding redundant bits in a structured manner. This structured redundancy makes the transmitted information bits less susceptible to distortions such as interference in the channel medium, and receiver noise. A \emph{channel} is a physical medium through which the information is transferred from transmitter to receiver. A \emph{code} is a set of rules that defines the encoding principle of the encoder. The type of code is chosen depending on the channel and the application requirements. In general, the source encoder or decoder is placed at higher layers of the \ac{osi} model, while channel-coding blocks are placed at the \ac{phy} layer. The outputs of the channel encoder are called channel codewords. The codeword vectors $\mathbf{x}$ are then modulated in the \emph{modulator} block where the bits are transformed into symbol vectors $\mathbf{u}$. The symbol vectors are then transmitted as analog signals through the channel. Due to the addition of interference and noise, the channel output is in general not the same as the channel input: $\mathbf{v}\neq\mathbf{u}$. The \emph{demodulator} converts the received symbol vectors $\mathbf{v}$ into vectors of bits $\mathbf{y}$ which corresponds to the vector of encoded bits $\mathbf{x}$. The \emph{channel decoder} uses the redundancy in the received codeword to deduce an estimate $\widehat{\mathbf{m}}$ of the source codeword. The source decoder then deduces an estimate $\widehat{\mathbf{d}}$ of the information bit vector from $\widehat{\mathbf{m}}$.

\begin{figure}[htbp]
  \centering
  \includegraphics[width=\textwidth, height=\textwidth]{graphics/channel_coding}
  \caption{Block diagram of a digital communication system.}
  \label{fig:chanCoding}
\end{figure}

The addition of redundant bits by the channel encoder enables the mapping between a set of information words and a set of all possible \emph{receive words}. Lets us consider the length of an information word $\mathbf{m}$ to be $k$ bits and the length of a codeword $\mathbf{x}$ to be $n$ bits such that $n>k$. Thus the information word set has $2^k$ words and the receive word set has $2^n$ words. The codeword set of size $2^k$ is a subset of the receive word set. The mapping between different set sizes allows us to detect if the received word is in the codeword set. The information words and codewords contain elements in the binary set $\mathbb{F}_2=\{0,1\}$ as their alphabets. $\mathbb{F}_2$ or GF(2) is called a finite field of order $2$. Hence, all arithmetic operations with information bits and codewords are performed modulo 2.

\subsection{Channel Codes}
There are different types of channel codes. The choice of one depends on the application requirements, type of channel medium and resource availability. In this thesis, we focus on \acp{ldpccc}, a special class of \acp{ldpcbc}.
\subsubsection{Linear Block Codes}
Linear block codes are codes in which a codeword is formed by a linear combination of two or more base vectors that span the codeword space~\cite{proak} and hence the base vectors are also codewords. As a result, a linear combination of any two or more codewords forms another codeword. The codeword space of $2^k$ vectors is a subspace of the space of all $2^n$ vectors. An $(n,k)$ linear block code maps $k$ message bits to $n$ codeword bits. The remaining $n-k$ redundant bits are called parity bits and they are determined by an encoding rule. Linear block codes are classified into two categories: \emph{systematic} and \emph{non-systematic}. Systematic codes have all their message bits transmitted in an unaltered manner whereas the non-systematic codes do not have such formation. A codeword of a systematic linear block code can have one of the following structures: $$\x^T=(\m^T,\bb^T)$$ or $$\x^T=(\bb^T,\m^T)$$ where $\x\in\mathbb{F}_2^{n\times 1}$ is the codeword vector, $\m\in\mathbb{F}_2^{k\times 1}$ and $\bb\in\mathbb{F}_2^{(n-k)\times 1}$ denote message and parity vectors, respectively. The code rate is given by $$R=\frac{k}{n}.$$

Codewords of linear block codes are expressed using the linear expression $$\x=\G\odot\m$$ where $\G\in\mathbb{F}_2^{n\times k}$ is called the \ac{gm} and $\odot$ represents multiplication modulo 2. A parity check is described by the expression $$\H\odot\x=\mathbf{s}$$ where $\H\in\mathbb{F}_2^{(n-k)\times n}$ is called the \ac{pcm} and $\mathbf{s}\in\mathbb{F}_2^{(n-k)\times 1}$ is called the syndrome. Each row of the \ac{pcm} represents a parity-check equation. Only when $\mathbf{s}=\mathbf{0}$, the parity checks are fulfilled. The relation between \ac{pcm} and \ac{gm} is given by $\H\odot\G=\mathbf{0}$. With either \ac{gm} or \ac{pcm} given, the other one is not unique. For example, if the \ac{pcm} of a $(7,4)$ hamming code is given by
\begin{align} \label{eq:H_ham}
\H =
\begin{bmatrix}
1 &1 &1 &0 &1 &0 &0 \\
1 &1 &0 &1 &0 &1 &0 \\
1 &0 &1 &1 &0 &0 &1
\end{bmatrix},
\end{align}
then the \ac{gm} can be formed by combining any 3 rows of null$(\H)$, i.e., the right null space of $\H$.
%\begin{align} \label{eq:G_ham}
%\text{null}(\H) =
%\begin{bmatrix}
%1 &1 &0 &0 &0 &0 &1 \\
%1 &1 &1 &0 &1 &0 &0 \\
%0 &0 &1 &1 &1 &1 &0 \\
%0 &1 &1 &1 &0 &0 &0
%\end{bmatrix}
%\end{align}

The \ac{pcm} can be represented by a bipartite graph, called Tanner graph~\cite{Tanner1981}. The Tanner graph has two sets of nodes: \acp{vn} represent columns and \acp{cn} represent rows of the \ac{pcm}. Each non-zero entry in the \ac{pcm} is represented by an edge between the respective \ac{vn} and \ac{cn}. The \emph{degree} of a node is the number of edges connected to it. The Tanner graph of the example \ac{pcm} in (\ref{eq:H_ham}) is shown in Figure \ref{fig:tannGraph}.

\begin{figure}[htbp]
  \centering
  %\tikzsetnextfilename{tanner_graph}
  \includegraphics[width=0.5\textwidth, height=0.2\textwidth]{graphics/tanner_graph}
  \caption{Tanner graph of the code from (\ref{eq:H_ham}). The dark shaded circles represents \acp{vn} and the crossed circles \acp{cn}.}
  \label{fig:tannGraph}
\end{figure}

\subsubsection{\acl{ldpc} Block Codes}
\acp{ldpcbc} are a class of linear block codes which were introduced by Robert Gallager in 1963~\cite{Gallager1963}. As the name specifies, they are defined by a sparse \ac{pcm} containing mostly 0's and relatively few 1's. The sparsity of the \ac{pcm} or its Tanner graph is a key property that allows for the algorithmic efficiency of \acp{ldpcbc}. These codes are divided into two types: regular and irregular codes.

In a regular $(n,q,r)$ code, all \acp{vn} have degree $q$ and all \acp{cn} have degree $r$.

\subsubsection{\acl{ldpc} Convolutional Codes}
Convolutional codes in general are codes in which the parity bits are generated by convolving information bits or information and parity bits. The rule for the convolution is expressed by generator-polynomial. This generator-polynomial is similar to the taps of an \ac{fir} filter in case of non-recursive codes and an \ac{iir} filter in case of recursive codes. An example of parity bit generation in a non-recursive systematic convolutional code is shown in Figure~\ref{fig:conv_code}.
\begin{figure}[htbp]
  \centering
  %\tikzsetnextfilename{tanner_graph}
  \includegraphics[width=0.5\textwidth, height=0.2\textwidth]{graphics/conv_code}
  \caption{Example of a non-recursive convolutional code with rate $R=1/2$. $x[k]$ is the input and $y[k]$ is the output.}
  \label{fig:conv_code}
\end{figure}
The generator-polynomials of this example is given by
\begin{align*}
G^{(0)}(D)&=1\\
G^{(1)}(D)&=1+D^2.
\end{align*}
The impulse response of the parity-bit generator is given by
$$g^{(1)}[k]=\begin{bmatrix}
1 &0 &1
\end{bmatrix}.$$
The output is given by the convolution form:
\begin{align}
y^{(1)}[k]&=x[k]*g^{(1)}[k]\nonumber\\
&=\sum_{l=0}^{2}x[l]g^{(1)}[k-l].
\end{align}
The \emph{constraint length} of a convolutional code is $l_c=m_s+1$ where $m_s$ is the largest degree in the generator-polynomial. In the example in Figure~\ref{fig:conv_code}, the constraint length is 3.

\acp{ldpccc} or \ac{scldpc} codes are formed by imposing the above mentioned convolutional structure on \acp{ldpcbc}. They were invented by Alberto Felstr{\"o}m and Kamil Zigangirov~\cite{Felstrom1999}. These codes are characterized by a sparse infinite-length \ac{pcm} which has a diagonal structure. The \ac{pcm} of these codes is constructed by coupling \acp{pcm} of \acp{ldpcbc} as given by

\begin{align}\label{eq:H_infty}
\H_{[-\infty,\infty]} =& 
\begin{bmatrix}
  \ddots &\ddots &\ddots &\ddots\\
  &\H_{m_s}(t-1) &\dots &\H_1(t-1) &\H_0(t-1)\\
  & &\H_{m_s}(t) &\dots &\H_1(t) &\H_0(t)\\
  & & &\H_{m_s}(t+1) &\dots &\H_1(t+1) &\H_0(t+1)\\
  & & & &\ddots &\ddots &\ddots &\ddots
  \end{bmatrix}
  \begin{matrix}
  \mathbf{s}(t-1)\\
  \mathbf{s}(t)\\
  \mathbf{s}(t+1)\\
  \end{matrix}
\end{align}
where the $\H_\mu(t)\in\mathbb{F}_2^{(n-k)\times n},\mu=0,\dots,m_s$ are \acp{pcm} of different \acp{ldpcbc} of rate $R=k/n$ for different time instances and $m_s$ is the memory of the code. Hence, the asymptotic rate of the resulting \ac{ldpccc} is $R_\infty=k/n$. $\mathbf{s}(t)\in\mathbb{F}_2^{(n-k)\times 1}$ denotes the syndromes resulting from the parity check equations. The codewords of such a code have the form $\x^T=(\dots,\x(t-1)^T,\x(t)^T,\x(t+1)^T,\dots)$ where each $\x(t)\in\mathbb{F}_2^{n\times 1}$. Given the \ac{pcm} $\H$ and a valid codeword $\x$, the following expression holds:
\begin{align}\label{eq:ldpccc_conv}
\mathbf{s}(t)=\sum_{\tau=0}^{m_s}\H_\tau(t)\x(t-\tau)\mod 2.
\end{align}
The equation (\ref{eq:ldpccc_conv}) is a convolution representing the convolutional structure of $\H$ in (\ref{eq:H_infty}).

The bits in the codeword $\x$ are coupled together over a distance called the \emph{constraint length} which is given by $l_c=(m_s+1)n$ bits.

\subsubsection{Termination of Convolutional Codes}
In general, \acp{ldpccc} have codewords and \acp{pcm} of infinite length. For packet-based communication networks, however, the whole packet has to be retransmitted in case of incorrect information bits in higher layers. Also, in a wireless medium the channel parameters change over time which requires the encoder to change its code rate on the fly. For the aforementioned reasons, terminated codes are a better choice.

Termination is the process of limiting the coupling length, so that the codewords have finite length. This allows the decoder to stop decoding the current received word if a bit cannot be corrected, thus reducing the decoding complexity. The termination process requires adding \emph{termination bits} to the end of the codeword to ensure that the last $m_s$ parity-check equations of the terminated \ac{pcm} are fulfilled. Termination also ensures that the encoder returns to an all-zero state before encoding the next codeword and so termination bits are determined by solving a system of linear equations in $\mathbb{F}_2$. 

Termination introduces a rate loss because the termination bits are transmitted which are known at the receiver. Hence for the rate calculation of a terminated \ac{ldpccc}, the termination bits are not taken into account. However, the rate loss is compensated by an increase in decoding performance as the termination reduces the \ac{cn} degrees at the end of the codeword and smaller \ac{cn} degrees are better.

The \ac{pcm} of a terminated \ac{ldpccc} is a sub-matrix of the infinitely long \ac{pcm} of the code (\ref{eq:H_infty}). The terminated \ac{pcm} has a structure as given by

\begin{align}
\H_L = 
\overbrace{\begin{bmatrix}
  \H_{0}(0)\\
  \H_1(1) &\H_0(1)\\
  \vdots &\H_1(2) &\ddots\\
  \H_{m_s}(m_s) &\vdots &\ddots &\H_0(L-1)\\
  &\H_{m_s}(m_s+1) &\ddots &\H_{1}(L)\\
  & & &\vdots\\
  & & &\H_{m_s}(L+m_s)
\end{bmatrix}}^{Ln}
\left.\begin{matrix}
\\
\\
\\
\\
\\
\\
\\
\\
\end{matrix}\right\}(L+m_s)(n-k)
\end{align}
where $L$ is the \emph{coupling length} denoting the number of \acp{cb} in the codeword. Each \ac{cb} contains $n$ bits. Hence, the total length of the terminated codeword is $n_L=Ln$ bits. The effect of termination in the Tanner graph of a $R_\infty=1/2$ code is shown in Figure~\ref{fig:tannGraphLdpccc}.
\begin{figure}[htbp]
  \centering
  %\tikzsetnextfilename{tanner_graph}
  \includegraphics[width=0.5\textwidth, height=0.2\textwidth]{graphics/tanner_graph_ldpccc}
  \caption{Tanner graph of a terminated \ac{ldpccc}. The dark circles and lines are the \acp{vn} and edges of terminated code, the light circles and dashed lines are the omitted \acp{vn} and edges as a result of termination.}
  \label{fig:tannGraphLdpccc}
\end{figure}
The entire graph in Figure~\ref{fig:tannGraphLdpccc} can be seen as a Tanner graph of an infinitely long \ac{ldpccc}. As a result of termination, only the center part of the graph remains. The dark circles are the \acp{vn} of the terminated \ac{ldpccc} and the solid lines are their corresponding edges.

\subsubsection{\aclp{ldpccc} Used in IEEE 1901}
In this thesis, we use the \acp{ldpccc} specified in the \ac{bpl} or IEEE 1901 standard to evaluate our decoder~\cite{Bpl}. From now on, we refer to the \acp{ldpccc} in the IEEE 1901 standard as \emph{\ac{bpl}~codes}. The \ac{bpl} codes are specified as sets of parity-check polynomials for all asymptotic rates $R_\infty=k/n,\ n\in\{2,3,4,5\}$ where $k=n-1$. In other words, the \ac{bpl} codes have only one parity bit in each \ac{cb}. 

The codes are defined as parity-check polynomials expressed as
\begin{align}
\sum_{i=1}^{k}A_{i,\tau}(D)M_i(D)+\sum_{i=1}^{n-k}C_{i,\tau}(D)B_i(D)=0\mod 2
\end{align}
where $k$ is the number of message bits in each \ac{cb}, $\tau \in \{0,\dots,T-1\}$ is the phase of the code that is given by $\tau=(t\ \text{mod}\ T)$, $T$ is the periodicity of the codes, $M_i(D),i=1,\dots,k$ represents message bits and $B_i(D),i=1,\dots,n-k$ represents parity bits, $A_{i,\tau}$ and $C_{i,\tau}$ defines the connection between the bits based on delay $D$.

The memory $m_s$ of the code is
\begin{align}
m_s=\max\left(\{\deg(A_{i,\tau}(D)):i=1,\dots,k;\forall\tau\}\cup\{\deg(C_{i,\tau}(D)):i=1,\dots,n-k;\forall\tau\}\right)
\end{align}
where $\tau \in \{0,\dots,T-1\}$ and deg$(f(x))$ denotes the set of all degrees of $x$ in $f(x)$.

The \ac{bpl} codes are periodic with $T=3$. Periodic codes have time-varying parity-check polynomials which repeat every $T$ \acp{cb}. For illustration, the parity-check polynomial of the \ac{bpl} code for $R_\infty=2/3$ and $\tau=0$ is given by
\begin{align}
&(D^{214}+D^{185}+1)M_1(D)+(D^{194}+D^{67}+1)M_2(D)+(D^{215}+D^{145}+1)B(D)=0\mod 2.
\end{align}
$m_s=215$ for $n=2$ and $m_s=226$ for $n\neq2$. All $A_{i,\tau}$ and $C_{i,\tau}$ have three taps for each bit in the \ac{cb}. So there is a maximum of $3n$ taps or edges per \ac{cn}. The parity-check equation can be given by
\begin{align}\label{eq:parity_check}
\sum_{i=0}^{m_s}\mathbf{h}_{M,i}\m(t-i)+\sum_{i=0}^{m_s}h_{B,i}b(t-i) \mod 2,
\end{align}
The encoder generates one parity bit per $n-1$ message bits and the parity bit expression is given by rearranging equation (\ref{eq:parity_check}) as
\begin{align}
b(t)=\sum_{i=0}^{m_s}\mathbf{h}_{M,i}\m(t-i)+\sum_{i=1}^{m_s}h_{B,i}b(t-i) \mod 2,
\end{align}
where $b(t)\in\mathbb{F}_2$ is the parity bit and $\m(t)\in\mathbb{F}_2^{k\times 1}$ is a vector of message bits at $t$-th time instance or \ac{cb}, $h_{B,i}\in\mathbb{F}_2$ is the coefficient of polynomial of the parity bit. $\mathbf{h}_{M,i}\in\mathbb{F}_2^{1\times k}$ is a vector of coefficients of polynomials of message bits and is given by
\begin{align}
\mathbf{h}_{M,i}&= \big[[D^i]A_{1,\tau},\dots,[D^i]A_{k,\tau}\big],
\end{align}
where $[D^i]A_{1,\tau}$ represents the coefficient of $D^i$ in polynomial $A_{1,\tau}(D)$.

The resulting codewords $\x\in\mathbb{F}_2^{Ln\times 1}$ have a systematic structure given by $$\x^T=(\x(0)^T,\x(1)^T,\dots,\x(L-1)^T).$$

Similarly, the \ac{pcm} can be formed using the coefficients of the parity-check polynomials:
\begin{align}\label{eq:bpl_pcm}
\H=
\begin{bmatrix}
\mathbf{h}_{M,0}\\
\mathbf{h}_{M,1} &\mathbf{h}_{M,0}\\
\vdots &\mathbf{h}_{M,1} &\ddots\\
\mathbf{h}_{M,m_s} &\vdots &\ddots &\mathbf{h}_{M,0}\\
&\mathbf{h}_{M,m_s} &\ddots &\mathbf{h}_{M,1}\\
& &\ddots &\vdots\\
& & &\mathbf{h}_{M,m_s}
\end{bmatrix}.
\end{align}

\cite{Bpl} specifies that termination is achieved by appending bits with value $0$ to the end of the message bits before encoding. These bits are called \emph{zero-tail bits}. The number of zero-tail bits $n_z$ depends on the number of message bits $n_m$ in the codeword and the asymptotic rate $R_\infty$. The number of \acp{cb} in the terminated codeword is $$L=\frac{n_m+n_z}{n-1}.$$ Since the zero-tail bits are known at the receiver, they are not transmitted. Only the parity bits generated from the zero-tail bits are transmitted. Hence, the actual rate of the terminated \ac{bpl} code is
\begin{align}\label{eq:rate_term}
R_L=\frac{n_m}{Ln-n_z}.
\end{align}
Encoding and termination of \ac{bpl} codes is explained in detail in Section \ref{sec:encode}.

\subsection{Decoding of \acl{ldpc} Codes}\label{sec:decoding_ldpc}
A channel decoder attempts to find the transmitted codeword $\x$ from the received word $\mathbf{y}$. The best decoder in terms of performance is a \ac{map} based decoder. Hence, its complexity grows exponentially with the information word length because it finds among all possible codewords the codeword that has the highest probability given the received word. The estimate of the transmitted codeword from a \ac{map} decoder is given by
\begin{align}
\widehat{\x}&=\argmax_{\x_i}\pdf_{X\mid Y}\left(\x_i,\mathbf{y}\right) \nonumber\\
&=\argmax_{\x_i}\frac{\pdf_{Y\mid X}\left(\mathbf{y},\x_i\right)\prob(\x_i)}{\prob(\mathbf{y})} \nonumber\\
&=\argmax_{\x_i}\pdf_{Y\mid X}\left(\mathbf{y},\x_i\right)\prob(\x_i)
\end{align}
where $\x_i$ is a codeword from the set of all codewords, $\mathbf{y}$ is the received word, $Y$ and $X$ are random variables representing received word and transmitted codeword respectively. For equiprobable codewords $\x_i$, a \ac{map} decoder is equivalent to a \ac{ml} decoder.

Due to the high complexity of \ac{map} decoders, \ac{ldpc} codes are usually decoded using iterative \acfp{mpa}.
\subsubsection{Belief Propagation}
The \ac{mpa} uses the \ac{bp} technique~\cite{Hagenauer1996} to compute the \emph{a-posteriori} probability of the bits in the transmitted codeword given the received word in an iterative fashion. The idea behind belief propagation is exchanging uncertainties between the bits which are connected as defined by the encoder or \ac{pcm}. Refer to Section~\ref{sec:enc_design} to see how different bits in the codeword are dependent on each other. The algorithm uses \acp{llr} instead of a-posteriori probabilities for numerical stability. The \ac{llr} values given by the channel for the received bits are
$$\mathcal{L}(y_i)=\log\frac{\prob\left(x_i=1\mid\mathbf{y}\right)}{\prob\left(x_i=0\mid\mathbf{y}\right)}$$
where $i=0,\dots,n-1$ is the index of bits in the codeword.

In a single iteration of the algorithm, the \acp{llr} of each bits in the codeword are updated through two intermediate message computations: \ac{v2c} message and \ac{c2v} message.
\begin{itemize}
  \item \ac{v2c} message: Each \ac{vn} passes its \acp{llr} on to its neighboring nodes (neighboring nodes are the \acp{cn} to which the \ac{vn} is connected in the Tanner graph). These \acp{llr} contain only extrinsic information from all other \acp{cn} in the previous iteration. The expression for the \ac{v2c} message is given by~\cite{Hagenauer1996}
  \begin{align}
    \mathcal{L}^{\mathrm{vc}}_{ij}=\mathcal{L}(y_i)+\sum_{j^\prime\in\mathcal{E}_v(i)\backslash j} \mathcal{L}^{\mathrm{cv}}_{j^\prime i}
  \end{align}
  where $\mathcal{L}^{\mathrm{vc}}_{ij}$ is the \ac{v2c} message from the $i$-th \ac{vn} to the $j$-th \ac{cn}, $\mathcal{L}^{\mathrm{cv}}_{ji}$ is the C2V message from the $j$-th \ac{cn} to the $i$-th \ac{vn} in the previous iteration and $\mathcal{E}_v(i)$ is the set of all \acp{cn} connected to the $i$-th \ac{vn}.
  \item \ac{c2v} message: Each \ac{cn} processes the received \ac{v2c} messages and computes extrinsic information for its neighboring \acp{vn}. These extrinsic informations contain \ac{v2c} messages from \acp{vn} other than the destination \ac{vn}. The expression for C2V messages is given by~\cite{Hagenauer1996}
  \begin{align}\label{eq:c2v}
  \mathcal{L}^{\mathrm{cv}}_{ji}=2\arctanh\left(\prod_{i^\prime\in\mathcal{E}_c(j)\backslash i}\tanh\left(\frac{\mathcal{L}^{\mathrm{vc}}_{i^\prime j}}{2}\right)\right)
  \end{align}
  where $\mathcal{E}_c(j)$ is the set of all \acp{vn} connected to the $j$-th \ac{cn}.
\end{itemize}
The process of sending a \ac{v2c} message, receiving a \ac{c2v} message from the same edge, and summing it up with the current \ac{llr} is termed an \emph{edge update}. The above steps indicate the \ac{bp} technique. It is also called \ac{spa}.

The high complexity C2V message computation can be approximated by a low-complexity computation called the \emph{\acf{msa}}. The \ac{msa} version of the expression in (\ref{eq:c2v}) is given by
\begin{align}\label{eq:msa}
\mathcal{L}^{\mathrm{cv}}_{ji}\approx\left(\prod_{i^\prime\in\mathcal{E}_c(j)\backslash i}\sign\left(\mathcal{L}^{\mathrm{vc}}_{i^\prime j}\right)\right)\cdot\min_{i^\prime}\lvert\mathcal{L}^{\mathrm{vc}}_{i^\prime j}\rvert.
\end{align}
In our implementation, we use an improved \ac{msa} as proposed in~\cite{Jones2003} which is a combination of (\ref{eq:c2v}) and (\ref{eq:msa}) with serial scheduling.

One can perform the edge updates in different sequences. The sequencing is termed as \emph{scheduling}. Two such scheduling methods are \emph{parallel scheduling} and \emph{serial scheduling}. In parallel scheduling which is also referred to as \emph{flooding}, the \acp{vn} send V2C messages to all \acp{cn} at once and then C2V messages are computed and passed to all \acp{vn}. On a parallel computing platform, the parallel scheduling is much faster than its serial counterpart but it lacks performance because the messages are not shared between different \acp{cn}. In serial scheduling~\cite{Zhang2002}, the \acp{vn} are updated in a \ac{cn}-by-\ac{cn} or row-by-row (in the \ac{pcm}) manner. Each \ac{cn} processes its V2C messages and sends the corresponding C2V messages to its neighboring \acp{vn}. This is called a \emph{row update} or a \emph{layer update}.

\subsubsection{Windowed Decoding}\label{sec:back_wd}
The conventional \ac{bp}-based decoder can be used to decode any \ac{ldpcbc} or terminated \ac{ldpccc} in which the belief propagation is performed throughout the whole Tanner graph at once. For \acp{ldpccc}, the convolutional structure imposes a constraint on the \acp{vn}: two \acp{vn} of the \ac{pcm} that are at least $(m_s+1)n$ columns apart cannot be involved in the same parity-check equation. This characteristic can be exploited to perform belief-propagation decoding only to a \emph{window} (part) of the received codeword at once~\cite{Iyengar2012}. This technique is called \emph{\ac{wd}} and is shown in Figure~\ref{fig:wd}.
\begin{figure}[htbp]
  \centering
  %\tikzsetnextfilename{tanner_graph}
  \includegraphics[width=0.5\textwidth, height=0.2\textwidth]{graphics/pcm_wd}
  \caption{\ac{pcm} illustrating the \acl{wd} technique for $R_\infty=2/3$ codes and $m_s=5$.}
  \label{fig:wd}
\end{figure}

In Figure~\ref{fig:wd} the window size is $W=6$. The vertically hatched \acp{vn} are the target nodes for the current window (thick line rectangle). The backhatched \acp{vn} are outside the window but still receive updates from the \acp{cn} inside the window. The hatched \acp{vn} are the target nodes for the next window (dashed rectangle). All \acp{vn} with patterns are updated during the current window. The size of the window $W$ should be $W\geq (m_s+1)$ because smaller window size will not include all the edges of the last \ac{cn} of the window. Thus the window includes $W$ \acp{cn} and $(W+m_s)n$ \acp{vn} connected to them. At each window instance, the first $n$ \acp{vn} are considered to be the \emph{target nodes}. However, all \acp{vn} connected to the $W$ \acp{cn} are updated, only the correctness of target nodes are considered as the criteria for moving on to the next window. The \ac{bp} decoding is performed within the current window for a maximum number of iterations or until the target nodes are decoded (whichever occurs earlier). Then the window shifts forward such that the next $n$ \acp{vn} become the target nodes. This process continues until all \acp{vn} in the received word are decoded.

The main benefit of window decoding is that the memory requirements are reduced because at any instance the \ac{bp} is performed on a smaller number of \acp{vn} rather than the whole graph. For non-packet-wise transmissions, the decoding latency is also reduced because the decoded bits from the previous windows can be sent to the higher layer for processing. However, these benefits come with a cost of reduced performance since the \ac{bp} is limited to fewer \acp{vn} and \acp{cn}.

\subsection{System Model}\label{sec:sys_mod}
In this thesis, we consider a digital baseband system model for simulations as shown in Figure \ref{fig:system}. The information bits are generated using a random-number generator with uniform distribution. The sequence of bits $\m$ of the generated random numbers is encoded using the \ac{bpl} encoder. The \emph{\ac{qam} Modulator} block receives the codewords $\x\in\mathbb{F}_2^{Ln\times 1}$ from the encoder and maps them to complex-valued symbols depending on the chosen modulation scheme. The output of the modulator is a vector of symbols given by $\mathbf{u}\in\mathbb{C}^{\frac{Ln}{o}\times 1}$ where $o$ is the order of modulation and lets assume that $Ln=go, g\in\mathbb{Z}_+$. In all our simulations we use \ac{qpsk} with gray mapping and no interleaving. Hence, $\mathbf{u}_i,i=0,\dots,\frac{Ln}{o}-1\in\{e^{j\frac{\pi}{4}},e^{j\frac{3\pi}{4}},e^{-j\frac{3\pi}{4}},e^{-j\frac{\pi}{4}}\}$.
\begin{figure}[htbp]
  \centering
  \includegraphics[width=\textwidth, height=\textwidth]{graphics/system_model}
  \caption{System model for simulations.}
  \label{fig:system}
\end{figure}

The channel model we consider is a simple \ac{awgn} channel with no fading or multi-path components. So the received symbols are given by $$\mathbf{v}=\mathbf{u}+\mathbf{n}$$where $\mathbf{n}=\mathcal{CN}(0,\sigma^2)\in\mathbb{C}^{\frac{Ln}{o}\times 1}$ is a complex random variable of Gaussian distribution with zero mean and variance $\sigma^2$. Note that only one symbol in $\mathbf{u}$ is transmitted per channel use. The received symbol vector is $\mathbf{v}\in\mathbb{C}^{\frac{Ln}{o}\times 1}$. The output of the \emph{QAM Demodulator} is a vector of \acp{llr} $\mathbf{y}\in\mathbb{R}^{Ln\times 1}$ corresponding to the bits in $\x$. The \acp{llr} are computed using the symbols in $\mathbf{v}$ and the modulation order. With \ac{qpsk} modulation, the \ac{llr} of first bit of each received symbol $y$ is computed as
\begin{align}
\mathcal{L}_{\text{first}}&=\log\frac{\prob\left(X_1=0\mid\re(y)\right)}{\prob\left(X_1=1\mid\re(y)\right)}
\end{align}
and of the second bit as
\begin{align}
\mathcal{L}_{\text{second}}&=\log\frac{\prob\left(X_2=0\mid\im(y)\right)}{\prob\left(X_2=1\mid\im(y)\right)}.
\end{align}
We assume that the bits in $\x$ are equiprobable hence,
\begin{align}
\mathcal{L}_{\text{first}}=\log\frac{\pdf_{\re(Y)\mid X}\left(\re(y), X_1=0\right)}{\pdf_{\re(Y)\mid X}\left(\re(y), X_1=1\right)},\\
\mathcal{L}_{\text{second}}=\log\frac{\pdf_{\im(Y)\mid X}\left(\im(y), X_2=0\right)}{\pdf_{\im(Y)\mid X}\left(\im(y), X_2=1\right)}.
\end{align}
On substituting the conditional \acp{pdf} of the Gaussian random variable, we get
\begin{align}
\mathcal{L}_{\text{first}}&=\frac{2}{\sigma^2}\re(y)\label{eq:final_llr1},\\
\mathcal{L}_{\text{second}}&=\frac{2}{\sigma^2}\im(y)\label{eq:final_llr2}.
\end{align}

From equations (\ref{eq:final_llr1}) and (\ref{eq:final_llr2}), we see that the \acp{llr} are directly proportional to the \ac{snr} $=\frac{2}{\sigma^2}$ of the received symbols. After the \acp{llr} are computed, they are decoded using the \ac{bp} and \ac{wd} techniques as mentioned in Section~\ref{sec:decoding_ldpc}. The estimated bits $\widehat{\m}$ are then compared with the output of the random bit generator $\m$ to calculate the \ac{ber} and \ac{bler}. \ac{ber} is the ratio between the number of error-bits and the total number of bits transmitted. \ac{bler} is the ratio between the number of error-blocks and the total number of blocks transmitted. A block is considered to be an error-block if at least one information bit is incorrect.
  \chapter{Encoding of the \acl{bpl} Codes}\label{ch:encode}
In this chapter, we discuss how the encoder for \ac{bpl} codes is designed and how the termination is handled. In some places of this section, we use examples of \acp{ldpccc} with small $L$ and $m_s$ as the \ac{bpl} codes are too large to be represented on paper.

\section{Encoder Design}\label{sec:enc_design}
Encoding algorithms are simpler than decoding algorithms because there exists no uncertainty in the output bits of an encoder. The encoder only generates the parity bits from the information bits and previously generated parity bits. From Section~\ref{sec:bpl_bg}, the parity-check equation of \ac{bpl} codes is given by
\begin{align}\label{eq:parity_check}
\sum_{i=0}^{m_s}\mathbf{h}_{M,i}\m(t-i)+\sum_{i=0}^{m_s}h_{B,i}b(t-i) \mod 2.
\end{align}
The encoder generates one parity bit per $n-1$ message bits and the parity bit expression is given by rearranging equation (\ref{eq:parity_check}) as
\begin{align}\label{eq:parity_bit}
b(t)=\sum_{i=0}^{m_s}\mathbf{h}_{M,i}\m(t-i)+\sum_{i=1}^{m_s}h_{B,i}b(t-i) \mod 2,
\end{align}
where $b(t)\in\mathbb{F}_2$ is the parity bit and $\m(t)\in\mathbb{F}_2^{k\times 1}$ is a vector of message bits at $t$-th time instance or \ac{cb}, $h_{B,i}\in\mathbb{F}_2$ is the coefficient of polynomial of the parity bit. $\mathbf{h}_{M,i}\in\mathbb{F}_2^{1\times k}$ is a vector of coefficients of polynomials of message bits and is given by
\begin{align}
\mathbf{h}_{M,i}&= \big[[D^i]A_{1,\tau},\dots,[D^i]A_{k,\tau}\big],
\end{align}
where $[D^i]A_{1,\tau}$ represents the coefficient of $D^i$ in polynomial $A_{1,\tau}(D)$.

The resulting codewords $\x\in\mathbb{F}_2^{Ln\times 1}$ have a systematic structure given by \begin{align}\x^T=(\x(0)^T,\x(1)^T,\dots,\x(L-1)^T).\end{align}

Similarly, the \ac{pcm} can be formed using the coefficients of the parity-check polynomials:
\begin{align}\label{eq:bpl_pcm}
\H=
\begin{bmatrix}
\mathbf{h}_{M,0}\\
\mathbf{h}_{M,1} &\mathbf{h}_{M,0}\\
\vdots &\mathbf{h}_{M,1} &\ddots\\
\mathbf{h}_{M,m_s} &\vdots &\ddots &\mathbf{h}_{M,0}\\
&\mathbf{h}_{M,m_s} &\ddots &\mathbf{h}_{M,1}\\
& &\ddots &\vdots\\
& & &\mathbf{h}_{M,m_s}
\end{bmatrix}.
\end{align}

The encoding of \ac{bpl} codes is done by generating parity bits as per equation (\ref{eq:parity_bit}). This is performed through following steps:
\begin{enumerate}
  \item The information bits are divided into several \acp{cb} and copied into the output buffer of the encoder. An example of this step for a $R_\infty=2/3$ code is shown in Figure~\ref{fig:encode_copy}.
  \begin{figure}[htbp]
    \centering
    \includegraphics[width=\textwidth, height=\textwidth]{graphics/encoder_copy}
    \caption{Copy input bits to output buffer. The red box represents the position of the parity bits.}
    \label{fig:encode_copy}
  \end{figure}
  \item The parity bits are generated by performing addition modulo 2 with all bits (both information and parity bits) in the output buffer according to the parity-check equation (\ref{eq:parity_bit}) of the code. An example of the parity-bit generation of a $R_\infty=2/3$ code is shown in Figure~\ref{fig:encoder_paritygen}.
   \begin{figure}[htbp]
    \centering
    \includegraphics[width=\textwidth, height=\textwidth]{graphics/encoder_parity}
    \caption{The encoder moves along the output buffer to generate parity bit for each \ac{cb}. Note: An example code.}
    \label{fig:encoder_paritygen}
  \end{figure}
  \item The termination sequence is appended to the output buffer of the encoder and is discussed in Section~\ref{sec:bpl_termi} in detail.
\end{enumerate}

\section{Termination Sequence}\label{sec:bpl_termi}
The objective of terminating any \ac{ldpccc} is to have a finite codeword length. Termination essentially means truncating the \ac{pcm} to have a finite number of rows and columns. Truncating the \ac{pcm} leads to a condition where there are not enough bits in the end of the codeword to fulfill the last $m_s$ parity-check equations of the \ac{pcm}. To satisfy the last $m_s$ parity-check equations, an appropriate termination sequence $\mathbf{a}$ must be appended to the codeword. This is illustrated in Figure~\ref{fig:term_req} with a \ac{pcm} that has a structure similar to (\ref{eq:bpl_pcm}).

Since \ac{bpl} codes are recursive codes, feeding zero-bits into the encoder after the information bits will not reset the states of the encoder i.e., the output of the encoder will never be all-zeros. Hence, to find a proper termination sequence that brings the encoder to all-zero state, one must solve a system of linear equations.

\begin{figure}[htbp]
  \centering
  \includegraphics[width=\textwidth, height=\textwidth]{graphics/term_req}
  \caption{\ac{pcm} illustrating that a truncated codeword do not satisfy all \acp{cn} in a $R_\infty=2/3$ code with $m_s=4$.}
  \label{fig:term_req}
\end{figure}

\begin{figure}[htbp]
  \centering
  \includegraphics[width=\textwidth, height=\textwidth]{graphics/term_solve}
  \caption{\ac{pcm} of an example \ac{ldpccc} code with $R_\infty=2/3$ and $m_s=4$ illustrating sub-matrices used to determine the termination sequence. Depicted above the \ac{pcm} is the codeword vector.}
  \label{fig:bpl_term}
\end{figure}

The procedure for calculating a termination sequence is illustrated in Figure~\ref{fig:bpl_term} where $\mathbf{e}\in\mathbb{F}_2^{m_sn\times 1}$ is the last part of the codeword vector. Vector $\mathbf{a}\in\mathbb{F}_2^{l_t\times 1}$ is the termination sequence that should be determined and appended to the end of $\mathbf{e}$. The sub-matrix $\mathbf{P}$ of the \ac{pcm} contains the last $m_s$ \acp{cn} of the actual \ac{pcm}. The sub-matrix $\mathbf{D}$ (containing the blue edges) is an extension to the actual \ac{pcm} due to the termination sequence $\mathbf{a}$. The codeword combined with a proper termination sequence must fulfill all the parity-check rows in matrix $[\mathbf{P},\mathbf{D}]$. Hence, provided a correct termination sequence $\mathbf{a}$, the following equations from~\cite{Chen2006} hold true.
\begin{align}
\begin{bmatrix}\mathbf{P} &\mathbf{D}\end{bmatrix}\odot
\begin{bmatrix}
\mathbf{e}\\
\mathbf{a}
\end{bmatrix}&=\mathbf{0}\\
\mathbf{P}\odot\mathbf{e}&=\mathbf{D}\odot\mathbf{a}
\end{align}
The termination sequence is given by
\begin{align}\label{eq:term_sol}
\mathbf{a}=\mathbf{D}^{-1}\odot\mathbf{P}\odot\mathbf{e}.
\end{align}
The maximum length $l_t$ of the termination sequence $\mathbf{a}$ should not exceed the one given in the \ac{bpl} standard. However, a shorter termination sequence with $l_t\geq n$ can be chosen such that $\mathbf{D}$ is a square matrix. Once the termination sequence $\mathbf{a}$ is determined from equation (\ref{eq:term_sol}), it is appended to the codeword and the remaining part of the termination sequence (if any) is filled with zeros. These zeros do not contribute to any parity checks. For an encoder that is implemented on hardware, these zeros are required to bring the encoder to an all-zero state. But we omit these zeros since we do software implementation. Hence, the resulting \ac{pcm} and codeword looks like the one shown in Figure~\ref{fig:bpl_term} with no zeros at the end of the codeword and without their corresponding edges in the \ac{pcm}.
%\begin{figure}[htbp]
%  \centering
%  \includegraphics[width=\textwidth, height=\textwidth]{graphics/term_result}
%  \caption{The last part of the \ac{pcm} and codeword after adding the actual termination sequence.}
%  \label{fig:term_res}
%\end{figure}

Unfortunately, we found that a proper termination sequence cannot be computed for \ac{bpl} codes. It is found through numerical evaluations that $\mathbf{D}$ is not full-rank and hence its inverse do not exist. A full rank $\mathbf{D}$ matrix is possible if $\mathbf{D}\in\mathbb{F}_2^{r\times c}$ and $r>c$. But with the help of numerical solvers, we found that the solution for such an overdetermined system do not exist either. Hence, a termination sequence to satisfy the last $m_s$ parity checks cannot be found for \ac{bpl} codes. So we choose to omit the last $m_s$ parity checks from the \ac{pcm} and do the termination by \emph{zero-tailing} as mentioned in the standard.

The steps for performing zero-tailing termination are as follows.
\begin{enumerate}
  \item The $n_z$ zero-tail bits are appended to the input buffer of the encoder after the $n_m$ information bits.
  \item The encoding as mentioned in Section~\ref{sec:enc_design} is performed to all the information bits and zero-tail bits.
\end{enumerate}
With the zero-tail-termination, the zero-tail bits are always known at the receiver. Hence, the zero-tail bits are not transmitted in an actual transmission system. Only the parity bits generated from the zero-tail bits are transmitted.

The zero-tail-termination does not satisfy the last $m_s$ parity checks of the \ac{pcm}. Hence, the \ac{pcm} of a zero-tail-terminated \ac{bpl} codes looks like the one shown in Figure~\ref{fig:pcm_zero}.
\begin{figure}[htbp]
  \centering
  \includegraphics[width=\textwidth, height=\textwidth]{graphics/term_zero}
  \caption{\ac{pcm} and codeword after adding the zero-tailing termination sequence.}
  \label{fig:pcm_zero}
\end{figure}

The \ac{pcm} of the zero-tail-terminated codes have the same \ac{cn} degrees in the middle and in the end. But in contrast, properly terminated codes have lower \ac{cn} degrees at the end. This reduces the performance of the codes as lower \ac{cn} degree means better reliability of the variable nodes connected to them. However, since the zero-tail bits are known at the receiver, they improve the reliability of the connected \acp{vn} during the \ac{bp} decoding process. The structure of a zero-tail-terminated codeword looks like the one shown in Figure~\ref{fig:term_cw}.

\begin{figure}[htbp]
  \centering
  \includegraphics[width=\textwidth, height=\textwidth]{graphics/term_cw}
  \caption{Illustration of a zero-tail-terminated codeword of $R_\infty=2/3$ code. Black box represents information bits, blue box for zero-tail bits and red box for parity bits.}
  \label{fig:term_cw}
\end{figure}

\begin{table}[htbp]
  \centering
  \begin{tabular}{|l|p{7cm}|p{7cm}|}
    \hline
    \textbf{S.No.} &\textbf{Proper Termination} &\textbf{Zero-tail Termination}\\
    \hline
    \hline
    1. &All parity-check equations of the \ac{pcm} are satisfied. &Last $m_s$ parity-checks are not satisfied.\\
    \hline
    2. &Encoder returns to all-zero state. &Encoder do not return to all-zero state.\\
    \hline
    3. &Predetermined matrices $\mathbf{F}=\mathbf{D}^{-1}\odot\mathbf{P}$ are stored for each $\tau$ and $R_\infty$ to compute $\mathbf{a}=\mathbf{F}\odot\mathbf{e}$. &Termination sequence is determined on the fly using the encoder.\\
    \hline
    4. &Lower \ac{cn} degree at the end of the codeword than the middle. This enhances decoding performance. &Same \ac{cn} degree at the end of the codeword as the middle.\\
    \hline
    5. &Termination sequence is unknown at the receiver. &Zero-tail bits are known at the receiver. This enhances decoding performance.\\
    \hline
  \end{tabular}
  \caption{Difference between proper termination and zero-tail termination.}
  \label{tab:diff_term}
\end{table}

We conclude this section by summarizing that proper termination for \ac{bpl} codes are not feasible and hence we perform zero-tail termination. The important differences between these two types of terminations are listed in Table~\ref{tab:diff_term}.
  \chapter{Decoding Improvements}\label{ch:dec_improve}
In this chapter, we discuss the techniques that are developed in this thesis that improve the decoding performance of window decoders. We start with the discussion of already existing techniques that improve the decoding performance and the motivation to develop new techniques. We then discuss the two techniques that are developed in this thesis.

\section{Existing Techniques and Motivation}
Improvement techniques for window decoding of \gls{ldpccc} are being widely discussed in literatures. An attractive technique that we found in the literature is the \emph{Zigzag decoder} proposed in~\cite{Shadi2015}. In a conventional window decoder (sliding-window decoder), the window moves from left to right of the \gls{pcm}~\cite{Iyengar2012}. So, the information about the probability of the bits flow from left to right of the codeword, i.e., in only one direction. This results in a loss in performance compared to a full-block decoder. The Zigzag decoder moves the window from right to left of the \gls{pcm} within a small part where the windows did not converge (i.e., the maximum number of iterations have reached and the \glspl{cn} inside the window did not fulfill the parity checks). It allows the information about probability of the bits to flow from right to left of the codeword, i.e., in the opposite direction. Hence, the different set of information have a positive effect on the reliability of the whole codeword.

The Zigzag decoder performs better than the sliding-window decoder but the decoding complexity in terms of number of iterations is higher. The implementation complexity is also high due to the nature of the decoder. Also, the Zigzag decoder uses flooding scheduling in its windows which is subpar compared to serial scheduling. Hence, the aforementioned factors motivates us to develop a decoder that moves in both directions in the codeword but has similar complexity and better performance than a sliding-window decoder.

Another improvement technique that we found to be interesting is about the condition for convergence of a window. Although its very natural to stop decoding when the target \gls{vn} are correct, one should choose an optimum criterion for deciding whether the target \gls{vn} are correct. A heuristic choice is to check whether the \emph{target-\glspl{cn}}, i.e., the \glspl{cn} connected to the target \glspl{vn} are fulfill their parity checks. A new criterion called \gls{psc} was proposed in~\cite{Kang2018} based on reliable \glspl{vn}. They distinguish the \glspl{vn} inside the window into \emph{complete-\glspl{vn}} and \emph{incomplete-\glspl{vn}}. Complete-\glspl{vn} are \glspl{vn} whose all the connected \glspl{cn} are inside the window. This includes the target \glspl{vn} also. So, the first $W-m_s$ \glspl{vn} inside the window are complete-\glspl{vn}. The remaining \glspl{vn} in the window are the incomplete-\glspl{vn}. This is illustrated in Figure~\ref{fig:comp_vn}. The authors say that the complete-\glspl{vn} are more reliable than the incomplete ones as they get more updates from \glspl{vn} from left of the window which are more reliable that are to the right. Hence, the parity checks of only the complete-\glspl{cn}, i.e., the \glspl{cn} connected to only the complete-\glspl{vn} are considered as convergence criterion.
\begin{figure}[htbp]
  \centering
  \tikzsetnextfilename{comp_vn}
  \includegraphics[width=0.5\textwidth, height=0.2\textwidth]{graphics/comp_vn}
  \caption{\gls{pcm} illustrating complete-\glspl{vn}, incomplete-\glspl{vn} and complete-\glspl{cn}.}
  \label{fig:comp_vn}
\end{figure}

Although the \gls{psc} rule proposed in~\cite{Kang2018} reduces the number of iterations with same performance, they are suitable only for small windows where $W<2m_s+1$. For larger windows, the number complete-\glspl{vn} are greater than the number of incomplete-\glspl{vn} which makes the number of complete-\glspl{cn} to be greater than the number of target-\glspl{cn}, i.e., the \glspl{cn} connected to target nodes. So, for larger windows the \gls{psc} rule actually increases the number of iterations. These factors motivates us develop a convergence criterion that depends on the window size.

\section{Base Decoder Configuration}
Before discussing the details of the techniques that are developed in this work, it is essential to define a decoder configuration on top of which the techniques are intended to apply. This allows better understanding of the techniques and evaluation of the simulation results. Let us call the decoder without our improvements as \gls{bd}. The \gls{bd} uses the sliding-window technique to decoder the \gls{bpl} codes. The window slides from the left end of the \gls{pcm} to the right end. Within each window, serial scheduling is performed by updating the \glspl{cn} from top to bottom for a maximum of $I_{\text{BD}}$ iterations.

In the last window instance, i.e., when the window touches the right most column of the \gls{pcm}, all the \glspl{vn} inside the window are considered as target \glspl{vn}. Hence, the convergence criterion is to check if all \glspl{cn} inside the window fulfill their parity checks. This idea was proposed in~\cite{Ali2018} as Early-Stopping rule. We adapt this technique in our \gls{bd} to minimize the total number to iterations. The convergence criterion for all the window instances in our \gls{bd} is that the target-\glspl{cn} should fulfill their parity checks.

We now know that the \gls{bpl} codes are terminated using the zero-tail termination. The zero-tail bits and their positions are always known at the receiver. Hence, the \glspl{llr} of the zero-tail bits are made to $+\infty$ which indicate that the value of these bits most certainly 0.

The configurations of the \gls{bd} is summarized below.
\begin{enumerate}
  \item \gls{bp} based window decoder and the window moves from left to right.
  \item Maximum number of iterations for each window is $I_{\text{BD}}$.
  \item Serial scheduling with an update order of \glspl{cn} from top to bottom.
  \item In the last window, all \glspl{vn} are considered as target \glspl{vn}.
  \item Target \glspl{cn} are considered for convergence criterion.
\end{enumerate}

\section{\acrlong{lrl} Decoder}
In this thesis, we propose a \emph{\gls{lrl} decoder}. The \gls{lrl} decoder moves the window from left to right and from right to left of the \gls{pcm} unlike the \gls{bd} that moves the window only from left to right. As already mentioned earlier, the bits in the left and right of the codeword are more reliable than the bits in the middle due to the lower \gls{cn} degrees in both the ends of the \gls{pcm}. This characteristic of the codeword can be seen in Figure~\ref{fig:indiv_ber} which plots the probability of error for all the bits in the codeword. We can see that the bits in the left and right of the codeword have lower probability of error than the bits in the middle. During the first decoding phase of the \gls{lrl} decoder, the window moves from the first window position to the last window position of the \gls{pcm}. During the second phase, i.e., after decoding the last window position, the window moves to the left till it reaches the first window position. Figure~\ref{fig:pcm_lrl} illustrates the \gls{lrl} decoder.
\begin{figure}[htbp]
  \centering
  \tikzsetnextfilename{indiv_ber}
  \includegraphics[width=0.9\linewidth]{plots/indiv_ber}
  \caption{Probability of error for each bit in the codeword of a code with $R_\infty=2/3$, $n_i=3500$. Termination bits are excluded.}
  \label{fig:indiv_ber}
\end{figure}

\begin{figure}[htbp]
  \centering
  \tikzsetnextfilename{pcm_lrl}
  \includegraphics[width=0.5\textwidth, height=0.2\textwidth]{graphics/pcm_lrl}
  \caption{\gls{pcm} illustrating \gls{lrl} decoder.}
  \label{fig:pcm_lrl}
\end{figure}
During the first decoding phase, the information from the \glspl{vn} in the left propagates to the \glspl{vn} to the right as the window moves forward. Then during the second decoding phase, the information from the \glspl{vn} in the right propagates to the \glspl{vn} to the left as the window moves backward. So, the highly reliable information from both ends of the codeword flows to the other parts of the codeword. The maximum number of iterations within each window is half of the maximum number of iterations in the \gls{bd}, i.e., $I_{\text{LRL}}=I_{\text{BD}}/2$. This is done to maintain the same decoding complexity as the \gls{bd} because each window decoding in the \gls{lrl} decoder is performed twice.

Within the \gls{lrl} decoder, we propose two different window configurations. They are illustrated in Figure~\ref{fig:win_config_lrl}. The red box indicates the window. The illustration in the left shows a window configuration where the left most \glspl{vn} (blue hatched) are the target \glspl{vn}. The order of \gls{cn} update is from top to bottom as one indicated by the brown arrow. Let us call this window configuration-\rom{1}. The illustration to the right shows that the right most \glspl{vn} in the window are the target \glspl{vn}. The order of \gls{cn} update is from bottom to top as indicated by the brown arrow. Let us call this window configuration-\rom{2}.

\begin{figure}[htbp]
  \centering
  \tikzsetnextfilename{win_config_lrl_new}
  \includegraphics[width=0.5\textwidth, height=0.2\textwidth]{graphics/win_config_lrl_new}
  \caption{Different window configurations used in \gls{lrl} decoder. Left image illustrates \gls{lrl} window configuration-\rom{1} and right image illustrates \gls{lrl} window configuration-\rom{2}.}
  \label{fig:win_config_lrl}
\end{figure}

We discuss the performance of the \gls{lrl} decoder with both window configurations. At first, we use \gls{lrl} decoder with window configuration-\rom{1} in both phases of the \gls{lrl} decoder. Secondly, we use the window configuration-\rom{1} during the first phase and window configuration-\rom{2} during the second phase. During the second phase, the \glspl{cn} are updated from bottom to top so that the information from right of the codeword flows to the left more consistently as the window moves backward. The simulation results of both the configurations are evaluated in Chapter~\ref{ch:simulation}.

\section{\acrlong{ipsc}}
The next technique we propose is the \gls{ipsc} rule. As mentioned earlier, a \gls{psc} was proposed in~\cite{Kang2018} that uses only complete-\gls{cn} as the convergence criterion. In the \gls{ipsc} rule, we introduce an additional convergence criterion for $W>2m_s+1$. When the window size $W>2m_s+1$ the target \glspl{cn} are considered for convergence criterion because the number of complete-\gls{cn} are greater than the number of target \glspl{cn}. Table~\ref{tab:ipsc} summarizes the \gls{ipsc} technique. The simulation results of the \gls{ipsc} are evaluated in Chapter~\ref{ch:simulation}.

\begin{table}[htbp]
\centering
\begin{tabular}{|l|l|}
  \hline
  \textbf{Window Size} &\textbf{Convergence Criterion}\\
  \hline
  \hline
  $W\leq2m_s+1$ &Complete-\glspl{vn}.\\
  \hline
  $W>2m_s+1$ &Target \glspl{vn}.\\
  \hline
\end{tabular}
\caption{Early-success criteria for \gls{ipsc}}
\label{tab:ipsc}
\end{table}
  \chapter{Simulation Results and Evaluation}\label{ch:simulation}
In this chapter, we analyze the simulation results of our proposed techniques. We start by discussing the simulation setup. Then we analyze the plots to evaluate the performance of our techniques.

\section{Experiment Setup}
For our simulations we used the baseband system model described in section~\ref{sec:sys_mod}. The Table~\ref{tab:sim_param} lists the different parameters of the simulation setup. All the simulations are performed with these parameters unless otherwise specified in the plot captions.
\begin{table}[htbp]
\centering
\begin{tabular}{|l|l|l|}
  \hline
  \textbf{S.No.} &\textbf{Parameter} &\textbf{Value}\\
  \hline
  \hline
  1. &No. of information bits $n_i$ &3500\\
  \hline
  2. &Asymptotic code rate $R_\infty$ &$2/3$\\
  \hline
  3. &No. of termination bits $m_t$ &380\\
  \hline
  5. &Modulation &\gls{qpsk}\\
  \hline
  6. &Window Size $W$ &300, 700\\
  \hline
  7. &No. of Iterations $I$ &10\\
  \hline
\end{tabular}
\caption{Experimental settings for simulations.}
\label{tab:sim_param}
\end{table}

\section{Evaluation of Zero-tail Termination}
In Chapter~\ref{ch:encode} we concluded that the termination for \gls{bpl} codes are performed through zero-tail termination. Here, we will evaluate the effect of zero-tail termination on the probability of error for each bit in the codeword. Figure~\ref{fig:eval_no_sat} shows the probability of error for each bit in the codeword which are calculated after being decoded by the \gls{bd}. The decoding is performed assuming that the zero-tail bits are not known at the receiver and so no effect of termination is applied on the codeword bits. Figure~\ref{fig:eval_sat} shows the probability of bits in the codeword when zero-tail bits are known at the receiver. The plots include the information bits and the parity bits but do not include the termination sequence.
\begin{figure}[htbp]
  \centering
  \includegraphics[width=0.9\linewidth]{plots/eval_no_sat}
  \caption{Probability of error for information and parity bits in the codeword. Zero-tail bits are not known at the receiver. Simulation parameters are $n_i=3500$, $R_\infty=2/3$, $W=700$ and $\zeta=2$ dB.}
  \label{fig:eval_no_sat}
\end{figure}

\begin{figure}[htbp]
  \centering
  \includegraphics[width=0.9\linewidth]{plots/eval_sat}
  \caption{Probability of error for information and parity bits in the codeword. Zero-tail bits are known at the receiver. Simulation parameters are $n_i=3500$, $R_\infty=2/3$, $W=700$ and $\zeta=2$ dB.}
  \label{fig:eval_sat}
\end{figure}

With all zero-tail terminated \gls{bpl} codes, the zero-tail bits are always knows at the receiver. 
From Figure~\ref{fig:eval_sat} we can see that when zero-tail bits are know at the receiver, the decoder reduces the $P(i)$ of the information bits and parity bits in the right of the codeword. While, Figure~\ref{fig:eval_no_sat} shows that the lack of knowledge of zero-tail bits at the receiver do not reduce $P(i)$ at the end of the codeword. So, the lack of proper termination does not reduce the $P(i)$ of the bits in the right of the codeword as seen in Figure~\ref{fig:eval_no_sat} but the zero-tail termination reduces the $P(i)$ which is seen in Figure~\ref{fig:eval_sat}. Hence, zero-tail termination is an acceptable alternative to proper termination.
 
\section{Evaluation of \acrfull{bd}}
Here, we analyze the performance of our \gls{bd}. First we analyze the performance vs complexity over different window sizes. Figure~\ref{fig:eval_bd_bler} shows the overall \gls{bler} $P_l$ over an \gls{snr} range of $2\leq\zeta\text{ (dB)}\leq 5$ over different window sizes. Figure~\ref{fig:eval_bd_aneu} shows the \gls{aneu} over the same range of \gls{snr}.
\begin{figure}[htbp]
  \centering
  \includegraphics[width=0.9\linewidth]{plots/eval_bd_bler}
  \caption{\gls{bler} vs \gls{snr} of the \acrfull{bd} with $n_i=3500$ and $R_\infty=2/3$.}
  \label{fig:eval_bd_bler}
\end{figure}
\begin{figure}[htbp]
   \centering
  \includegraphics[width=0.8\linewidth]{plots/eval_bd_aneu}
  \caption{\gls{bler} vs \gls{snr} of the \acrfull{bd} with $n_i=3500$ and $R_\infty=2/3$.}
  \label{fig:eval_bd_aneu}
\end{figure}

In Figure~\ref{fig:eval_bd_bler} we see the \gls{bler} improves with increasing window size $W$. When the window size is increased, more \glspl{vn} are included in the window enabling messaging passing over a large number of \glspl{vn}. This is seen the Figure~\ref{fig:eval_bd_aneu} as higher window size has higher \gls{aneu} in the low \gls{snr} region. In high \gls{snr} region, larger window size has an advantage in improving the \gls{vn} reliability thus converging the window quicker. This is why the complexity is less compared to smaller windows in high \glspl{snr}.

Figure~\ref{fig:eval_bler_rate} shows different \gls{bler} plots for all available rates $R_\infty$ for \gls{bpl} codes. It is well known that the performance of the code increases with decreasing $R_\infty$.
\begin{figure}[htbp]
  \centering
  \includegraphics[width=0.9\linewidth]{plots/eval_bler_rate}
  \caption{\gls{bler} vs \gls{snr} of the \acrfull{bd} with $n_i=3500$ and $W=500$.}
  \label{fig:eval_bler_rate}
\end{figure}

Figure~\ref{fig:eval_bd_iter_bler_300} and Figure~\ref{fig:eval_bd_iter_aneu_300} shows the \gls{bd} performance and complexity over different number of iterations $I$ per window. We can see that the performance of the decoder increases with increasing number of iterations because more iterations of message passing increases the reliability of the \glspl{vn}. And with more iterations comes more complexity.
\begin{figure}[htbp]
  \centering
  \includegraphics[width=0.9\linewidth]{plots/eval_bd_iter_bler_300}
  \caption{Comparison of \gls{bler} of the \acrfull{bd} for different $I$ with $n_i=3500$ and $W=300$.}
  \label{fig:eval_bd_iter_bler_300}
\end{figure}
\begin{figure}[htbp]
  \centering
  \includegraphics[width=0.8\linewidth]{plots/eval_bd_iter_aneu_300}
  \caption{Comparison of \gls{aneu} of the \acrfull{bd} for different $I$ with $n_i=3500$ and $W=300$.}
  \label{fig:eval_bd_iter_aneu_300}
\end{figure}

\section{Evaluation of \acrfull{lrl} Decoder}
Now we compare and evaluate the performance of \gls{lrl} decoder configuration-\rom{1} with the \gls{bd}. Figure~\ref{fig:eval_bd_lrl_bler} shows two plots of \gls{bler} for \gls{bd} and \gls{lrl} decoder with window size of $W=300$. Figure~\ref{fig:eval_bd_lrl_aneu} shows two plots of \gls{aneu} for \gls{bd} and \gls{lrl} decoder with window size of $W=300$.
\begin{figure}[htbp]
  \centering
  \includegraphics[width=0.9\linewidth]{plots/eval_bd_lrl_bler}
  \caption{Comparison of \gls{bler} between the Base Decoder and \gls{lrl} decoder with $W=300$.}
  \label{fig:eval_bd_lrl_bler}
\end{figure}
\begin{figure}[htbp]
  \centering
  \includegraphics[width=0.8\linewidth]{plots/eval_bd_lrl_aneu}
  \caption{Comparison of \gls{aneu} between the Base Decoder and \gls{lrl} decoder with $W=300$.}
  \label{fig:eval_bd_lrl_aneu}
\end{figure}

From both the figures, we see a significant decrease in \gls{bler} and \gls{aneu} for the \gls{lrl} decoder. The second phase of the \gls{lrl} decoder has improved the certainty of the \glspl{vn} through the highly reliable \glspl{vn} in the right end of the codeword. Hence, the proposed \gls{lrl} decoder is better than the \gls{bd} in terms of performance and complexity.

Figure~\ref{fig:eval_bd_lrl_bler_700} and Figure~\ref{fig:eval_bd_lrl_aneu_700} compares the performance and complexity between the \gls{bd} and \gls{lrl} decoder configuration-\rom{1} for window size $W=700$. We see that as the window size increases there is an improvement in \gls{bler} but not much decrease in the complexity. This is because with larger windows, each window converges quicker than smaller windows. So, it is more likely that not all iterations are used in the \gls{bd} and the second phase of the \gls{lrl} decoder has an advantage in decreasing \gls{bler}. 
\begin{figure}[htbp]
  \centering
  \includegraphics[width=0.9\linewidth]{plots/eval_bd_lrl_bler_700}
  \caption{Comparison of \gls{bler} between the Base Decoder and \gls{lrl} decoder with $W=700$.}
  \label{fig:eval_bd_lrl_bler_700}
\end{figure}
\begin{figure}[htbp]
  \centering
  \includegraphics[width=0.8\linewidth]{plots/eval_bd_lrl_aneu_700}
  \caption{Comparison of \gls{aneu} between the Base Decoder and \gls{lrl} decoder with $W=700$.}
  \label{fig:eval_bd_lrl_aneu_700}
\end{figure}

Figure~\ref{fig:eval_comp_lrl_bler_300} and Figure~\ref{fig:eval_comp_lrl_aneu_300} compares the performance and complexity between the \gls{lrl} decoders configuration-\rom{1} and configuration-\rom{2}. We see that the \gls{lrl} decoder with configuration-\rom{2} yields the same \gls{bler} performance with a slightly reduced complexity. The \gls{ber} plot from Figure~\ref{fig:eval_comp_lrl_ber_300} also indicates the same. The reduced complexity is mainly because of the bottom to top \gls{cn} update in the second phase of the \gls{lrl} decoder configuration-\rom{2}. Also since the target \gls{vn} are in the right end of the window, the window converges faster than the configuration-\rom{1} decoder.
\begin{figure}[htbp]
  \centering
  \includegraphics[width=0.9\linewidth]{plots/eval_comp_lrl_bler_300}
  \caption{Comparison of \gls{bler} between the \gls{lrl} decoders configuration-\rom{1} and configuration-\rom{2} with $W=300$.}
  \label{fig:eval_comp_lrl_bler_300}
\end{figure}
\begin{figure}[htbp]
  \centering
  \includegraphics[width=0.9\linewidth]{plots/eval_comp_lrl_ber_300}
  \caption{Comparison of \gls{ber} between the \gls{lrl} decoders configuration-\rom{1} and configuration-\rom{2} with $W=300$.}
  \label{fig:eval_comp_lrl_ber_300}
\end{figure}
\begin{figure}[htbp]
  \centering
  \includegraphics[width=0.8\linewidth]{plots/eval_comp_lrl_aneu_300}
  \caption{Comparison of \gls{aneu} between the \gls{lrl} decoders configuration-\rom{1} and configuration-\rom{2} with $W=300$.}
  \label{fig:eval_comp_lrl_aneu_300}
\end{figure}

\section{Evaluation of \acrfull{ipsc} Technique}
Here, we evaluate the performance of our \gls{ipsc} technique. Figure~\ref{fig:eval_ipsc_bler_300} and Figure~\ref{fig:eval_ipsc_aneu_300} compares the performance and complexity between the \gls{bd} with different early-success criteria with $W=300$. Similarly, Figure~\ref{fig:eval_ipsc_bler_600} and Figure~\ref{fig:eval_ipsc_aneu_600} compares for $W=600$.
\begin{figure}[htbp]
  \centering
  \includegraphics[width=0.9\linewidth]{plots/eval_ipsc_bler_300}
  \caption{Comparison of \gls{bler} between different early-success criteria with $W=300$.}
  \label{fig:eval_ipsc_bler_300}
\end{figure}
\begin{figure}[htbp]
  \centering
  \includegraphics[width=0.9\linewidth]{plots/eval_ipsc_aneu_300}
  \caption{Comparison of \gls{aneu} between different early-success criteria with $W=300$.}
  \label{fig:eval_ipsc_aneu_300}
\end{figure}
\begin{figure}[htbp]
  \centering
  \includegraphics[width=0.9\linewidth]{plots/eval_ipsc_bler_600}
  \caption{Comparison of \gls{bler} between different early-success criteria with $W=600$.}
  \label{fig:eval_ipsc_bler_600}
\end{figure}
\begin{figure}[htbp]
  \centering
  \includegraphics[width=0.9\linewidth]{plots/eval_ipsc_aneu_600}
  \caption{Comparison of \gls{aneu} between different early-success criteria with $W=600$.}
  \label{fig:eval_ipsc_aneu_600}
\end{figure}

From the Figures~\ref{fig:eval_ipsc_bler_300}, \ref{fig:eval_ipsc_aneu_300}, \ref{fig:eval_ipsc_bler_600} and \ref{fig:eval_ipsc_aneu_600} we see that for $W\geq2(m_s+1)$, checking only the target \glspl{cn} as early-success criteria decreases the decoding complexity.
  \chapter{Implementation Aspects}
\begin{itemize}
  \item Compact packed bits vs uint8 bits
  \item Encoder output buffer format
  \item In decoder, different approximations and its performance vs complexity
  \item setIdxVar on the fly vs once
\end{itemize}
In this chapter, we will discuss how the encoder and decoder are implemented. We start with describing the encoder's implementation and reasoning the chosen method. Then we discuss about the implementation of the decoder and the implications of different implementations on the performance.
\section{\aclp{vn}' Memory Format}
The \acp{vn} or the bits in the codeword can be stored in two different ways:
\begin{enumerate}
  \item Byte for a bit: Each byte of memory contains eight bits out of which the \ac{lsb} represents one \ac{vn}. An example of such a storage scheme is shown below.
  \begin{figure}[htbp]
    \centering
    \includegraphics[width=\textwidth, height=\textwidth]{graphics/bit_byte}
    \caption{Three bits of value 0, 1 and 1 are stored in single byte each. The arrow indicates the position of \ac{lsb} where the bit is stored in each byte. The bits marked with x are unused.}
  \end{figure}
  This form of storage allows us to directly access and use each bit as \texttt{uint8}.
  \begin{figure}[htbp]
    \centering
    \includegraphics[width=\textwidth, height=\textwidth]{graphics/bit_byte_sp}
    \caption{Direct access and use of bits.}
  \end{figure}
  \item Packed byte of bits: Each byte of memory contains eight bits representing eight \acp{vn}. An example of such storage is shown below.
  \begin{figure}[htbp]
    \centering
    \includegraphics[width=\textwidth, height=\textwidth]{graphics/packed_bits}
    \caption{Packed bits of bytes.}
  \end{figure}
  With this form of storage, additional functions are required to access and store each bit from and to its corresponding position because the minimum quantity of bits that can be accessed from the memory at once is a byte or \texttt{char} or \texttt{uint8}. This is shown in figure~\ref{fig:packed_spb}.
  \begin{figure}[htbp]
    \centering
    \includegraphics[width=\textwidth, height=\textwidth]{graphics/packed_bits_sp}
    \caption{Access and use of bits using helper functions.}
    \label{fig:packed_spb}
  \end{figure}
\end{enumerate}
In our implementations, we use packed byte of bits format. This format reduces the memory requirements for storing information and codeword bits by a factor of eight.

\section{\aclp{vn} Indexing}
The \ac{llr} values of \acp{vn} are stored in an array of memory where each element is a 32-bit floating point value. During the belief propagation, these values are accessed and used in the \ac{v2c} message computation. For each layer update, only specific \ac{llr} values from the array are used. This is done by computing and storing the absolute indices 
  \chapter{Conclusion and Outlook}\label{ch:conclu}
The accomplishments of this work is summarized in this chapter. Some advices for future works in window decoding of \gls{ldpccc} are also given.

As mobile cellular technologies in the future adapt \glspl{ldpccc} as error-correcting codes, there will be needs for efficient decoding algorithms. Smart phones now-a-days come with multiple cameras and sensors hence, increasing the power consumption. Also, popular applications such as 3D games and photo editors consume more power. These factors force the mobile device manufacturers to use power efficient modems. Our simulation results proved that the developed \acrfull{lrl} decoder has a better \acrfull{bler} performance at a much lower decoding complexity than a conventional sliding-window decoder. The \acrfull{ipsc} is also proved to decrease the decoding complexity. The decrease in decoding complexity means increase in battery-power saving.

We also showed why the \gls{bpl} codes cannot be terminated normally and has to be zero-tail terminated. We also proved that the zero-tail termination effectively reduces the \gls{cn} degree and hence the \acrfull{ber}. Although the zero-tail termination decreases the \gls{cn} degree at the end termination, it is not as low as the \gls{cn} degree in the start termination. Hence, care should be taken to ensure that the codeword can be terminated in a proper manner when a code is being designed.

Several adjustments can be made \gls{lrl} decoder. One such adjustment could be to move the window once from left to middle and right to middle of the \gls{pcm}. It could arguably give better \gls{ber} performance than just moving the window once from left end to right end of the \gls{pcm}. The \gls{lrl} decoder can also be combined with other decoding techniques. Another suggestion with regard to convergence criterion is to use soft value based parity check along with the \gls{ipsc} technique.
  \newpage
  \pagenumbering{Roman}
  \setcounter{page}{\value{romanpagenumbers}}
  \bibliographystyle{IEEEtran}
  \bibliography{IEEEabrv,references}
\end{document}
