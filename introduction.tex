\chapter{Introduction}
Error-correcting codes are a vital part in any reliable communication system. Since the first use of error-correcting codes, they have evolved to a great extent. Convolutional codes are used in \gls{gsm} (2G)~\cite{Gsm} which is the first standardized digital cellular network. After the introduction of Turbo codes, they are being used in \gls{umts} (3G)~\cite{3g} and \gls{lte} (4G)~\cite{Lte}. The low decoding complexities and high performance of \gls{ldpc} codes made them better candidates for the next generation mobile cellular technology: the \gls{nr} (5G)~\cite{Nra}. \gls{nr} uses a set of \glspl{ldpcbc} for user data transmissions and Polar codes for transferring all other control information.

Terminated \glspl{ldpccc} with large codeword lengths are proven to be better than \glspl{ldpcbc}. Alongside the code's performance, the \glspl{ldpccc} have an advantage in their decodability. Its convolutional structure allows for windowed decoding which requires fewer resources than a full-block decoder. However, window decoding always has a trade-off between decoding latency and decoding performance. \glspl{ldpccc} are used in applications like \gls{bpl} (IEEE 1901)~\cite{Bpl} and \gls{wimax} (IEEE 802.16)~\cite{Wimax} standards. But due to the better performance and increasing popularity of \gls{ldpccc}, their applications are more likely to be increased in the future. Figure~\ref{fig:mob_gen} shows the evolution of mobile cellular technologies and error-correcting codes used.

\begin{figure}[htbp]
  \centering
  \tikzsetnextfilename{mob_gen}
  \includegraphics[width=\textwidth, height=\textwidth]{graphics/mob_gen}
  \caption[Evolution of mobile telephone technologies and error-correcting codes.]{Evolution of mobile telephone technologies and type of error-correcting codes used.}
  \label{fig:mob_gen}
\end{figure}

%This thesis is focused on developing improvement techniques for window decoders. There are quite a number of window decoding algorithms that are proposed in recent years. One of them is the \emph{Zigzag decoder} that was proposed by Abu-Surra in~\cite{Shadi2015}. The decoder moves the window forward and backward in small parts of the \gls{pcm}. Another decoder proposed in~\cite{Kang2018} used a technique to reduce the incorrect information about the bits propagating from left to right of the codeword as the window moves. Three more improvement techniques for window decoder were proposed in~\cite{Ali2018} which are discussed in later chapters. The aforementioned decoding techniques either use flooding schedule or are evaluated with \gls{bec}. We are interested in serial scheduling and \gls{awgn} channel.

In this thesis, we propose techniques to efficiently decode \gls{ldpccc}. The \gls{lrl} decoder moves the decoding window forward and backward within the \gls{pcm}. The \gls{ipsc} is an improvement over \gls{psc} proposed in~\cite{Kang2018}. We also investigate the termination problem in the \gls{ldpccc} used in the standard IEEE 1901.

The organization of this thesis is as follows. Chapter~\ref{ch:back} contains the theory of channel coding and channel codes with an emphasis on \gls{ldpccc}. We also discuss the system model used for our simulations. In Chapter~\ref{ch:encode}, we discuss the termination problem in IEEE 1901's \gls{ldpccc}. We then discuss the proposed decoding techniques and their evaluation in Chapters~\ref{ch:dec_improve} and \ref{ch:simulation}, respectively. Chapter~\ref{ch:imple} contains details about implementation of the proposed techniques. Finally in Chapter~\ref{ch:conclu}, we conclude our findings of this and discuss about possible future works.