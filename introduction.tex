\chapter{Introduction}
\glspl{ldpccc} are a class of linear block codes which are widely used in data transmission systems. Terminated \glspl{ldpccc} are proven to be better than \glspl{ldpcbc} because of their convolutional nature and larger codeword lengths. Along side the code performance, the \gls{ldpccc} have an advantage to its decodability. Its convolutional structure allows for window decoding which requires fewer resources than a full-block decoder. However, window decoding always have a trade-off between decoding latency and decoding performance.

\glspl{ldpccc} are used in applications like \gls{bpl} (IEEE 1901) and \gls{wimax} (IEEE 802.16) standards to name a few. \gls{ldpcbc} are being used in more widely used applications like wireless \gls{lan}, \gls{dvb} and the upcoming \gls{nr}. But due to the better performance and increasing popularity of \gls{ldpccc}, their applications are more likely to be increased in the future.
\begin{figure}[htbp]
  \centering
  \tikzsetnextfilename{mob_gen}
  \includegraphics[width=\textwidth, height=\textwidth]{graphics/mob_gen}
  \caption{Evolution of mobile telephone technologies and type of error-correcting codes used.}
  \label{fig:mob_gen}
\end{figure}

This thesis is focused on developing improvement techniques for window decoders. There are quite a number of window decoding algorithms that are proposed in recent years. One of them is the \emph{Zigzag decoder} that was proposed by Abu-Surra in~\cite{Shadi2015}. The decoder moves the window forward and backward in small parts of the \gls{pcm}. Another decoder proposed in~\cite{Kang2018} used a technique to reduce the incorrect information about the bits propagating from left to right of the codeword as the window moves. Three more improvement techniques for window decoder were proposed in~\cite{Ali2018} which are discussed in later chapters. The aforementioned decoding techniques either use flooding schedule or are evaluated with \gls{bec}. We are interested in serial scheduling and \gls{awgn} channel.

In this thesis, we propose a \gls{lrl} decoder with two configurations and an \gls{ipsc} technique. The \gls{lrl} decoder which is motivated by the Zigzag decoder and moves the window forward and backward within the \gls{pcm}. The \gls{ipsc} is an improvement over \gls{psc} proposed in~\cite{Kang2018}. We also investigate the termination problem in the \gls{ldpccc} used in the standard IEEE 1901. We evaluate the proposed techniques with an \gls{awgn} channel system model.

The organization of this thesis is as follows. Chapter~\ref{ch:back} contains the theory of channel coding and channel codes with an emphasis on \gls{ldpccc}. We also discuss the system model used for our simulations. In Chapter~\ref{ch:encode}, we discuss the termination problem in IEEE 1901's \gls{ldpccc}. We then discuss the proposed decoding techniques and their evaluation in Chapters~\ref{ch:dec_improve} and \ref{ch:simulation}, respectively. Finally in Chapter~\ref{ch:conclu}, we conclude our findings of this and discuss about possible future works.