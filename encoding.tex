\chapter{Encoding of the \texorpdfstring{\acrlong{bpl}}{BPL} Codes}\label{ch:encode}
In this chapter, we discuss how the encoder for \gls{bpl} codes is designed and how termination is handled. In some places of this chapter, we use examples of \glspl{ldpccc} with small coupling length $L$ and syndrome-former memory $m_s$ as the \gls{bpl} codes are too large to be represented legibly on paper.

\section{Encoder Design}\label{sec:enc_design}
Encoding algorithms are not as complex as decoding algorithm because decoders need to correct the incorrect bits in the codeword. The encoder for any Convolutional code or \gls{ldpccc} only needs to generate the parity bits from the information bits and previously generated parity bits. From equation (\ref{eq:bpl_poly}), the parity-check term of \gls{bpl} codes is given by
\begin{align}\label{eq:parity_check}
\sum_{i=0}^{m_s}\mathbf{h}_{M,i}\m(t-i)+\sum_{i=0}^{m_s}h_{B,i}b(t-i) \mod 2
\end{align}
where $\m(t)\in\mathbb{F}_2^{k\times 1}$ is a vector of message bits at $t$-th time instance or \gls{cb}, $h_{B,i}\in\mathbb{F}_2$ is the coefficient of polynomial of the parity bit. $\mathbf{h}_{M,i}\in\mathbb{F}_2^{1\times k}$ is a vector of coefficients of polynomials of message bits and is given by (\ref{eq:poly_coef_vec}).
The encoder generates one parity bit per $n-1$ message bits and the expression for generating the parity bit is given by rearranging (\ref{eq:parity_check}) as
\begin{align}\label{eq:parity_bit}
b(t)=\sum_{i=0}^{m_s}\mathbf{h}_{M,i}\m(t-i)+\sum_{i=1}^{m_s}h_{B,i}b(t-i) \mod 2,
\end{align}
where $b(t)\in\mathbb{F}_2$ is the parity bit.
\begin{align}\label{eq:poly_coef_vec}
\mathbf{h}_{M,i}&=
\begin{bmatrix}
[D^i]A_{1,\tau} &\dots &[D^i]A_{k,\tau}
\end{bmatrix},
\end{align}
where $[D^i]A_{1,\tau}$ represents the coefficient of $D^i$ in polynomial $A_{1,\tau}(D)$.

The resulting codewords $\x\in\mathbb{F}_2^{Ln\times 1}$ have a systematic structure given by \begin{align}\x^T=\begin{bmatrix}\x(0)^T &\x(1)^T &\dots &\x(L-1)^T\end{bmatrix}.\end{align}

Similarly, the \gls{pcm} can be formed using $\mathbf{h}_{M,i}$ and $h_{B,i}$ for all $i=0,\dots,m_s$:
\begin{align}\label{eq:bpl_pcm}
\H=
\begin{bmatrix}
[\mathbf{h}_{M,0},h_{B,0}]\\
[\mathbf{h}_{M,1},h_{B,1}] &[\mathbf{h}_{M,0},h_{B,0}]\\
\vdots &[\mathbf{h}_{M,1},h_{B,1}] &\ddots\\
[\mathbf{h}_{M,m_s},h_{B,m_s}] &\vdots &\ddots &[\mathbf{h}_{M,0},h_{B,0}]\\
&[\mathbf{h}_{M,m_s},h_{B,m_s}] &\ddots &[\mathbf{h}_{M,1},h_{B,1}]\\
& &\ddots &\vdots\\
& & &[\mathbf{h}_{M,m_s},h_{B,m_s}]
\end{bmatrix}.
\end{align}

Encoding of \gls{bpl} codes is done by generating parity bits as per (\ref{eq:parity_bit}). This is performed through the following steps:
\begin{enumerate}
  \item The information bits are divided into several \glspl{cb} and copied into the output buffer of the encoder. An example of this step for a code with $R_\infty=2/3$ is shown in Figure~\ref{fig:encode_copy}.
  \begin{figure}[htbp]
    \centering
     \tikzsetnextfilename{encoder_copy}
    \includegraphics[width=\textwidth, height=\textwidth]{graphics/encoder_copy}
    \caption[Illustration of input and output buffers of encoder.]{Illustration of input and output buffers and its contents. The red boxes represents the position of the parity bits.}
    \label{fig:encode_copy}
  \end{figure}
  \item The parity bits are generated by performing addition modulo 2 with the bits (both information and parity bits) in the output buffer according (\ref{eq:parity_bit}). An example of the parity-bit generation of a code with  $R_\infty=2/3$ is shown in Figure~\ref{fig:encoder_paritygen}.
   \begin{figure}[htbp]
    \centering
    \tikzsetnextfilename{encoder_parity}
    \includegraphics[width=\textwidth, height=\textwidth]{graphics/encoder_parity}
    \caption[Illustration of a parity bit being generation.]{Illustration of a parity bit being generated using information and other parity bits in the output buffer. The arrow indicates the direction of movement of the parity-bit generator. Note: An example code.}
    \label{fig:encoder_paritygen}
  \end{figure}
  \item The termination sequence is appended to the output buffer of the encoder. The procedure to generate the termination sequence is discussed in Section~\ref{sec:bpl_termi}.
\end{enumerate}

\section{Termination Sequence}\label{sec:bpl_termi}
The objective of terminating any \gls{ldpccc} is to introduce some low degree \glspl{cn} in the \gls{pcm}. Termination essentially means truncating the \gls{pcm} to have a finite number of rows and columns. Truncating the \gls{pcm} leads to a condition where the last $m_s$ bits of the codeword do not satisfy the last $m_s$ parity-check equations of the \gls{pcm}. To satisfy the last $m_s$ parity-check equations, either the last $m_s$ or lesser bits need to modified or an appropriate termination sequence $\mathbf{a}$ must be appended to the codeword. Since the last $m_s$ bits of the codeword should not be modified as the contain information, the later approach is used. The appending of termination sequence is illustrated in Figure~\ref{fig:term_req} with a \gls{pcm} that has a structure similar to (\ref{eq:bpl_pcm}).

\subsection{Proper Termination}
Since the generator polynomials of the \gls{bpl} codes are recursive, continuously feeding in zero-bits into the encoder after the information bits will not reset the states of the encoder. i.e., the output of the encoder will never be all zeros. Hence, one must solve a system of linear equations to find a proper termination sequence that brings the encoder to all-zero state.

\begin{figure}[htbp]
  \centering
  \tikzsetnextfilename{term_req}
  \includegraphics[width=\textwidth, height=\textwidth]{graphics/term_req}
  \caption[Illustration of truncated codeword and its \acrshort{pcm}.]{\gls{pcm} illustrating that a truncated codeword do not satisfy all \glspl{cn} in a $R_\infty=2/3$ code with $m_s=4$.}
  \label{fig:term_req}
\end{figure}

\begin{figure}[htbp]
  \centering
  \tikzsetnextfilename{term_solve}
  \includegraphics[width=\textwidth, height=\textwidth]{graphics/term_solve}
  \caption[Illustration of proper termination procedure.]{\gls{pcm} of an example \gls{ldpccc} code with $R_\infty=2/3$ and $m_s=4$ illustrating sub-matrices used to determine the termination sequence. Depicted above the \gls{pcm} is the codeword vector.}
  \label{fig:bpl_term}
\end{figure}

The procedure for determining a termination sequence is reproduced here from~\cite{Chen2006}. Figure~\ref{fig:bpl_term} illustrates the different matrices and vectors that are used in determining the termination sequence. $\mathbf{e}\in\mathbb{F}_2^{m_sn\times 1}$ is the last part of the codeword vector, vector $\mathbf{a}\in\mathbb{F}_2^{l_t\times 1}$ is the termination sequence that should be determined and appended to the end of $\mathbf{e}$. The sub-matrix $\mathbf{P}$ of the \gls{pcm} contains the last $m_s$ \glspl{cn} of the actual \gls{pcm}. The sub-matrix $\mathbf{D}$ (containing the blue edges) is an extension to the actual \gls{pcm} due to the termination sequence $\mathbf{a}$. The codeword combined with a proper termination sequence must fulfill all the parity-check rows in matrix $[\mathbf{P},\mathbf{D}]$. Hence, provided a correct termination sequence $\mathbf{a}$, the following equations from~\cite{Chen2006} hold true.
\begin{align}
\begin{bmatrix}\mathbf{P} &\mathbf{D}\end{bmatrix}\odot
\begin{bmatrix}
\mathbf{e}\\
\mathbf{a}
\end{bmatrix}&=\mathbf{0}\\
\mathbf{P}\odot\mathbf{e}+\mathbf{D}\odot\mathbf{a}&=0
\end{align}
One possible solution under the constraint that $\mathbf{D}$ is square and invertible is given by
\begin{align}\label{eq:term_sol}
\mathbf{a}=\mathbf{D}^{-1}\odot\mathbf{P}\odot\mathbf{e}.
\end{align}
The maximum length $l_t$ of the termination sequence $\mathbf{a}$ should not exceed the one given in the \gls{bpl} standard. However, a shorter termination sequence with $l_t\geq n$ can be chosen such that $\mathbf{D}$ is a square matrix. Once the termination sequence $\mathbf{a}$ is determined from (\ref{eq:term_sol}), it is appended to the codeword and the remaining part of the termination sequence (if any) is filled with zeros. These zeros do not contribute to any parity checks. For an encoder that is implemented with feedback shift registers, these zeros are required to bring the encoder to an all-zero state. But, we omit these zeros since we implement the encoder in software. Hence, the resulting \gls{pcm} and codeword looks like the one shown in Figure~\ref{fig:bpl_term} with no zeros at the end of the codeword and without their corresponding edges in the \gls{pcm}.
%\begin{figure}[htbp]
%  \centering
%  \includegraphics[width=\textwidth, height=\textwidth]{graphics/term_result}
%  \caption{The last part of the \gls{pcm} and codeword after adding the actual termination sequence.}
%  \label{fig:term_res}
%\end{figure}

\subsection{Zero-Tail Termination}
Unfortunately, we found that a proper termination sequence cannot be computed for \gls{bpl} codes. It is found through numerical evaluations that $\mathbf{D}$ is not full-rank and hence its inverse do not exist. A full-rank $\mathbf{D}$ matrix is possible if $\mathbf{D}\in\mathbb{F}_2^{r\times c}$ and $r>c$. But, with the help of numerical solvers, we found that the solution for such an overdetermined system does not exist either. Hence, a termination sequence to satisfy the last $m_s$ parity checks cannot be found for \gls{bpl} codes. So, we choose to omit the last $m_s$ parity checks from the \gls{pcm} and do the termination by \emph{zero-tailing} as mentioned in the standard.

The steps for performing zero-tailing termination are as follows.
\begin{enumerate}
  \item The $n_z$ zero-tail bits are appended to the input buffer of the encoder after the $n_m$ information bits.
  \item The encoding as mentioned in Section~\ref{sec:enc_design} is performed using all the information bits and zero-tail bits.
\end{enumerate}
With the zero-tail termination, the zero-tail bits are always known at the receiver. Hence, the zero-tail bits are not transmitted in an actual system. Only the parity bits generated from the zero-tail bits are transmitted and used during decoding.

The zero-tail termination does not satisfy the last $m_s$ parity checks of the \gls{pcm}. Hence, the \gls{pcm} of a zero-tail-terminated \gls{bpl} codes looks like the one shown in Figure~\ref{fig:pcm_zero}.
\begin{figure}[htbp]
  \centering
  \tikzsetnextfilename{term_zero}
  \includegraphics[width=\textwidth, height=\textwidth]{graphics/term_zero}
  \caption[Illustration of zero-tail termination.]{\gls{pcm} and codeword after adding the zero-tailing termination sequence.}
  \label{fig:pcm_zero}
\end{figure}

The \gls{pcm} of the zero-tail-terminated codes has the same \gls{cn} degrees in the middle and in the end. By contrast, properly terminated codes have lower \gls{cn} degrees at the end. The lack of low \gls{cn} degrees reduce the performance of the code as lower \gls{cn} degree means better reliability of the variable nodes connected to them. However, the zero-tail bits improve the reliability of the connected \glspl{vn} during the \gls{bp} decoding process since the zero-tail bits are known at the receiver. The structure of a zero-tail-terminated codeword looks like the one shown in Figure~\ref{fig:term_cw}.

\begin{figure}[htbp]
  \centering
  \tikzsetnextfilename{term_cw}
  \includegraphics[width=\textwidth, height=\textwidth]{graphics/term_cw}
  \caption[Illustration of a zero-tail-terminated codeword.]{Illustration of a zero-tail-terminated codeword of a code with $R_\infty=2/3$. Black boxes represent information bits, blue boxes represent zero-tail bits and red boxes represent parity bits.}
  \label{fig:term_cw}
\end{figure}

\begin{table}[htbp]
  \centering
  \begin{tabular}{|p{5cm}|p{5cm}|p{5cm}|}
    \hline
    \textbf{Aspect of Comparison} &\textbf{Proper Termination} &\textbf{Zero-Tail Termination}\\
    \hline
    \hline
    Omission of \glspl{cn} &No \glspl{cn} are omitted from \gls{pcm}. &Last $m_s$ \glspl{cn} are omitted from the \gls{pcm}.\\
    \hline
    Ease for implementation using shift registers &Encoder returns to all-zero state. &Encoder does not return to all-zero state.\\
    \hline
    Memory and complexity &Predetermined matrices $\mathbf{F}=\mathbf{D}^{-1}\odot\mathbf{P}$ are stored for each $\tau$ and $R_\infty$ to compute $\mathbf{a}=\mathbf{F}\odot\mathbf{e}$. Requires matrix multiplication to be performed. &No storage or matrix multiplication is required. Termination sequence is determined by just using the encoder.\\
    \hline
    \gls{cn} degree &Lower \gls{cn} degree at the end of the \gls{pcm} than the middle. This enhances decoding performance. &Same \gls{cn} degree at the end of the \gls{pcm} as the middle.\\
    \hline
    Knowledge of termination sequence at receiver. &Termination sequence is unknown at the receiver. &Zero-tail bits are known at the receiver. This enhances decoding performance.\\
    \hline
  \end{tabular}
  \caption{Difference between proper termination and zero-tail termination.}
  \label{tab:diff_term}
\end{table}

We conclude this section by summarizing that proper termination for \gls{bpl} codes are not feasible because of the nature of the \glspl{pcm} and hence, we perform zero-tail termination. The important differences between these two types of terminations are listed in Table~\ref{tab:diff_term} and we see that both terminations have advantages and disadvantages over the other.