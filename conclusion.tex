\chapter{Conclusion and Outlook}\label{ch:conclu}
The accomplishments of this work is summarized in this chapter. Some advices for future works in window decoding of \gls{ldpccc} are also given.

As mobile cellular technologies in the future adapt \glspl{ldpccc} as error-correcting codes, there will be needs for efficient decoding algorithms. Smart phones now-a-days come with multiple cameras and sensors hence, increasing the power consumption. Also, popular applications such as 3D games and photo editors consume more power. These factors force the mobile device manufacturers to use power efficient modems. Our simulation results proved that the developed \acrfull{lrl} decoder has a better \acrfull{bler} performance at a much lower decoding complexity than a conventional sliding-window decoder. The \acrfull{ipsc} is also proved to decrease the decoding complexity. The decrease in decoding complexity means increase in battery-power saving.

We also showed why the \gls{bpl} codes cannot be terminated normally and has to be zero-tail terminated. We also proved that the zero-tail termination effectively reduces the \gls{cn} degree and hence the \acrfull{ber}. Although the zero-tail termination decreases the \gls{cn} degree at the end termination, it is not as low as the \gls{cn} degree in the start termination. Hence, care should be taken to ensure that the codeword can be terminated in a proper manner when a code is being designed.

Several adjustments can be made \gls{lrl} decoder. One such adjustment could be to move the window once from left to middle and right to middle of the \gls{pcm}. It could arguably give better \gls{ber} performance than just moving the window once from left end to right end of the \gls{pcm}. The \gls{lrl} decoder can also be combined with other decoding techniques. Another suggestion with regard to convergence criterion is to use soft value based parity check along with the \gls{ipsc} technique.