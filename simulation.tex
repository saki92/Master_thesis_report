\chapter{Simulation Results and Evaluation}\label{ch:simulation}
In this chapter, we analyze the simulation results of our proposed techniques. We start by discussing the simulation setup. Then, we analyze the plots to evaluate the performance of our techniques.

\section{Experiment Setup}
For our simulations, we used the baseband system model described in Section~\ref{sec:sys_mod}. The Table~\ref{tab:sim_param} lists the different parameters of the simulation setup. All the simulations are performed with these parameters unless otherwise specified in the plot captions.
\begin{table}[htbp]
\centering
\begin{tabular}{|l|l|}
  \hline
  \textbf{Parameter} &\textbf{Value}\\
  \hline
  \hline
  No. of information bits $n_i$ &3500\\
  \hline
  Asymptotic code rate $R_\infty$ &$2/3$\\
  \hline
  No. of termination bits $m_t$ &380\\
  \hline
  Modulation &\gls{qpsk}\\
  \hline
  Window Size $W$ &300\\
  \hline
  No. of Iterations $I$ &5\\
  \hline
  No. of Runs &4000\\
  \hline
\end{tabular}
\caption{Experimental settings for simulations.}
\label{tab:sim_param}
\end{table}

In most of our results we plot \gls{bler} instead of \gls{ber} because in modern packet based transmission system, the whole block is discarded in physical layer if they are incorrect. So, it is better to evaluate the performance by \gls{bler} than \gls{ber}. All the simulations results are averaged over a number of runs specified in the Table~\ref{tab:sim_param}.
\section{Effect of Zero-tail Termination}
In Chapter~\ref{ch:encode} we concluded that the termination for \gls{bpl} codes are performed through zero-tail termination. Here, we evaluate the effect of zero-tail termination on the probability of error of each bit in the codeword. Figure~\ref{fig:eval_no_sat} shows the probability of error of each bit in the codeword when the zero-tail bits are not known at the receiver, i.e., at the decoder the \glspl{llr} of zero-tail bits are saturated to $+\infty$. Figure~\ref{fig:eval_sat} shows the probability of bits in the codeword when zero-tail bits are known at the receiver. The probabilities are calculated using (\ref{eq:indiv_prob}) and are averaged over 4000 runs.  The plots include the information bits and the parity bits but do not include the termination sequence.
\begin{figure}[htbp]
  \centering
  \tikzsetnextfilename{eval_no_sat}
  \includegraphics[width=0.9\linewidth]{plots/eval_no_sat}
  \caption{Probability of error for information and parity bits in the codeword for $\zeta=2$ dB. Zero-tail bits are not known at the receiver.}
  \label{fig:eval_no_sat}
\end{figure}

\begin{figure}[htbp]
  \centering
  \tikzsetnextfilename{eval_sat}
  \includegraphics[width=0.9\linewidth]{plots/eval_sat}
  \caption{Probability of error for information and parity bits in the codeword for $\zeta=2$ dB. Zero-tail bits are known at the receiver.}
  \label{fig:eval_sat}
\end{figure}
 
From Figure~\ref{fig:eval_sat} we can see that when zero-tail bits are know at the receiver, the decoder reduces the $P_b(i)$ of the information and parity bits in the right of the codeword. On the other hand, Figure~\ref{fig:eval_no_sat} shows that the lack of knowledge of zero-tail bits at the receiver does not reduce $P_b(i)$ at the end of the codeword. So, the knowledge of zero-tail bits at the decoder effectively reduces the \gls{cn} degrees at the right of the \gls{pcm} as in a properly terminated codeword. Hence, zero-tail termination is an acceptable alternative to proper termination.
 
\section{Evaluation of \acrlong{bd}}
Here, we analyze the performance of our \gls{bd}. At first, we analyze the performance and complexity over different window sizes. Figure~\ref{fig:eval_bd_bler} shows the overall \gls{bler} $P_l$ over an \gls{snr} range of $2\leq\zeta\text{ (dB)}\leq 5$ for different window sizes. Figure~\ref{fig:eval_bd_aneu} shows the \gls{aneu} over the same range of \gls{snr}.
\begin{figure}[htbp]
  \centering
  \tikzsetnextfilename{eval_bd_bler}
  \includegraphics[width=0.9\linewidth]{plots/eval_bd_bler}
  \caption{\gls{bler} vs \gls{snr} of the \acrfull{bd}.}
  \label{fig:eval_bd_bler}
\end{figure}
\begin{figure}[htbp]
   \centering
   \tikzsetnextfilename{eval_bd_aneu}
  \includegraphics[width=0.8\linewidth]{plots/eval_bd_aneu}
  \caption{\gls{bler} vs \gls{snr} of the \acrfull{bd}.}
  \label{fig:eval_bd_aneu}
\end{figure}

In Figure~\ref{fig:eval_bd_bler}, we see that the \gls{bler} improves with increasing window size $W$. When the window size is increased, more \glspl{vn} are included in the window enabling information to flow between a large number of \glspl{vn}. Figure~\ref{fig:eval_bd_aneu} shows that the \gls{aneu} is high for larger window in the low \gls{snr} region as the windows do not converge. In high \gls{snr} region, larger windows converge quicker and hence, they have lower complexity than smaller windows.

Secondly, we analyze the performance over different code rates. Figure~\ref{fig:eval_bler_rate} shows different \gls{bler} plots for all available rates $R_\infty$ for \gls{bpl} codes. It is well known that the performance of the code increases with decreasing $R_\infty$.
\begin{figure}[htbp]
  \centering
  \tikzsetnextfilename{eval_bler_rate}
  \includegraphics[width=0.9\linewidth]{plots/eval_bler_rate}
  \caption{\gls{bler} vs \gls{snr} of the \acrfull{bd} with $W=500$.}
  \label{fig:eval_bler_rate}
\end{figure}

Finally, we compare the performance over $I=5,10$. Figure~\ref{fig:eval_bd_iter_bler_300} and Figure~\ref{fig:eval_bd_iter_aneu_300} shows the \gls{bd}'s performance and complexity over different number of iterations $I$ per window. We can see that the complexity is less for fewer $I=5$ but the performance of the decoder remains the same which indicates that the windows converge quicker, not utilizing all 5 iterations.
\begin{figure}[htbp]
  \centering
  \tikzsetnextfilename{eval_bd_iter_bler_300}
  \includegraphics[width=0.9\linewidth]{plots/eval_bd_iter_bler_300}
  \caption{Comparison of \gls{bler} of the \acrfull{bd} for different $I$.}
  \label{fig:eval_bd_iter_bler_300}
\end{figure}
\begin{figure}[htbp]
  \centering
  \tikzsetnextfilename{eval_bd_iter_aneu_300}
  \includegraphics[width=0.8\linewidth]{plots/eval_bd_iter_aneu_300}
  \caption{Comparison of \gls{aneu} of the \acrfull{bd} for different $I$.}
  \label{fig:eval_bd_iter_aneu_300}
\end{figure}

\section{Evaluation of \acrlong{lrl} Decoder}
Now, we compare the performance of \gls{lrl} decoder configuration-\rom{1} with the \gls{bd}. Figure~\ref{fig:eval_bd_lrl_bler} shows two plots of \gls{bler} each for \gls{bd} and \gls{lrl} decoder with window size $W=300$. Similarly, Figure~\ref{fig:eval_bd_lrl_aneu} shows two plots of \gls{aneu} each for \gls{bd} and \gls{lrl} decoder with window size of $W=300$.
\begin{figure}[htbp]
  \centering
  \tikzsetnextfilename{eval_bd_lrl_bler}
  \includegraphics[width=0.9\linewidth]{plots/eval_bd_lrl_bler}
  \caption{Comparison of \gls{bler} between the Base Decoder and \gls{lrl} decoder.}
  \label{fig:eval_bd_lrl_bler}
\end{figure}
\begin{figure}[htbp]
  \centering
  \tikzsetnextfilename{eval_bd_lrl_aneu}
  \includegraphics[width=0.8\linewidth]{plots/eval_bd_lrl_aneu}
  \caption{Comparison of \gls{aneu} between the Base Decoder and \gls{lrl} decoder.}
  \label{fig:eval_bd_lrl_aneu}
\end{figure}

From both the figures, we see a significant decrease in \gls{bler} and \gls{aneu} of the \gls{lrl} decoder. It indicates that the second phase of the \gls{lrl} decoder has improved the certainty of the bits through the information from the bits in the right of the codeword. This phenomenon can be seen in Figure~\ref{fig:eval_bd_lrl_iber} which plots the individual bit error probabilities for both decoders. Hence, the proposed \gls{lrl} decoder is better than the \gls{bd} in terms of decoding performance and complexity.
%\begin{figure}[htbp]
%  \centering
%  \tikzsetnextfilename{eval_bd_lrl_iber}
%  \includegraphics[width=0.9\linewidth]{plots/eval_bd_lrl_iber}
%  \caption{Individual bit error probability for \gls{bd} and \gls{lrl} decoder for $W=500$ and $\zeta=2.4$ dB.}
%  \label{fig:eval_bd_lrl_iber}
%\end{figure}

Figure~\ref{fig:eval_bd_lrl_bler_700} and Figure~\ref{fig:eval_bd_lrl_aneu_700} compares the performance and complexity between the \gls{bd} and \gls{lrl} decoder configuration-\rom{1} for window size $W=700$. We see that as the window size increases there is an improvement in \gls{bler} but no significant decrease in the complexity. It is because with larger windows, the \gls{bp}'s performance is better as seen in Figure~\ref{fig:eval_bd_bler}. So, as the window size increases the performance of both \gls{bd} and \gls{lrl} decoder approaches the performance of full-block decoder, i.e., a decoder with maximum $W$.
\begin{figure}[htbp]
  \centering
  \tikzsetnextfilename{eval_bd_lrl_bler_700}
  \includegraphics[width=0.9\linewidth]{plots/eval_bd_lrl_bler_700}
  \caption{Comparison of \gls{bler} between the Base Decoder and \gls{lrl} decoder with $W=700$.}
  \label{fig:eval_bd_lrl_bler_700}
\end{figure}
\begin{figure}[htbp]
  \centering
  \tikzsetnextfilename{eval_bd_lrl_aneu_700}
  \includegraphics[width=0.8\linewidth]{plots/eval_bd_lrl_aneu_700}
  \caption{Comparison of \gls{aneu} between the Base Decoder and \gls{lrl} decoder with $W=700$.}
  \label{fig:eval_bd_lrl_aneu_700}
\end{figure}

Next, we compare the performance of both window configurations of \gls{lrl} decoder. Table~\ref{tab:wind_conf} shows different window configurations used in simulation.
\begin{table}[htbp]
  \centering
  \begin{tabular}{|l|l|l|}
    \hline
    \textbf{Simulation Configuration} &\textbf{First Phase} &\text{Second Phase}\\
    \hline
    \hline
    \gls{lrl} configuration-\rom{1} &Window configuration-\rom{1} &Window configuration-\rom{1}\\
    \hline
    \gls{lrl} configuration-\rom{2} &Window configuration-\rom{1} &Window configuration-\rom{2}\\
    \hline
  \end{tabular}
  \caption{Experimental settings for simulations.}
  \label{tab:wind_conf}
\end{table}
Figure~\ref{fig:eval_comp_lrl_bler_300} compares the performance and Figure~\ref{fig:eval_comp_lrl_aneu_300} compares the complexity between both \gls{lrl} configurations. We see that the \gls{lrl} decoder with configuration-\rom{2} yields the same \gls{bler} performance as configuration-\rom{1} with a slightly reduced complexity. The \gls{ber} plot from Figure~\ref{fig:eval_comp_lrl_ber_300} also indicates the same. The reduced complexity is mainly because of the bottom to top \gls{cn} update in the second phase of the \gls{lrl} decoder configuration-\rom{2}. Also since the target \gls{vn} are in the right end of the window, the window converges faster than the configuration-\rom{1}.
\begin{figure}[htbp]
  \centering
  \tikzsetnextfilename{eval_comp_lrl_bler_300}
  \includegraphics[width=0.9\linewidth]{plots/eval_comp_lrl_bler_300}
  \caption{Comparison of \gls{bler} between the \gls{lrl} decoders configuration-\rom{1} and configuration-\rom{2}.}
  \label{fig:eval_comp_lrl_bler_300}
\end{figure}
\begin{figure}[htbp]
  \centering
  \tikzsetnextfilename{eval_comp_lrl_ber_300}
  \includegraphics[width=0.9\linewidth]{plots/eval_comp_lrl_ber_300}
  \caption{Comparison of \gls{ber} between the \gls{lrl} decoders configuration-\rom{1} and configuration-\rom{2}.}
  \label{fig:eval_comp_lrl_ber_300}
\end{figure}
\begin{figure}[htbp]
  \centering
  \tikzsetnextfilename{eval_comp_lrl_aneu_300}
  \includegraphics[width=0.8\linewidth]{plots/eval_comp_lrl_aneu_300}
  \caption{Comparison of \gls{aneu} between the \gls{lrl} decoders configuration-\rom{1} and configuration-\rom{2}.}
  \label{fig:eval_comp_lrl_aneu_300}
\end{figure}

\section{Evaluation of \acrlong{ipsc} Technique}
Here, we evaluate the performance of our \gls{ipsc} technique. Figure~\ref{fig:eval_ipsc_bler_300} and Figure~\ref{fig:eval_ipsc_aneu_300} compares the performance and complexity between the \gls{bd} with different convergence criteria with $W=300$. Similarly, Figure~\ref{fig:eval_ipsc_bler_600} and Figure~\ref{fig:eval_ipsc_aneu_600} compares for $W=600$.
\begin{figure}[htbp]
  \centering
  \tikzsetnextfilename{eval_ipsc_bler_300}
  \includegraphics[width=0.9\linewidth]{plots/eval_ipsc_bler_300}
  \caption{Comparison of \gls{bler} between different early-success criteria.}
  \label{fig:eval_ipsc_bler_300}
\end{figure}
\begin{figure}[htbp]
  \centering
  \tikzsetnextfilename{eval_ipsc_aneu_300}
  \includegraphics[width=0.9\linewidth]{plots/eval_ipsc_aneu_300}
  \caption{Comparison of \gls{aneu} between different early-success criteria.}
  \label{fig:eval_ipsc_aneu_300}
\end{figure}
\begin{figure}[htbp]
  \centering
  \tikzsetnextfilename{eval_ipsc_bler_600}
  \includegraphics[width=0.9\linewidth]{plots/eval_ipsc_bler_600}
  \caption{Comparison of \gls{bler} between different early-success criteria with $W=600$.}
  \label{fig:eval_ipsc_bler_600}
\end{figure}
\begin{figure}[htbp]
  \centering
  \tikzsetnextfilename{eval_ipsc_aneu_600}
  \includegraphics[width=0.9\linewidth]{plots/eval_ipsc_aneu_600}
  \caption{Comparison of \gls{aneu} between different early-success criteria with $W=600$.}
  \label{fig:eval_ipsc_aneu_600}
\end{figure}

From the Figures~\ref{fig:eval_ipsc_bler_300}, \ref{fig:eval_ipsc_aneu_300}, \ref{fig:eval_ipsc_bler_600} and \ref{fig:eval_ipsc_aneu_600} we see that for $W\geq2(m_s+1)$, checking only the target \glspl{cn} for convergence decreases the decoding complexity for $W>2m_s+1$. Hence, for convergence criteria it is better to use complete-\glspl{cn} if $W\leq2m_s+1$ and target-\glspl{cn} if $W>2m_s+1$.