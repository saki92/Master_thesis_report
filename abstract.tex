\chapter*{Abstract}
\addcontentsline{toc}{chapter}{Abstract}
A windowed decoder in its basic form converges rather slowly and has a large performance gap to a full-block decoder. In this work, we propose two techniques to improve the performance of windowed decoders for \glspl{ldpccc}. The first technique: the \acrshort{lrl} decoder, focuses on the movement direction of the window in which the window moves forward and backward across the \gls{pcm}. The second technique: the \acrshort{ipsc}, focuses on the convergence criterion for the windows where the criterion is dependent on window size. We choose the \glspl{ldpccc} specified in the standard IEEE 1901 \gls{bpl} to evaluate our techniques. We found that a proper end-termination for the \gls{bpl}'s \glspl{ldpccc} is infeasible. We show that although the termination procedure mentioned in the standard fails to reduces the \gls{cn} degree, the known termination bits at the decoder effectively reduce the \gls{cn} degree. Simulation results show that the \acrshort{lrl} decoder has a decoding performance gain of about $1.5$ dB while simultaneously reducing the decoding complexity by up to $28\%$. On the other hand, the \acrshort{ipsc} technique proves to reduce the decoding complexity for large window sizes by up to $24\%$.